%!TEX root=./pfc.tex
\chapter[Conclusiones y futuras mejoras]{\label{}
Conclusiones y futuras mejoras}

\section{Conclusiones}

El principal objetivo de este proyecto ha sido alcanzado puesto que se ha realizado un editor de código fuente en lenguaje C, colaborativo, e integrado dentro un sistema e-learning. Para este fin se ha conseguir diseñar un módulo compatible con Moodle en todas sus versiones.

De este modo, todos los objetivos referentes al hecho de crear un editor on-line se han cumplido con todas las características de un entorno de desarrollo. También se ha permitido al tutor la configuración de una serie de parámetros que son utilizados a la hora de compilar y desarrollar los códigos fuentes. Estos parámetros nos permiten compilar en cualquier OS que deseemos instalar el sistema así como tener una serie de estadísticas.

Una de las principales ventajas que se han obtenido con el desarrollo de este proyecto es que ahora los alumnos cuentan con una gran herramienta para programar, sin necesidad de tener conocimiento informático alguno y sin necesidad de tener que instalar nada en sus equipos. A la vez, fomenta la colaboración entre los compañeros a la hora de realizar cualquier ejercicio de programación e impide hacer el típico \emph{copy\&paste} para la realización de prácticas. Por otra parte, el profesor cuenta ahora con una herramienta más potente a la hora de evaluar los conocimientos en este área de su alumnado.

Además el back-end\footnote{\textbf{Back-end: }Parte de la aplicación encargada de administrar el sistema.} de la aplicación es totalmente independiente. De este modo, si se deseara exportar a otro sistema e-learning bastaría con diseñar un front-end dentro de ese sistema y enlazar con nuestro sistema. 

Otro de los objetivos cumplidos es la realización de un estudio de los sistemas e-learning más importantes en la actualidad analizando sus pros y sus contras así como realizando una explicación detallada de la elección de Moodle.

En cuanto al diseño de interfaz que se ha utilizado, como realmente lo que se ha desarrollado es un bloque para Moodle y este no tiene mucha complejidad, se ha seguido el mismo estilo gráfico de dicha plataforma. Sin embargo, también se han añadido algunos efectos visuales utilizando técnicas de diseño avanzadas como es el caso del framework JQuery.

\section{Posibles ampliaciones o mejoras}

A continuación se exponen una serie de ampliaciones o mejoras que podrían realizarse en un futuro sobre este sistema:

\begin{itemize}
	\item Se podría aumentar el número de lenguajes de programación disponibles. Ahora mismo el sistema sólo funciona para el lenguaje \textbf{C}, y eso puede limitar la exportación de este módulo para otras asignaturas en las que se enseñe algún otro lenguaje de programación. 
	\item Se puede permitir la realización de proyectos informáticos, dando la posibilidad de manejar varios ficheros en lugar de un único código fuente.
	\item Crear un sistema de importación y exportación de proyectos, de modo que el alumno pueda descargar todos los datos en su equipo para trabajar off-line.
	\item Crear un sistema de ejecución embebido dentro de la propia plataforma de desarrollo, así el alumno no necesitaría descargar el archivo en su equipo para ejecutarlo.
\end{itemize}