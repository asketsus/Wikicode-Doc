%!TEX root=./pfc.tex
\chapter[Módulo para Moodle 1.x]{\label{}
Módulo para Moodle 1.x}

\section{Front-End}

\subsection{create.php}
\begin{lstlisting}[language=PHP]
<?php

require_once('../../config.php');
require_once(dirname(__FILE__) . '/create_form.php');
require_once($CFG->dirroot . '/mod/wikicode/lib.php');
require_once($CFG->dirroot . '/mod/wikicode/locallib.php');
require_once($CFG->dirroot . '/mod/wikicode/pagelib.php');

// this page accepts two actions: new and create
// 'new' action will display a form contains page title and page format
// selections
// 'create' action will create a new page in db, and redirect to
// page editing page.
$action = optional_param('action', 'new', PARAM_TEXT);
// The title of the new page, can be empty
$title = optional_param('title', '', PARAM_TEXT);
$wid = optional_param('wid', 0, PARAM_INT);
$swid = optional_param('swid', 0, PARAM_INT);
$gid = optional_param('gid', 0, PARAM_INT);
$uid = optional_param('uid', 0, PARAM_INT);

// 'create' action must be submitted by moodle form
// so sesskey must be checked
if ($action == 'create') {
    if (!confirm_sesskey()) {
        print_error('invalidsesskey');
    }
}

if (!empty($swid)) {
    $subwiki = wikicode_get_subwiki($swid);
	$wid = $subwiki->wikiid;

    if (!$wiki = wikicode_get_wiki($subwiki->wikiid)) {
        print_error('invalidwikiid', 'wikicode');
    }

} else {
    $subwiki = wikicode_get_subwiki_by_group($wid, $gid, $uid);

    if (!$wiki = wikicode_get_wiki($wid)) {
        print_error('invalidwikiid', 'wikicode');
    }

}

if (!$cm = get_coursemodule_from_instance('wikicode', $wiki->id)) {
    print_error('invalidcoursemoduleid', 'wikicode');
}

$course = get_record('course', 'id', $cm->course);

require_login($course->id, true, $cm);

add_to_log($course->id, 'createpage', 'createpage', 'view.php?id=' . $cm->id, $wiki->id);

$wikipage = new page_wikicode_create($wiki, $subwiki, $cm);

if (!empty($swid)) {
    $wikipage->set_gid($subwiki->groupid);
    $wikipage->set_uid($subwiki->userid);
    $wikipage->set_swid($swid);
	$wikipage->set_wid($wid);
} else {
    $wikipage->set_wid($wid);
    $wikipage->set_gid($gid);
    $wikipage->set_uid($uid);
}

// set page action, and initialise moodle form

$wikipage->set_action($action);

switch ($action) {
case 'create':	
    $wikipage->create_page($title);
    break;
case 'new':
    if ((int)$wiki->forceformat == 1 && !empty($title)) {
        $wikipage->create_page($title);
    } else {
        // create link from moodle navigation block without pagetitle
        $wikipage->print_header($cm, $course);
        
        // new page without page title
        $wikipage->print_content($title);
    }
    print_footer($course);
    break;
}

\end{lstlisting}

\subsection{create\_form.php}
\begin{lstlisting}[language=PHP]
<?php

require_once($CFG->libdir.'/formslib.php');

class mod_wikicode_create_form extends moodleform {

     function definition() {
        global $CFG;
        $mform =& $this->_form;

        $formats = $this->_customdata['formats'];
        $defaultformat = $this->_customdata['defaultformat'];
        $forceformat = $this->_customdata['forceformat'];

        $mform->addElement('header', 'general', get_string('createpage', 'wikicode'));

        $textoptions = array();
        if (!empty($this->_customdata['disable_pagetitle'])) {
            $textoptions = array('readonly'=>'readonly');
        }
        $mform->addElement('text', 'pagetitle', get_string('newpagetitle', 'wikicode'), $textoptions);

        if ($forceformat) {
            $mform->addElement('hidden', 'pageformat', $defaultformat);
        } else {
            $mform->addElement('static', 'format', get_string('format', 'wikicode'));
            foreach ($formats as $format) {
                if ($format == $defaultformat) {
                    $attr = array('checked'=>'checked');
                }else if (!empty($forceformat)){
                    $attr = array('disabled'=>'disabled');
                } else {
                    $attr = array();
                }
                $mform->addElement('radio', 'pageformat', '', get_string('format'.$format, 'wikicode'), $format, $attr);
            }
        }

        //hiddens
        $mform->addElement('hidden', 'action');
        $mform->setDefault('action', 'create');

        $this->add_action_buttons(false, get_string('createpage', 'wikicode'));
    }
}
\end{lstlisting}

\subsection{diff.php}
\begin{lstlisting}[language=PHP]
<?php

require_once('../../config.php');

require_once($CFG->dirroot . '/mod/wikicode/lib.php');
require_once($CFG->dirroot . '/mod/wikicode/locallib.php');
require_once($CFG->dirroot . '/mod/wikicode/pagelib.php');

require_once($CFG->dirroot . '/mod/wikicode/diff/difflib.php');
require_once($CFG->dirroot . '/mod/wikicode/diff/diff_nwiki.php');

$pageid = required_param('pageid', PARAM_TEXT);
$compare = required_param('compare', PARAM_INT);
$comparewith = required_param('comparewith', PARAM_INT);

if (!$page = wikicode_get_page($pageid)) {
    print_error('incorrectpageid', 'wikicode');
}

if (!$subwiki = wikicode_get_subwiki($page->subwikiid)) {
    print_error('incorrectsubwikiid', 'wikicode');
}

if (!$wiki = wikicode_get_wiki($subwiki->wikiid)) {
    print_error('incorrectwikiid', 'wikicode');
}

if (!$cm = get_coursemodule_from_instance('wikicode', $wiki->id)) {
    print_error('invalidcoursemodule');
}

$course = get_record('course', 'id', $cm->course);

if ($compare >= $comparewith) {
    print_error("A page version can only be compared with an older version.");
}

require_login($course->id, true, $cm);
add_to_log($course->id, "wikicode", "diff", "diff.php?id=$cm->id", "$wiki->id");

$wikipage = new page_wikicode_diff($wiki, $subwiki, $cm);

$wikipage->set_page($page); 
$wikipage->set_comparison($compare, $comparewith);

$wikipage->print_header($cm, $course);

$wikipage->print_content();

print_footer($course);
\end{lstlisting}

\subsection{edit.php}
\begin{lstlisting}[language=PHP]
<?php

require_once('../../config.php');

require_once($CFG->dirroot . '/mod/wikicode/lib.php');
require_once($CFG->dirroot . '/mod/wikicode/locallib.php');
require_once($CFG->dirroot . '/mod/wikicode/pagelib.php');


$pageid = required_param('pageid', PARAM_INT);
$contentformat = optional_param('contentformat', '', PARAM_ALPHA);
$option = optional_param('editoption', '', PARAM_TEXT);
$section = optional_param('section', "", PARAM_TEXT);
$version = optional_param('version', -1, PARAM_INT);
$attachments = optional_param('attachments', 0, PARAM_INT);
$deleteuploads = optional_param('deleteuploads', 0, PARAM_RAW);
$compiled = optional_param('compiled', 0, PARAM_INT);

$newcontent = '';
if (!empty($newcontent) && is_array($newcontent)) {
    $newcontent = $newcontent['text'];
} 

if (!empty($option) && is_array($option)) {
    $option = $option['editoption'];
}

if (!$page = wikicode_get_page($pageid)) {
    print_error('incorrectpageid', 'wikicode');
}

if (!$subwiki = wikicode_get_subwiki($page->subwikiid)) {
    print_error('incorrectsubwikiid', 'wikicode');
}

if (!$wiki = wikicode_get_wiki($subwiki->wikiid)) {
    print_error('incorrectwikiid', 'wikicode');
}

if (!$cm = get_coursemodule_from_instance('wikicode', $wiki->id)) {
    print_error('invalidcoursemodule');
}

$course = get_record('course', 'id', $cm->course);

if (!empty($section) && !$sectioncontent = wikicode_get_section_page($page, $section)) {
    print_error('invalidsection', 'wikicode');
}

require_login($course, true, $cm);

$context = get_context_instance(CONTEXT_MODULE, $cm->id);
require_capability('mod/wikicode:editpage', $context);

add_to_log($course->id, 'wikicode', 'edit', "edit.php?id=$cm->id", "$wiki->id");

if ($option == get_string('save', 'wikicode')) {
    if (!confirm_sesskey()) {
        print_error(get_string('invalidsesskey', 'wikicode'));
    }
    $wikipage = new page_wikicode_save($wiki, $subwiki, $cm);
    $wikipage->set_page($page);
    $wikipage->set_newcontent($newcontent);
    $wikipage->set_upload(true);
} else {  
    if ($option == 'Compile' or $option == 'Download') {
        if (!confirm_sesskey()) {
            print_error(get_string('invalidsesskey', 'wikicode'));
        }
        $wikipage = new page_wikicode_compile($wiki, $subwiki, $cm);
        $wikipage->set_page($page);
		$wikipage->set_download(($option == 'Download'));
    }
	else {
        if ($option == get_string('cancel')) {
            //delete lock
            wikicode_delete_locks($page->id, $USER->id, $section);

            redirect($CFG->wwwroot . '/mod/wikicode/view.php?pageid=' . $pageid);
        } else {
            $wikipage = new page_wikicode_edit($wiki, $subwiki, $cm);
            $wikipage->set_page($page);
            $wikipage->set_upload($option == get_string('upload', 'wikicode'));
			$wikipage->set_compiled($compiled);
        }
    }

    if (has_capability('mod/wikicode:overridelock', $context)) {
        $wikipage->set_overridelock(true);
    }
}

if ($version >= 0) {
    $wikipage->set_versionnumber($version);
}

if (!empty($section)) {
    $wikipage->set_section($sectioncontent, $section);
}

if (!empty($attachments)) {
    $wikipage->set_attachments($attachments);
}

if (!empty($deleteuploads)) {
    $wikipage->set_deleteuploads($deleteuploads);
}

if (!empty($contentformat)) {
    $wikipage->set_format($contentformat);
}

$wikipage->print_header($cm, $course);

$wikipage->print_content();

print_footer($course);
\end{lstlisting}

\subsection{edit\_form.php}
\begin{lstlisting}[language=PHP]
<?php

if (!defined('MOODLE_INTERNAL')) {
    die('Direct access to this script is forbidden.');    ///  It must be included from a Moodle page
}

require_once($CFG->dirroot . '/mod/wikicode/editors/wikieditor.php');
require_once($CFG->dirroot . '/mod/wikicode/chat/wikichat.php');
require_once($CFG->dirroot . '/lib/formslib.php');

class mod_wikicode_edit_form extends moodleform {

    function definition() {
		global $CFG, $USER;
        $mform =& $this->_form;

        $version = $this->_customdata['version'];
        $format  = $this->_customdata['format'];
        $tags    = !isset($this->_customdata['tags'])?"":$this->_customdata['tags'];

        if ($format != 'html') {
            $contextid  = $this->_customdata['contextid'];
            $filearea   = $this->_customdata['filearea'];
            $fileitemid = $this->_customdata['fileitemid'];
        }

        if (isset($this->_customdata['pagetitle'])) {
            $pagetitle = get_string('editingpage', 'wikicode', $this->_customdata['pagetitle']);
        } else {
            $pagetitle = get_string('editing', 'wikicode');
        }

        //editor
        $mform->addElement('header', 'general', $pagetitle);

        $fieldname = get_string('format' . $format, 'wikicode');
        if ($format != 'html') {
            // Use wiki editor
			$buttoncommands=array();
			$buttoncommands[] =& $mform->createElement('button','editoption','Unlock', array('id' => 'btnunlock', 'class' => 'btnunlock'));
			$buttoncommands[] =& $mform->createElement('button','editoption','Refresh', array('id' => 'btnref', 'class' => 'btnref'));
			$buttoncommands[] =& $mform->createElement('submit', 'editoption', 'Save', array('id' => 'save'));
			$mform->addGroup($buttoncommands, 'editoption', 'Actions:', '', true);
            $mform->addElement('wikicodeeditor', 'newcontent', $fieldname, array('cols' => 150, 'rows' => 30, 'Wiki_format' => $format, 'files'=>$files));
        } else {
            $mform->addElement('editor', 'newcontent_editor', $fieldname, null, page_wikicode_edit::$attachmentoptions);
        }
		
		//chat
		$mform->addElement('header','chat','Chat');
		$mform->addElement('wikicodechat', 'wikicodechat', null, array('itemid'=>$fileitemid));
		
		//compiler
		$mform->addElement('header','compiler', 'Compiler');
		$mform->addElement('textarea', 'textCompiler', '', 'wrap="virtual" rows="3" cols="100" readonly="readonly" ');
		
		$buttonarray=array();
		$buttonarray[] =& $mform->createElement('submit','editoption','Compile', array('id' => 'compile'));
		$buttonarray[] =& $mform->createElement('submit','editoption','Download', array('id' => 'compile'));
		$mform->addGroup($buttonarray, 'editoption', 'Options:', '', true);

        //hiddens
        if ($version >= 0) {
            $mform->addElement('hidden', 'version');
            $mform->setDefault('version', $version);
        }

        $mform->addElement('hidden', 'contentformat');
        $mform->setDefault('contentformat', $format);
		
		$mform->addElement('hidden', 'insert');
		$mform->setDefault('insert', 1);

    }

}
\end{lstlisting}

\subsection{history.php}
\begin{lstlisting}[language=PHP]
<?php
require_once('../../config.php');

require_once($CFG->dirroot . '/mod/wikicode/lib.php');
require_once($CFG->dirroot . '/mod/wikicode/locallib.php');
require_once($CFG->dirroot . '/mod/wikicode/pagelib.php');

$pageid = required_param('pageid', PARAM_TEXT);
$paging = optional_param('page', 0, PARAM_INT);
$allversion = optional_param('allversion', 0, PARAM_INT);

if (!$page = wikicode_get_page($pageid)) {
    print_error('incorrectpageid', 'wikicode');
}

if (!$subwiki = wikicode_get_subwiki($page->subwikiid)) {
    print_error('incorrectsubwikiid', 'wikicode');
}

if (!$wiki = wikicode_get_wiki($subwiki->wikiid)) {
    print_error('incorrectwikiid', 'wikicode');
}

if (!$cm = get_coursemodule_from_instance('wikicode', $wiki->id)) {
    print_error('invalidcoursemodule');
}

$course = get_record('course', 'id', $cm->course);

require_login($course->id, true, $cm);
$context = get_context_instance(CONTEXT_MODULE, $cm->id);
require_capability('mod/wikicode:viewpage', $context);
add_to_log($course->id, 'wikicode', 'history', 'history.php?id=' . $cm->id, $wiki->id);

/// Print the page header
$wikipage = new page_wikicode_history($wiki, $subwiki, $cm);

$wikipage->set_page($page);
$wikipage->set_paging($paging);
$wikipage->set_allversion($allversion);

$wikipage->print_header($cm, $course);
$wikipage->print_content();

print_footer($course);
\end{lstlisting}

\subsection{index.php}
\begin{lstlisting}[language=PHP]
<?php

    require_once("../../config.php");
    require_once("lib.php");

    $id = required_param('id', PARAM_INT);   // course

    if (! $course = get_record("course", "id", $id)) {
        error("Course ID is incorrect");
    }

    require_course_login($course);

    add_to_log($course->id, "wikicode", "view all", "index.php?id=$course->id", "");


/// Get all required strings

    $strwikicodes = get_string("modulenameplural", "wikicode");
    $strwikicode  = get_string("modulename", "wikicode");


/// Print the header
    $navlinks = array();
    $navlinks[] = array('name' => $strwikicodes, 'link' => "index.php?id=$course->id", 'type' => 'activity');
    $navigation = build_navigation($navlinks);

    print_header_simple("$strwikicodes", "", $navigation, "", "", true, "", navmenu($course));

/// Get all the appropriate data

    if (! $wikicodes = get_all_instances_in_course("wikicode", $course)) {
        notice(get_string('thereareno', 'moodle', $strwikicodes), "../../course/view.php?id=$course->id");
        die;
    }

/// Print the list of instances (your module will probably extend this)

    $timenow = time();
    $strname  = 'Nombre';
    $strsummary = get_string('summary');
    $strtype = 'Tipo';
    $strlastmodified = 'Creacion';
    $strweek  = get_string('week');
    $strtopic  = get_string('topic');

    if ($course->format == "weeks") {
        $table->head  = array ($strweek, $strname, $strsummary, $strtype, $strlastmodified);
        $table->align = array ('CENTER', 'LEFT', 'LEFT', 'LEFT', 'LEFT');
    } else if ($course->format == "topics") {
        $table->head  = array ($strtopic, $strname, $strsummary, $strtype, $strlastmodified);
        $table->align = array ('CENTER', 'LEFT', 'LEFT', 'LEFT', 'LEFT');
    } else {
        $table->head  = array ($strname, $strsummary, $strtype, $strlastmodified);
        $table->align = array ('LEFT', 'LEFT', 'LEFT', 'LEFT');
    }

    foreach ($wikicodes as $wikicode) {
        if (!$wikicode->visible) {
            //Show dimmed if the mod is hidden
            $link = '<a class="dimmed" href="view.php?id='.$wikicode->coursemodule.'">'.format_string($wikicode->name,true).'</a>';
        } else {
            //Show normal if the mod is visible
            $link = '<a href="view.php?id='.$wikicode->coursemodule.'">'.format_string($wikicode->name,true).'</a>';
        }

        $timmod = '<span class="smallinfo">'.userdate($wikicode->timemodified).'</span>';
        $summary = '<div class="smallinfo">'.$wikicode->firstpagetitle.'</div>';

        $site = get_site();

        $wtype = '<span class="smallinfo">'.$wikicode->wikimode.'</span>';

        if ($course->format == "weeks" or $course->format == "topics") {
            $table->data[] = array ($wikicode->section, $link, $summary, $wtype, $timmod);
        } else {
            $table->data[] = array ($link, $summary, $wtype, $timmod);
        }
    }

    echo "<br />";

    print_table($table);

/// Finish the page

    print_footer($course);
\end{lstlisting}

\subsection{log.php}
\begin{lstlisting}[language=PHP]
<?php

require_once('../../config.php');

require_once($CFG->dirroot . '/mod/wikicode/lib.php');
require_once($CFG->dirroot . '/mod/wikicode/locallib.php');
require_once($CFG->dirroot . '/mod/wikicode/pagelib.php');

$pageid = required_param('pageid', PARAM_INT);
$section = optional_param('section', "", PARAM_TEXT);
$version = optional_param('version', -1, PARAM_INT);

if (!$page = wikicode_get_page($pageid)) {
    print_error('incorrectpageid', 'wikicode');
}

if (!$subwiki = wikicode_get_subwiki($page->subwikiid)) {
    print_error('incorrectsubwikiid', 'wikicode');
}

if (!$wiki = wikicode_get_wiki($subwiki->wikiid)) {
    print_error('incorrectwikiid', 'wikicode');
}

if (!$cm = get_coursemodule_from_instance('wikicode', $wiki->id)) {
    print_error('invalidcoursemodule');
}

$course = get_record('course', 'id', $cm->course);

if (!empty($section) && !$sectioncontent = wikicode_get_section_page($page, $section)) {
    print_error('invalidsection', 'wikicode');
}

require_login($course, true, $cm);

$context = get_context_instance(CONTEXT_MODULE, $cm->id);
require_capability('mod/wikicode:editpage', $context);

add_to_log($course->id, 'wikicode', 'log', "log.php?id=$cm->id", "$wiki->id");

$wikipage = new page_wikicode_log($wiki, $subwiki, $cm);

$wikipage->set_page($page);

$wikipage->print_header($cm, $course);

$wikipage->print_content();

print_footer($course);
\end{lstlisting}

\subsection{log\_form.php}
\begin{lstlisting}[language=PHP]
<?php

if (!defined('MOODLE_INTERNAL')) {
    die('Direct access to this script is forbidden.');    ///  It must be included from a Moodle page
}

require_once ($CFG->dirroot.'/lib/formslib.php');

class mod_wikicode_log_form extends moodleform {

    function definition() {
        global $CFG, $USER, $DB;

        $mform =& $this->_form;

        $version = $this->_customdata['version'];
        $format  = $this->_customdata['format'];
        $tags    = !isset($this->_customdata['tags'])?"":$this->_customdata['tags'];

        if ($format != 'html') {
            $contextid  = $this->_customdata['contextid'];
            $filearea   = $this->_customdata['filearea'];
            $fileitemid = $this->_customdata['fileitemid'];
        }

        if (isset($this->_customdata['pagetitle'])) {
            $pagetitle = get_string('logpage', 'wikicode', $this->_customdata['pagetitle']);
        } else {
            $pagetitle = get_string('loging', 'wikicode');
        }
		
		//Time
		$time = $this->_customdata['page']->timeendedit - $this->_customdata['page']->timestartedit;
		$seconds = $time % 60;
		$time = ($time - $seconds) / 60;
		$minutes = $time % 60;
		$hours = ($time - $minutes) / 60;		
		
		//Stats
		$attr = array('size' => '75', 'readonly' => 1);
		$mform->addElement('header','stats', 'Stats');
		$attr['value'] = $hours . " hours, " . $minutes . " minutes, " . $seconds . " seconds";
		$mform->addElement('text', 'timeedit', 'Editing Time', $attr);
		//$mform->addHelpButton('timeedit', 'timeedit', 'wikicode');
		$attr['value'] = $this->_customdata['page']->errorcompile;
		$mform->addElement('text', 'errorscompilation', 'Compilation Errors', $attr);
		//$mform->addHelpButton('errorscompilation', 'errorscompilation', 'wikicode');


        $mform->addElement('hidden', 'contentformat');
        $mform->setDefault('contentformat', $format);
		
		$mform->addElement('hidden', 'insert');
		$mform->setDefault('insert', 1);

    }

}
\end{lstlisting}

\subsection{mod\_form.php}
\begin{lstlisting}[language=PHP]
<?php

if (!defined('MOODLE_INTERNAL')) {
    die('Direct access to this script is forbidden.');    ///  It must be included from a Moodle page
}

require_once($CFG->dirroot . '/mod/wikicode/locallib.php');
require_once($CFG->libdir.'/formslib.php');
require_once($CFG->libdir.'/datalib.php');
require_once ($CFG->dirroot.'/course/moodleform_mod.php');

class mod_wikicode_mod_form extends moodleform_mod {

    function definition() {
        global $COURSE;
        $mform =& $this->_form;

        //-------------------------------------------------------------------------------
        /// Adding the "general" fieldset, where all the common settings are showed
        $mform->addElement('header', 'general', get_string('general', 'form'));
		
        /// Adding the standard "name" field
        $mform->addElement('text', 'name', "Wiki name", array('size' => '64'));
        $mform->setType('name', PARAM_TEXT);
        $mform->addRule('name', null, 'required', null, 'client');
        //-------------------------------------------------------------------------------
        /// Adding the rest of wiki settings, spreeading all them into this fieldset
        /// or adding more fieldsets ('header' elements) if needed for better logic

        $mform->addElement('header', 'wikifieldset', "Wiki Settings");

        $attr = array('size' => '20');
        if (!empty($this->_instance)) {
            $attr['disabled'] = 'disabled';
        } else {
            $attr['value'] = "First page name";
        }
        
	
        $mform->addElement('text', 'firstpagetitle', "First page name", $attr);

        if (empty($this->_instance)) {
            $mform->addRule('firstpagetitle', null, 'required', null, 'client');
        }

		$attr = array('size' => '180');
		
		$gccvalue = wikicode_get_gccpath();
		if ($gccvalue->gccpath != "") {
		    $attr['value'] = $gccvalue->gccpath;
		} else {
			$attr['value'] = 'gcc';
		}
		
		$mform->addElement('text', 'gccpath', 'Unix Compiler Path', $attr);
		
		$mingwvalue = wikicode_get_mingwpath();
		if ($mingwvalue->mingwpath != "") {
		    $attr['value'] = $mingwvalue->mingwpath;
		} else {
			$attr['value'] = 'mingw32-gcc';
		}
	
		$mform->addElement('text', 'mingwpath', 'Windows Compiler Path', $attr);

        $wikimodeoptions = array ('collaborative' => "Collaborative wiki", 'individual' => "Individual wiki");
        // don't allow to change wiki type once is set
        $wikitype_attr = array();
        if (!empty($this->_instance)) {
            $wikitype_attr['disabled'] = 'disabled';
        }
        $mform->addElement('select', 'wikimode', "Wiki mode", $wikimodeoptions, $wikitype_attr);

        $formats = wikicode_get_formats();
        $editoroptions = array();
        foreach ($formats as $format) {
            $editoroptions[$format] = get_string($format, 'wikicode');
        }
        $mform->addElement('select', 'defaultformat', get_string('defaultformat', 'wikicode'), $editoroptions);
        $mform->addElement('checkbox', 'forceformat', get_string('forceformat', 'wikicode'));
	
        //-------------------------------------------------------------------------------
        // add standard elements, common to all modules
        $this->standard_coursemodule_elements();
        //-------------------------------------------------------------------------------
        // add standard buttons, common to all modules
        $this->add_action_buttons();

    }
}
\end{lstlisting}

\subsection{pagelib.php}

\begin{lstlisting}[language=PHP]
<?php

require_once($CFG->dirroot . '/mod/wikicode/edit_form.php');
require_once($CFG->dirroot . '/mod/wikicode/log_form.php');
require_once($CFG->dirroot . '/tag/lib.php');
require_once($CFG->dirroot . '/lib/formslib.php');


/**
 * Class page_wikicode contains the common code between all pages
 *
 * @license   http://www.gnu.org/copyleft/gpl.html GNU GPL v3 or later
 */
abstract class page_wikicode {

    /**
     * @var object Current subwiki
     */
    protected $subwiki;

    /**
     * @var int Current page
     */
    protected $page;

    /**
     * @var string Current page title
     */
    protected $title;

    /**
     * @var int Current group ID
     */
    protected $gid;

    /**
     * @var object module context object
     */
    protected $modcontext;

    /**
     * @var int Current user ID
     */
    protected $uid;
    /**
     * @var array The tabs set used in wiki module
     */
    protected $tabs = array('view' => 'view', 'edit' => 'edit', 'history' => 'history', 'log' => 'log', 'admin' => 'admin');
    /**
     * @var array tabs options
     */
    protected $tabs_options = array();
    /**
     * @var object wiki renderer
     */
    protected $wikioutput;

    /**
     * page_wikicode constructor
     *
     * @param $wiki. Current wiki
     * @param $subwiki. Current subwiki.
     * @param $cm. Current course_module.
     */
    function __construct($wiki, $subwiki, $cm) {
    	
        global $PAGE, $CFG;
        $this->subwiki = $subwiki;
        $this->modcontext = get_context_instance(CONTEXT_MODULE, $cm->id);

    }

    /**
     * This method prints the top of the page.
     */
    function print_header($cm, $course) {
        global $OUTPUT, $PAGE, $CFG, $USER, $SESSION;

        if (isset($SESSION->wikipreviousurl) && is_array($SESSION->wikipreviousurl)) {
            $this->process_session_url();
        }
        $this->set_session_url();
        
        /// Print the page header

    	$strwikis = get_string("modulenameplural", "wikicode");
    	$strwiki  = get_string("modulename", "wikicode");

    	$navlinks = array();
		/// Add page name if not main page

        $navlinks[] = array('name' => format_string($this->title), 'link' => '', 'type' => 'title');


    	$navigation = build_navigation($navlinks, $cm);
    	print_header_simple($this->title, "", $navigation,
                "", "", $cacheme, update_module_button($cm->id, $course->id, $strwiki),
                navmenu($course, $cm));
		print_heading($this->title, 'center', 3);
    }

    /**
     * Protected method to print current page title.
     */
    protected function print_pagetitle() {
        global $OUTPUT;
        $html = '';
        $html .= $OUTPUT->container_start();
        $html .= $OUTPUT->heading(format_string($this->title), 2, 'wikicode_headingtitle');
        $html .= $OUTPUT->container_end();
        echo $html;
    }

    /**
     * Setup page tabs, if options is empty, will set up active tab automatically
     * @param array $options, tabs options
     */
    protected function setup_tabs($options = array()) {
        global $CFG, $PAGE;
        $groupmode = groups_get_activity_groupmode($PAGE->cm);

        if (empty($CFG->usecomments) || !has_capability('mod/wikicode:viewcomment', $PAGE->context)){
            unset($this->tabs['comments']);
        }

        if (!has_capability('mod/wikicode:editpage', $PAGE->context)){
            unset($this->tabs['edit']);
        }

        if ($groupmode and $groupmode == VISIBLEGROUPS) {
            $currentgroup = groups_get_activity_group($PAGE->cm);
            $manage = has_capability('mod/wikicode:managewiki', $PAGE->cm->context);
            $edit = has_capability('mod/wikicode:editpage', $PAGE->context);
            if (!$manage and !($edit and groups_is_member($currentgroup))) {
                unset($this->tabs['edit']);
            }
        } else {
            if (!has_capability('mod/wikicode:editpage', $PAGE->context)) {
                unset($this->tabs['edit']);
            }
        }


        if (empty($options)) {
            $this->tabs_options = array('activetab' => substr(get_class($this), 10));
        } else {
            $this->tabs_options = $options;
        }

    }

    /**
     * This method must be overwritten to print the page content.
     */
    function print_content() {
        throw new coding_exception('Page wiki class does not implement method print_content()');
    }

    /**
     * Method to set the current page
     *
     * @param object $page Current page
     */
    function set_page($page) {
        global $PAGE;

        $this->page = $page;
        $this->title = $page->title;
    }

    /**
     * Method to set the current page title.
     * This method must be called when the current page is not created yet.
     * @param string $title Current page title.
     */
    function set_title($title) {
        global $PAGE;

        $this->page = null;
        $this->title = $title;
    }

    /**
     * Method to set current group id
     * @param int $gid Current group id
     */
    function set_gid($gid) {
        $this->gid = $gid;
    }

    /**
     * Method to set current user id
     * @param int $uid Current user id
     */
    function set_uid($uid) {
        $this->uid = $uid;
    }

    /**
     * Method to set the URL of the page.
     * This method must be overwritten by every type of page.
     */
    protected function set_url() {
        throw new coding_exception('Page wiki class does not implement method set_url()');
    }

    /**
     * Protected method to create the common items of the navbar in every page type.
     */
    protected function create_navbar() {
    }

    /**
     * This method print the footer of the page.
     */
    function print_footer() {
        global $OUTPUT;
        echo $OUTPUT->footer();
    }

    protected function process_session_url() {
        global $USER, $SESSION;

        //delete locks if edit
        $url = $SESSION->wikipreviousurl;
        switch ($url['page']) {
        case 'edit':
            wikicode_delete_locks($url['params']['pageid'], $USER->id, $url['params']['section'], false);
            break;
        }
    }

    protected function set_session_url() {
        global $SESSION;
        unset($SESSION->wikipreviousurl);
    }
	
	function print_tab() {
		$tabs = array('view', 'edit','history','log');

        $tabrows = array();
        $row  = array();
        $currenttab = '';
        foreach ($tabs as $tab) {
            $tabname = $tab;
            $row[] = new tabobject($tabname, $ewbase.$tab.'.php?pageid='.$this->page->id, $tabname);
            if ($ewiki_action == "$tab" or in_array($page, $specialpages)) {
                $currenttab = $tabname;
            }
        }
        $tabrows[] = $row;

        print_tabs($tabrows, $currenttab);
	}

}

/**
 * View a wiki page
 *
 * @license   http://www.gnu.org/copyleft/gpl.html GNU GPL v3 or later
 */
class page_wikicode_view extends page_wikicode {
    /**
     * @var int the coursemodule id
     */
    private $coursemodule;

    function print_header($cm, $course) {

        parent::print_header($cm, $course);

        parent::print_tab();
		
		$wiki = wikicode_get_wikicode_from_pageid($this->page->id);
		
        $this->wikicode_print_subwiki_selector($wiki, $this->subwiki, $this->page, 'view');
    }
	
	function wikicode_print_subwiki_selector($wiki, $subwiki, $page, $pagetype = 'view') {
        global $CFG, $USER;
        switch ($pagetype) {
        case 'files':
            $baseurl = $CFG->wwwroot . '/mod/wikicode/files.php';
            break;
        case 'view':
        default:
            $baseurl = $CFG->wwwroot . '/mod/wikicode/view.php';
            break;
        }

        $cm = get_coursemodule_from_instance('wikicode', $wiki->id);
        $context = get_context_instance(CONTEXT_MODULE, $cm->id);
        // @TODO: A plenty of duplicated code below this lines.
        // Create private functions.
        switch (groups_get_activity_groupmode($cm)) {
        case NOGROUPS:
            if ($wiki->wikimode == 'collaborative') {
                // No need to print anything
                return;
            } else if ($wiki->wikimode == 'individual') {
                // We have private wikis here

                $view = has_capability('mod/wikicode:viewpage', $context);
                $manage = has_capability('mod/wikicode:managewiki', $context);

                // Only people with these capabilities can view all wikis
                if ($view && $manage) {
                    // @TODO: Print here a combo that contains all users.
                    $users = get_enrolled_users($context);
                    $options = array();
                    foreach ($users as $user) {
                        $options[$user->id] = fullname($user);
                    }

                    print_container_start();
                    if ($pagetype == 'files') {
                        $params['pageid'] = $page->id;
                    }
                    $baseurl = $baserurl . '?wid=' . $wiki->id . '&title=' . $page->title;
                    $name = 'uid';
                    $selected = $subwiki->userid;
                    print_container_end();
                }
                return;
            } else {
                // error
                return;
            }
        case SEPARATEGROUPS:
            if ($wiki->wikimode == 'collaborative') {
                // We need to print a select to choose a course group

                $params = array('wid'=>$wiki->id, 'title'=>$page->title);
                if ($pagetype == 'files') {
                    $params['pageid'] = $page->id;
                }
                $baseurl = $baserurl . '?wid=' . $wiki->id . '&title=' . $page->title;
				
                print_container_start(true);
				print_simple_box_start('center','70%','','20');
                groups_print_activity_menu($cm, $baseurl, false, true);
				print_simple_box_end();
                print_container_end();
                return;
            } else if ($wiki->wikimode == 'individual') {
                //  @TODO: Print here a combo that contains all users of that subwiki.
                $view = has_capability('mod/wikicode:viewpage', $context);
                $manage = has_capability('mod/wikicode:managewiki', $context);

                // Only people with these capabilities can view all wikis
                if ($view && $manage) {
                    $users = get_enrolled_users($context);
                    $options = array();
                    foreach ($users as $user) {
                        $groups = groups_get_all_groups($cm->course, $user->id);
                        if (!empty($groups)) {
                            foreach ($groups as $group) {
                                $options[$group->id][$group->name][$group->id . '-' . $user->id] = fullname($user);
                            }
                        } else {
                            $name = get_string('notingroup', 'wikicode');
                            $options[0][$name]['0' . '-' . $user->id] = fullname($user);
                        }
                    }
                } else {
                    $group = groups_get_group($subwiki->groupid);
                    $users = groups_get_members($subwiki->groupid);
                    foreach ($users as $user) {
                        $options[$group->id][$group->name][$group->id . '-' . $user->id] = fullname($user);
                    }
                }
                print_container_start();
                $params = array('wid' => $wiki->id, 'title' => $page->title);
                if ($pagetype == 'files') {
                    $params['pageid'] = $page->id;
                }
                $baseurl = $baserurl . '?wid=' . $wiki->id . '&title=' . $page->title;
                $name = 'groupanduser';
                $selected = $subwiki->groupid . '-' . $subwiki->userid;
                print_container_end();

                return;

            } else {
                // error
                return;
            }
        CASE VISIBLEGROUPS:
            if ($wiki->wikimode == 'collaborative') {
                // We need to print a select to choose a course group
                $params = array('wid'=>$wiki->id, 'title'=>urlencode($page->title));
                if ($pagetype == 'files') {
                    $params['pageid'] = $page->id;
                }
                $baseurl = $baserurl . '?wid=' . $wiki->id . '&title=' . $page->title;

                print_container_start();
                groups_print_activity_menu($cm, $baseurl);
                print_container_end();
                return;

            } else if ($wiki->wikimode == 'individual') {
                $users = get_enrolled_users($context);
                $options = array();
                foreach ($users as $user) {
                    $groups = groups_get_all_groups($cm->course, $user->id);
                    if (!empty($groups)) {
                        foreach ($groups as $group) {
                            $options[$group->id][$group->name][$group->id . '-' . $user->id] = fullname($user);
                        }
                    } else {
                        $name = get_string('notingroup', 'wikicode');
                        $options[0][$name]['0' . '-' . $user->id] = fullname($user);
                    }
                }

                print_container_start();
                $params = array('wid' => $wiki->id, 'title' => $page->title);
                if ($pagetype == 'files') {
                    $params['pageid'] = $page->id;
                }
                $baseurl = $baserurl . '?wid=' . $wiki->id . '&title=' . $page->title;
                $name = 'groupanduser';
                $selected = $subwiki->groupid . '-' . $subwiki->userid;
                print_container_end();

                return;

            } else {
                // error
                return;
            }
        default:
            // error
            return;

        }

    }

    function print_content() {
        global $PAGE, $CFG;

        if (wikicode_user_can_view($this->subwiki)) {

            if (!empty($this->page)) {
                wikicode_print_page_content($this->page, $this->modcontext, $this->subwiki->id);
                //$wiki = $PAGE->activityrecord;
            } else {
                print_string('nocontent', 'wikicode');
                // TODO: fix this part
                $swid = 0;
                if (!empty($this->subwiki)) {
                    $swid = $this->subwiki->id;
                }
            }
        } else {
            // @TODO: Tranlate it
            echo "You can not view this page";
        }
    }

    function set_url() {
        global $PAGE, $CFG;
        $params = array();

        if (isset($this->coursemodule)) {
            $params['id'] = $this->coursemodule;
        } else if (!empty($this->page) and $this->page != null) {
            $params['pageid'] = $this->page->id;
        } else if (!empty($this->gid)) {
            $params['wid'] = $PAGE->cm->instance;
            $params['group'] = $this->gid;
        } else if (!empty($this->title)) {
            $params['swid'] = $this->subwiki->id;
            $params['title'] = $this->title;
        } else {
            print_error(get_string('invalidparameters', 'wikicode'));
        }

        $PAGE->set_url($CFG->wwwroot . '/mod/wikicode/view.php', $params);
    }

    function set_coursemodule($id) {
        $this->coursemodule = $id;
    }

    protected function create_navbar() {
        global $PAGE, $CFG;

        $PAGE->navbar->add(format_string($this->title));
        $PAGE->navbar->add(get_string('view', 'wikicode'));
    }
}

/**
 * Wiki page editing page
 *
 * @license   http://www.gnu.org/copyleft/gpl.html GNU GPL v3 or later
 */
class page_wikicode_edit extends page_wikicode {

    public static $attachmentoptions;

    protected $sectioncontent;
    /** @var string the section name needed to be edited */
    protected $section;
    protected $overridelock = false;
    protected $versionnumber = -1;
    protected $upload = false;
    protected $attachments = 0;
    protected $deleteuploads = array();
    protected $format;
	protected $compiled = 0;

    function __construct($wiki, $subwiki, $cm) {
        global $CFG, $PAGE;
        parent::__construct($wiki, $subwiki, $cm);
    }

    protected function print_pagetitle() {
        global $OUTPUT;

        $title = $this->title;
        if (isset($this->section)) {
            $title .= ' : ' . $this->section;
        }
		echo "<script src=\"js/codemirror.js\" type=\"text/javascript\"></script>";
    }

    function print_header($cm, $course) {
        global $OUTPUT, $PAGE;

        parent::print_header($cm, $course);

        $this->print_pagetitle();
		parent::print_tab();
    }

    function print_content() {
        global $PAGE;

        if (wikicode_user_can_edit($this->subwiki)) {
            $this->print_edit(null, $compile);
        } else {
            // @TODO: Translate it
            echo "You can not edit this page";
        }
    }

    protected function set_url() {
        global $PAGE, $CFG;

        $params = array('pageid' => $this->page->id);

        if (isset($this->section)) {
            $params['section'] = $this->section;
        }

        $PAGE->set_url($CFG->wwwroot . '/mod/wikicode/edit.php', $params);
    }

    protected function set_session_url() {
        global $SESSION;

        $SESSION->wikipreviousurl = array('page' => 'edit', 'params' => array('pageid' => $this->page->id, 'section' => $this->section));
    }

    protected function process_session_url() {
    }

    function set_section($sectioncontent, $section) {
        $this->sectioncontent = $sectioncontent;
        $this->section = $section;
    }

    public function set_versionnumber($versionnumber) {
        $this->versionnumber = $versionnumber;
    }

    public function set_overridelock($override) {
        $this->overridelock = $override;
    }

    function set_format($format) {
        $this->format = $format;
    }

    public function set_upload($upload) {
        $this->upload = $upload;
    }

    public function set_attachments($attachments) {
        $this->attachments = $attachments;
    }

    public function set_deleteuploads($deleteuploads) {
        $this->deleteuploads = $deleteuploads;
    }
	
	public function set_compiled($compiled) {
		$this->compiled = $compiled;
	}

    protected function create_navbar() {
        global $PAGE, $CFG;

        parent::create_navbar();

        $PAGE->navbar->add(get_string('edit', 'wikicode'));
    }

    protected function check_locks() {
        global $OUTPUT, $USER, $CFG;

        return true;
    }

    protected function print_edit($content = null) {
        global $CFG, $OUTPUT, $USER, $PAGE;

        if (!$this->check_locks()) {
            return;
        }
		
        //delete old locks (> 1 hour)
        wikicode_delete_old_locks();

        $version = wikicode_get_current_version($this->page->id);
		$page = wikicode_get_page($this->page->id);
		
        $format = $version->contentformat;

        if ($content == null) {
            if (empty($this->section)) {
                $content = $version->content;
            } else {
                $content = $this->sectioncontent;
            }
        }
		
        $versionnumber = $version->version;
        if ($this->versionnumber >= 0) {
            if ($version->version != $this->versionnumber) {
                //print $OUTPUT->box(get_string('wrongversionlock', 'wikicode'), 'errorbox');
                $versionnumber = $this->versionnumber;
            }
        }
		
		

        $url = $CFG->wwwroot . '/mod/wikicode/edit.php?pageid=' . $this->page->id;
        if (!empty($this->section)) {
            $url .= "&section=" . urlencode($this->section);
        }

        $params = array('attachmentoptions' => page_wikicode_edit::$attachmentoptions, 'format' => $version->contentformat, 'version' => $versionnumber, 'pagetitle'=>$this->page->title);

        $data = new StdClass();
        $data->newcontent = wikicode_remove_tags_owner($content);
        $data->version = $versionnumber;
        $data->format = $format;

        switch ($format) {
        case 'html':
            $data->newcontentformat = FORMAT_HTML;
            // Append editor context to editor options, giving preference to existing context.
            page_wikicode_edit::$attachmentoptions = array_merge(array('context' => $this->modcontext), page_wikicode_edit::$attachmentoptions);
            $data = file_prepare_standard_editor($data, 'newcontent', page_wikicode_edit::$attachmentoptions, $this->modcontext, 'mod_wikicode', 'attachments', $this->subwiki->id);
            break;
        default:
            break;
            }

        if ($version->contentformat != 'html') {
            $params['fileitemid'] = $this->subwiki->id;
            $params['contextid']  = $this->modcontext->id;
            $params['component']  = 'mod_wikicode';
            $params['filearea']   = 'attachments';
        }

        if (!empty($CFG->usetags)) {
            $params['tags'] = tag_get_tags_csv('wikicode_pages', $this->page->id, TAG_RETURN_TEXT);
        }

        $form = new mod_wikicode_edit_form($url, $params);

        if ($formdata = $form->get_data()) {
            if (!empty($CFG->usetags)) {
                $data->tags = $formdata->tags;
            }
        } else {
            if (!empty($CFG->usetags)) {
                $data->tags = tag_get_tags_array('wikicode', $this->page->id);
            }
        }
		
		if ( $this->compiled == 1 ) {
			$data->newcontent = wikicode_remove_tags_owner($page->cachedcompile);
			$data->textCompiler = $page->cachedgcc;
		}
		
        $form->set_data($data);
        $form->display();
    }

}

/**
 * Wiki page editing page
 *
 * @license   http://www.gnu.org/copyleft/gpl.html GNU GPL v3 or later
 */
class page_wikicode_log extends page_wikicode {
	
	protected $sectioncontent;
    /** @var string the section name needed to be edited */
    protected $section;
    protected $overridelock = false;
    protected $versionnumber = -1;

    function __construct($wiki, $subwiki, $cm) {
        global $CFG, $PAGE;
        parent::__construct($wiki, $subwiki, $cm);
    }

    protected function print_pagetitle() {
        global $OUTPUT;

        $title = $this->title;
        if (isset($this->section)) {
            $title .= ' : ' . $this->section;
        }
        echo $OUTPUT->container_start('wikicode_clear');
        echo $OUTPUT->heading(format_string($title), 2, 'wikicode_headingtitle');
        echo $OUTPUT->container_end();
    }

    function print_header($cm, $course) {
        global $OUTPUT, $PAGE;

        parent::print_header($cm, $course);
		
		parent::print_tab();
    }

    function print_content() {
        global $PAGE;

        if (wikicode_user_can_edit($this->subwiki)) {
            $this->print_log();
        } else {
            // @TODO: Translate it
            echo "You can not edit this page";
        }
    }

    protected function set_url() {
        global $PAGE, $CFG;

        $params = array('pageid' => $this->page->id);

        if (isset($this->section)) {
            $params['section'] = $this->section;
        }

        $PAGE->set_url($CFG->wwwroot . '/mod/wikicode/log.php', $params);
    }

    protected function set_session_url() {
        global $SESSION;

        $SESSION->wikipreviousurl = array('page' => 'log', 'params' => array('pageid' => $this->page->id, 'section' => $this->section));
    }

    protected function process_session_url() {
    }

    protected function create_navbar() {
        global $PAGE, $CFG;

        parent::create_navbar();

        $PAGE->navbar->add(get_string('log', 'wikicode'));
    }

    protected function check_locks() {
        global $OUTPUT, $USER, $CFG;

        return true;
    }

    protected function print_log() {
        global $CFG, $OUTPUT, $USER, $PAGE;

        if (!$this->check_locks()) {
            return;
        }

        //delete old locks (> 1 hour)
        wikicode_delete_old_locks();

        $version = wikicode_get_current_version($this->page->id);
		$page = wikicode_get_page($this->page->id);
		
        $format = $version->contentformat;

        $params = array('attachmentoptions' => page_wikicode_edit::$attachmentoptions, 'format' => $version->contentformat, 'version' => $versionnumber, 'pagetitle'=>$this->page->title);

        $data = new StdClass();
		$params['page'] = $this->page;

        $form = new mod_wikicode_log_form($url, $params);
		
        $form->set_data($data);
        $form->display();
    }

}

/**
 * Class that models the behavior of wiki's view comments page
 *
 * @license   http://www.gnu.org/copyleft/gpl.html GNU GPL v3 or later
 */
class page_wikicode_comments extends page_wikicode {

    function print_header() {

        parent::print_header();

        $this->print_pagetitle();

    }

    function print_content() {
        global $CFG, $OUTPUT, $USER, $PAGE;
        require_once($CFG->dirroot . '/mod/wikicode/locallib.php');

        $page = $this->page;
        $subwiki = $this->subwiki;
        $wiki = $PAGE->activityrecord;
        list($context, $course, $cm) = get_context_info_array($this->modcontext->id);

        require_capability('mod/wikicode:viewcomment', $this->modcontext, NULL, true, 'noviewcommentpermission', 'wikicode');

        $comments = wikicode_get_comments($this->modcontext->id, $page->id);

        if (has_capability('mod/wikicode:editcomment', $this->modcontext)) {
            echo '<div class="midpad"><a href="' . $CFG->wwwroot . '/mod/wikicode/editcomments.php?action=add&amp;pageid=' . $page->id . '">' . get_string('addcomment', 'wikicode') . '</a></div>';
        }

        $options = array('swid' => $this->page->subwikiid, 'pageid' => $page->id);
        $version = wikicode_get_current_version($this->page->id);
        $format = $version->contentformat;

        if (empty($comments)) {
            echo $OUTPUT->heading(get_string('nocomments', 'wikicode'));
        }

        foreach ($comments as $comment) {

            $user = wikicode_get_user_info($comment->userid);

            $fullname = fullname($user, has_capability('moodle/site:viewfullnames', get_context_instance(CONTEXT_COURSE, $course->id)));
            $by = new stdclass();
            $by->name = '<a href="' . $CFG->wwwroot . '/user/view.php?id=' . $user->id . '&amp;course=' . $course->id . '">' . $fullname . '</a>';
            $by->date = userdate($comment->timecreated);

            $t = new html_table();
            $cell1 = new html_table_cell($OUTPUT->user_picture($user, array('popup' => true)));
            $cell2 = new html_table_cell(get_string('bynameondate', 'forum', $by));
            $cell3 = new html_table_cell();
            $cell3->atributtes ['width'] = "80%";
            $cell4 = new html_table_cell();
            $cell5 = new html_table_cell();

            $row1 = new html_table_row();
            $row1->cells[] = $cell1;
            $row1->cells[] = $cell2;
            $row2 = new html_table_row();
            $row2->cells[] = $cell3;

            if ($format != 'html') {
                if ($format == 'creole') {
                    $parsedcontent = wikicode_parse_content('creole', $comment->content, $options);
                } else if ($format == 'nwiki') {
                    $parsedcontent = wikicode_parse_content('nwiki', $comment->content, $options);
                }

                $cell4->text = format_text(html_entity_decode($parsedcontent['parsed_text']), FORMAT_HTML);
            } else {
                $cell4->text = format_text($comment->content, FORMAT_HTML);
            }

            $row2->cells[] = $cell4;

            $t->data = array($row1, $row2);

            $actionicons = false;
            if ((has_capability('mod/wikicode:managecomment', $this->modcontext))) {
                $urledit = new moodle_url('/mod/wikicode/editcomments.php', array('commentid' => $comment->id, 'pageid' => $page->id, 'action' => 'edit'));
                $urldelet = new moodle_url('/mod/wikicode/instancecomments.php', array('commentid' => $comment->id, 'pageid' => $page->id, 'action' => 'delete'));
                $actionicons = true;
            } else if ((has_capability('mod/wikicode:editcomment', $this->modcontext)) and ($USER->id == $user->id)) {
                $urledit = new moodle_url('/mod/wikicode/editcomments.php', array('commentid' => $comment->id, 'pageid' => $page->id, 'action' => 'edit'));
                $urldelet = new moodle_url('/mod/wikicode/instancecomments.php', array('commentid' => $comment->id, 'pageid' => $page->id, 'action' => 'delete'));
                $actionicons = true;
            }

            if ($actionicons) {
                $cell6 = new html_table_cell($OUTPUT->action_icon($urledit, new pix_icon('t/edit', get_string('edit'))) . $OUTPUT->action_icon($urldelet, new pix_icon('t/delete', get_string('delete'))));
                $row3 = new html_table_row();
                $row3->cells[] = $cell5;
                $row3->cells[] = $cell6;
                $t->data[] = $row3;
            }

            echo html_writer::tag('div', html_writer::table($t), array('class'=>'no-overflow'));

        }
    }

    function set_url() {
        global $PAGE, $CFG;
        $PAGE->set_url($CFG->wwwroot . '/mod/wikicode/comments.php', array('pageid' => $this->page->id));
    }

    protected function create_navbar() {
        global $PAGE, $CFG;

        parent::create_navbar();
        $PAGE->navbar->add(get_string('comments', 'wikicode'));
    }

}

/**
 * Class that models the behavior of wiki's edit comment
 *
 * @license   http://www.gnu.org/copyleft/gpl.html GNU GPL v3 or later
 */
class page_wikicode_editcomment extends page_wikicode {
    private $comment;
    private $action;
    private $form;
    private $format;

    function set_url() {
        global $PAGE, $CFG;
        $PAGE->set_url($CFG->wwwroot . '/mod/wikicode/comments.php', array('pageid' => $this->page->id));
    }

    function print_header() {
        parent::print_header();
        $this->print_pagetitle();
    }

    function print_content() {
        global $PAGE;

        require_capability('mod/wikicode:editcomment', $this->modcontext, NULL, true, 'noeditcommentpermission', 'wikicode');

        if ($this->action == 'add') {
            $this->add_comment_form();
        } else if ($this->action == 'edit') {
            $this->edit_comment_form($this->comment);
        }
    }

    function set_action($action, $comment) {
        global $CFG;
        require_once($CFG->dirroot . '/mod/wikicode/comments_form.php');

        $this->action = $action;
        $this->comment = $comment;
        $version = wikicode_get_current_version($this->page->id);
        $this->format = $version->contentformat;

        if ($this->format == 'html') {
            $destination = $CFG->wwwroot . '/mod/wikicode/instancecomments.php?pageid=' . $this->page->id;
            $this->form = new mod_wikicode_comments_form($destination);
        }
    }

    protected function create_navbar() {
        global $PAGE, $CFG;

        $PAGE->navbar->add(get_string('comments', 'wikicode'), $CFG->wwwroot . '/mod/wikicode/comments.php?pageid=' . $this->page->id);

        if ($this->action == 'add') {
            $PAGE->navbar->add(get_string('insertcomment', 'wikicode'));
        } else {
            $PAGE->navbar->add(get_string('editcomment', 'wikicode'));
        }
    }

    protected function setup_tabs() {
        parent::setup_tabs(array('linkedwhenactive' => 'comments', 'activetab' => 'comments'));
    }

    private function add_comment_form() {
        global $CFG;
        require_once($CFG->dirroot . '/mod/wikicode/editors/wiki_editor.php');

        $pageid = $this->page->id;

        if ($this->format == 'html') {
            $com = new stdClass();
            $com->action = 'add';
            $com->commentoptions = array('trusttext' => true, 'maxfiles' => 0);
            $this->form->set_data($com);
            $this->form->display();
        } else {
            wikicode_print_editor_wiki($this->page->id, null, $this->format, -1, null, false, null, 'addcomments');
        }
    }

    private function edit_comment_form($com) {
        global $CFG;
        require_once($CFG->dirroot . '/mod/wikicode/comments_form.php');
        require_once($CFG->dirroot . '/mod/wikicode/editors/wiki_editor.php');

        if ($this->format == 'html') {
            $com->action = 'edit';
            $com->entrycomment_editor['text'] = $com->content;
            $com->commentoptions = array('trusttext' => true, 'maxfiles' => 0);

            $this->form->set_data($com);
            $this->form->display();
        } else {
            wikicode_print_editor_wiki($this->page->id, $com->content, $this->format, -1, null, false, array(), 'editcomments', $com->id);
        }

    }

}

/**
 * Wiki page search page
 *
 * @license   http://www.gnu.org/copyleft/gpl.html GNU GPL v3 or later
 */
class page_wikicode_search extends page_wikicode {
    private $search_result;

    protected function create_navbar() {
        global $PAGE, $CFG;

        $PAGE->navbar->add(format_string($this->title));
    }

    function set_search_string($search, $searchcontent) {
        $swid = $this->subwiki->id;
        if ($searchcontent) {
            $this->search_result = wikicode_search_all($swid, $search);
        } else {
            $this->search_result = wikicode_search_title($swid, $search);
        }

    }

    function set_url() {
        global $PAGE, $CFG;
        $PAGE->set_url($CFG->wwwroot . '/mod/wikicode/search.php');
    }
    function print_content() {
        global $PAGE;

        require_capability('mod/wikicode:viewpage', $this->modcontext, NULL, true, 'noviewpagepermission', 'wikicode');

        echo $this->wikioutput->search_result($this->search_result, $this->subwiki);
    }
}

/**
 *
 * Class that models the behavior of wiki's
 * create page
 *
 */
class page_wikicode_create extends page_wikicode {

    private $format;
    private $swid;
    private $wid;
    private $action;
    private $mform;

    function print_header($cm, $course) {
        $this->set_url();
        parent::print_header($cm, $course);
    }

    function set_url() {
        global $PAGE, $CFG;

        $params = array();
        if ($this->action == 'new') {
            $params['action'] = 'new';
            $params['swid'] = $this->swid;
            $params['wid'] = $this->wid;
            if ($this->title != get_string('newpage', 'wikicode')) {
                $params['title'] = $this->title;
            }
            //$PAGE->set_url($CFG->wwwroot . '/mod/wikicode/create.php', $params);
        } else {
            $params['action'] = 'create';
            $params['swid'] = $this->swid;
            //$PAGE->set_url($CFG->wwwroot . '/mod/wikicode/create.php', $params);
        }
    }

    function set_format($format) {
        $this->format = $format;
    }

    function set_wid($wid) {
        $this->wid = $wid;
    }

    function set_swid($swid) {
        $this->swid = $swid;
    }

    function set_action($action) {
        global $PAGE;
        $this->action = $action;

        require_once(dirname(__FILE__) . '/create_form.php');
        $url = new moodle_url('./create.php', array('action' => 'create', 'wid' => $this->wid, 'gid' => $this->gid, 'uid' => $this->uid));
        $formats = wikicode_get_formats();
        $options = array('formats' => $formats, 'defaultformat' => $PAGE->activityrecord->defaultformat, 'forceformat' => $PAGE->activityrecord->forceformat);
        if ($this->title != get_string('newpage', 'wikicode')) {
            $options['disable_pagetitle'] = true;
        }
        $this->mform = new mod_wikicode_create_form(str_replace("amp;","",$url->out()), $options);
    }

    protected function create_navbar() {
        global $PAGE;

        $PAGE->navbar->add($this->title);
    }

    function print_content($pagetitle = '') {
 
        // @TODO: Change this to has_capability and show an alternative interface.
        require_capability('mod/wikicode:createpage', $this->modcontext, NULL, true, 'nocreatepermission', 'wikicode');
		
        $data = new stdClass();
        if (!empty($pagetitle)) {
            $data->pagetitle = $pagetitle;
        }
        $data->pageformat = "wcode";

        $this->mform->set_data($data);
        $this->mform->display();
    }

    function create_page($pagetitle) {
        global $USER, $CFG;
        $data = $this->mform->get_data();
        if (empty($this->subwiki)) {
            $swid = wikicode_add_subwiki($this->wid, $this->gid, $this->uid);
            $this->subwiki = wikicode_get_subwiki($swid);
        }
        if ($data) {
            $id = wikicode_create_page($this->subwiki->id, $data->pagetitle, $data->pageformat, $USER->id);
        } else {
            $id = wikicode_create_page($this->subwiki->id, $pagetitle, $PAGE->activityrecord->defaultformat, $USER->id);
        }
        redirect($CFG->wwwroot . '/mod/wikicode/edit.php?pageid=' . $id);
    }
}

/**
 * Class that models the behavior of wiki's
 * compile page
 *
 */
class page_wikicode_compile extends page_wikicode_edit {

    private $newcontent, $download;

    function print_header() {
    }

    function print_content() {
        $wiki = wikicode_get_wikicode_from_pageid($this->page->id);
    	$cm = get_coursemodule_from_instance('wikicode', $wiki->id);

        $context = get_context_instance(CONTEXT_MODULE, $cm->id);
        require_capability('mod/wikicode:editpage', $context, NULL, true, 'noeditpermission', 'wikicode');

        $this->print_compile();
    }

    function set_newcontent($newcontent) {
        $this->newcontent = $newcontent;
    }
	
	function set_download($download) {
		$this->download = $download;
	}

    protected function set_session_url() {
    }

    protected function print_compile() {
        global $CFG, $USER, $OUTPUT, $PAGE;

        $url = $CFG->wwwroot . '/mod/wikicode/edit.php?pageid=' . $this->page->id;
        if (!empty($this->section)) {
            $url .= "&section=" . urlencode($this->section);
        }

        $params = array('attachmentoptions' => page_wikicode_edit::$attachmentoptions, 'format' => $this->format, 'version' => $this->versionnumber);

        if ($this->format != 'html') {
            $params['fileitemid'] = $this->page->id;
            $params['contextid']  = $this->modcontext->id;
            $params['component']  = 'mod_wikicode';
            $params['filearea']   = 'attachments';
        }

        $form = new mod_wikicode_edit_form($url, $params);

        $save = false;
        $data = false;
        if ($data = $form->get_data()) {

            $save = wikicode_compile_page($this->page, $data->newcontent, $USER->id, $this->download);
      
            //deleting old locks
            wikicode_delete_locks($this->page->id, $USER->id, $this->section);

            redirect($CFG->wwwroot . '/mod/wikicode/edit.php?compiled=1&pageid=' . $this->page->id);
        } else {
            print_error('savingerror', 'wikicode');
        }
    }
}

/**
 *
 * Class that models the behavior of wiki's
 * view differences
 *
 */
class page_wikicode_diff extends page_wikicode {

    private $compare;
    private $comparewith;

    function print_header($cm, $course) {
        global $OUTPUT;

        parent::print_header($cm, $course);

        //$this->print_pagetitle();
        $vstring = new stdClass();
        $vstring->old = $this->compare;
        $vstring->new = $this->comparewith;
        echo get_string('comparewith', 'wikicode', $vstring);
    }

    /**
     * Print the diff view
     */
    function print_content() {
        global $PAGE;

        require_capability('mod/wikicode:viewpage', $this->modcontext, NULL, true, 'noviewpagepermission', 'wikicode');

        $this->print_diff_content();
    }

    function set_url() {
        global $PAGE, $CFG;

        $PAGE->set_url($CFG->wwwroot . '/mod/wikicode/diff.php', array('pageid' => $this->page->id, 'comparewith' => $this->comparewith, 'compare' => $this->compare));
    }

    function set_comparison($compare, $comparewith) {
        $this->compare = $compare;
        $this->comparewith = $comparewith;
    }

    protected function create_navbar() {
        global $PAGE, $CFG;

        parent::create_navbar();
        $PAGE->navbar->add(get_string('history', 'wikicode'), $CFG->wwwroot . '/mod/wikicode/history.php?pageid' . $this->page->id);
        $PAGE->navbar->add(get_string('diff', 'wikicode'));
    }

    protected function setup_tabs() {
        parent::setup_tabs(array('linkedwhenactive' => 'history', 'activetab' => 'history'));
    }

    /**
     * Given two versions of a page, prints a page displaying the differences between them.
     *
     * @global object $CFG
     * @global object $OUTPUT
     * @global object $PAGE
     */
    private function print_diff_content() {
        global $CFG, $OUTPUT, $PAGE;

        $pageid = $this->page->id;
        $total = wikicode_count_wikicode_page_versions($pageid) - 1;

        $oldversion = wikicode_get_wikicode_page_version($pageid, $this->compare);

        $newversion = wikicode_get_wikicode_page_version($pageid, $this->comparewith);

        if ($oldversion && $newversion) {

            $oldtext = format_text(wikicode_remove_tags($oldversion->content), FORMAT_PLAIN, array('overflowdiv'=>true));
			$newtext = format_text(wikicode_remove_tags($newversion->content), FORMAT_PLAIN, array('overflowdiv'=>true));
            list($diff1, $diff2) = ouwiki_diff_html($oldtext, $newtext);
            $oldversion->diff = $diff1;
            $oldversion->user = wikicode_get_user_info($oldversion->userid);
            $newversion->diff = $diff2;
            $newversion->user = wikicode_get_user_info($newversion->userid);
			
        } else {
            print_error('versionerror', 'wikicode');
        }
    }
}

/**
 *
 * Class that models the behavior of wiki's history page
 *
 */
class page_wikicode_history extends page_wikicode {
    /**
     * @var int $paging current page
     */
    private $paging;

    /**
     * @var int @rowsperpage Items per page
     */
    private $rowsperpage = 10;

    /**
     * @var int $allversion if $allversion != 0, all versions will be printed in a signle table
     */
    private $allversion;

    function __construct($wiki, $subwiki, $cm) {
        global $PAGE;
        parent::__construct($wiki, $subwiki, $cm);
        //$PAGE->requires->js_init_call('M.mod_wikicode.history', null, true);
    }

    function print_header($cm, $course) {
        parent::print_header($cm, $course);
        //$this->print_pagetitle();
        parent::print_tab();
    }

    function print_pagetitle() {
        global $OUTPUT;
        $html = '';

        $html .= $OUTPUT->container_start();
        $html .= $OUTPUT->heading_with_help(format_string($this->title), 'history', 'wikicode');
        $html .= $OUTPUT->container_end();
        echo $html;
    }

    function print_content() {
        global $PAGE;

        require_capability('mod/wikicode:viewpage', $this->modcontext, NULL, true, 'noviewpagepermission', 'wikicode');

        $this->print_history_content();
    }

    function set_url() {
        global $PAGE, $CFG;
        $PAGE->set_url($CFG->wwwroot . '/mod/wikicode/history.php', array('pageid' => $this->page->id));
    }

    function set_paging($paging) {
        $this->paging = $paging;
    }

    function set_allversion($allversion) {
        $this->allversion = $allversion;
    }

    protected function create_navbar() {
        global $PAGE, $CFG;

        parent::create_navbar();
        $PAGE->navbar->add(get_string('history', 'wikicode'));
    }

    /**
     * Prints the history for a given wiki page
     *
     * @global object $CFG
     * @global object $OUTPUT
     * @global object $PAGE
     */
    private function print_history_content() {
        global $CFG, $OUTPUT, $PAGE;

        $pageid = $this->page->id;
        $offset = $this->paging * $this->rowsperpage;
        // vcount is the latest version
        $vcount = wikicode_count_wikicode_page_versions($pageid) - 1;
        if ($this->allversion) {
            $versions = wikicode_get_wikicode_page_versions($pageid, 0, $vcount);
        } else {
            $versions = wikicode_get_wikicode_page_versions($pageid, $offset, $vcount);
        }
        // We don't want version 0 to be displayed
        // version 0 is blank page
        if (end($versions)->version == 0) {
            array_pop($versions);
        }

        $contents = array();

        $version0page = wikicode_get_wikicode_page_version($this->page->id, 0);
        $creator = wikicode_get_user_info($version0page->userid);
        $a = new StdClass;
        $a->date = userdate($this->page->timecreated, get_string('strftimedaydatetime', 'langconfig'));
        $a->username = fullname($creator);
	
        //echo $OUTPUT->heading(get_string('createddate', 'wikicode', $a), 4, 'wikicode_headingtime');
        if ($vcount > 0) {

            /// If there is only one version, we don't need radios nor forms
            if (count($versions) == 1) {
	
                $row = array_shift($versions);

                $username = wikicode_get_user_info($row->userid);
                $date = userdate($row->timecreated, get_string('strftimedate', 'langconfig'));
                $time = userdate($row->timecreated, get_string('strftimetime', 'langconfig'));
                $versionid = wikicode_get_version($row->id);
                $versionlink = $CFG->wwwroot . '/mod/wikicode/viewversion.php?pageid=' . $pageid . '&versionid=' . $versionid->id;
                $userlink = $CFG->wwwroot . '/user/view.php?id=' . $username->id;
                $contents[] = array('', html_writer::link($versionlink, $row->version), $picture . html_writer::link($userlink, fullname($username)), $time);

                $table = new html_table();
                $table->head = array('', get_string('version'), get_string('user'), get_string('modified'), '');
                $table->data = $contents;
                $table->attributes['class'] = 'mdl-align';
				$table->attributes['align'] = 'center';

                echo html_writer::table($table);

            } else {

                $checked = $vcount - $offset;
                $rowclass = array();

                foreach ($versions as $version) {
                    $user = wikicode_get_user_info($version->userid);
                    $date = userdate($version->timecreated, get_string('strftimedate'));
                    $rowclass[] = 'wikicode_histnewdate';
                    $time = userdate($version->timecreated, get_string('strftimetime', 'langconfig'));
                    $versionid = wikicode_get_version($version->id);
                    if ($versionid) {
                        $url = $CFG->wwwroot . '/mod/wikicode/viewversion.php?pageid=' . $pageid . '&versionid=' . $versionid->id;
                        $viewlink = html_writer::link($url, $version->version);
                    } else {
                        $viewlink = $version->version;
                    }
                    $userlink = new moodle_url($CFG->wwwroot . '/user/view.php', array('id' => $version->userid));
                    $contents[] = array($viewlink, $picture . html_writer::link($userlink->out(false), fullname($user)), $time, "");
                }

                $table = new html_table();

                //$icon = $OUTPUT->help_icon('diff', 'wikicode');

                $table->head = array(get_string('version'), get_string('user'), get_string('modified'), '');
                $table->data = $contents;
                $table->attributes['class'] = 'generaltable mdl-align';
				$table->attributes['align'] = 'center';
                $table->rowclasses = $rowclass;

                ///Print the form
				echo html_writer::start_tag('form', array('action'=>$CFG->wwwroot . '/mod/wikicode/diff.php?method=get&id=diff'));
                echo html_writer::tag('div', html_writer::empty_tag('input', array('type'=>'hidden', 'name'=>'pageid', 'value'=>$pageid)));
                echo html_writer::table($table);
                echo html_writer::start_tag('div', array('class'=>'mdl-align'));
                //echo html_writer::empty_tag('input', array('type'=>'submit', 'class'=>'wikicode_form-button', 'value'=>get_string('comparesel', 'wikicode')));
                echo html_writer::end_tag('div');
                echo html_writer::end_tag('form');

				//echo '<form action=' . new moodle_url('/mod/wikicode/diff.php') . ' id="diff" method="get">';
				
            }
        } else {
            print_string('nohistory', 'wikicode');
        }

        if (!$this->allversion) {
            //$pagingbar = moodle_paging_bar::make($vcount, $this->paging, $this->rowsperpage, $CFG->wwwroot.'/mod/wikicode/history.php?pageid='.$pageid.'&amp;');
            // $pagingbar->pagevar = $pagevar;
            //echo $OUTPUT->paging_bar($vcount, $this->paging, $this->rowsperpage, $CFG->wwwroot . '/mod/wikicode/history.php?pageid=' . $pageid . '&amp;');
            //print_paging_bar($vcount, $paging, $rowsperpage,$CFG->wwwroot.'/mod/wikicode/history.php?pageid='.$pageid.'&amp;','paging');
            } else {
            $link = new moodle_url('/mod/wikicode/history.php', array('pageid' => $pageid));
            //$OUTPUT->container(html_writer::link($link->out(false), get_string('viewperpage', 'wikicode', $this->rowsperpage)), 'mdl-align');
        }
        if ($vcount > $this->rowsperpage && !$this->allversion) {
            $link = new moodle_url('/mod/wikicode/history.php', array('pageid' => $pageid, 'allversion' => 1));
            //$OUTPUT->container(html_writer::link($link->out(false), get_string('viewallhistory', 'wikicode')), 'mdl-align');
        }
    }

    /**
     * Given an array of values, creates a group of radio buttons to be part of a form
     *
     * @param array  $options  An array of value-label pairs for the radio group (values as keys).
     * @param string $name     Name of the radiogroup (unique in the form).
     * @param string $onclick  Function to be executed when the radios are clicked.
     * @param string $checked  The value that is already checked.
     * @param bool   $return   If true, return the HTML as a string, otherwise print it.
     *
     * @return mixed If $return is false, returns nothing, otherwise returns a string of HTML.
     */
    private function choose_from_radio($options, $name, $onclick = '', $checked = '', $return = false) {

        static $idcounter = 0;

        if (!$name) {
            $name = 'unnamed';
        }

        $output = '<span class="radiogroup ' . $name . "\">\n";

        if (!empty($options)) {
            $currentradio = 0;
            foreach ($options as $value => $label) {
                $htmlid = 'auto-rb' . sprintf('%04d', ++$idcounter);
                $output .= ' <span class="radioelement ' . $name . ' rb' . $currentradio . "\">";
                $output .= '<input name="' . $name . '" id="' . $htmlid . '" type="radio" value="' . $value . '"';
                if ($value == $checked) {
                    $output .= ' checked="checked"';
                }
                if ($onclick) {
                    $output .= ' onclick="' . $onclick . '"';
                }
                if ($label === '') {
                    $output .= ' /> <label for="' . $htmlid . '">' . $value . '</label></span>' . "\n";
                } else {
                    $output .= ' /> <label for="' . $htmlid . '">' . $label . '</label></span>' . "\n";
                }
                $currentradio = ($currentradio + 1) % 2;
            }
        }

        $output .= '</span>' . "\n";

        if ($return) {
            return $output;
        } else {
            echo $output;
        }
    }
}

/**
 * Class that models the behavior of wiki's map page
 *
 */
class page_wikicode_map extends page_wikicode {

    /**
     * @var int wiki view option
     */
    private $view;

    function print_header() {
        parent::print_header();
        $this->print_pagetitle();
    }

    function print_content() {
        global $CFG, $PAGE;

        require_capability('mod/wikicode:viewpage', $this->modcontext, NULL, true, 'noviewpagepermission', 'wikicode');

        if ($this->view > 0) {
            //echo '<div><a href="' . $CFG->wwwroot . '/mod/wikicode/map.php?pageid=' . $this->page->id . '">' . get_string('backtomapmenu', 'wikicode') . '</a></div>';
        }

        switch ($this->view) {
        case 1:
            echo $this->wikioutput->menu_map($this->page->id, $this->view);
            $this->print_contributions_content();
            break;
        case 2:
            echo $this->wikioutput->menu_map($this->page->id, $this->view);
            $this->print_navigation_content();
            break;
        case 3:
            echo $this->wikioutput->menu_map($this->page->id, $this->view);
            $this->print_orphaned_content();
            break;
        case 4:
            echo $this->wikioutput->menu_map($this->page->id, $this->view);
            $this->print_index_content();
            break;
        case 5:
            echo $this->wikioutput->menu_map($this->page->id, $this->view);
            $this->print_page_list_content();
            break;
        case 6:
            echo $this->wikioutput->menu_map($this->page->id, $this->view);
            $this->print_updated_content();
            break;
        default:
            echo $this->wikioutput->menu_map($this->page->id, $this->view);
            $this->print_page_list_content();
        }
    }

    function set_view($option) {
        $this->view = $option;
    }

    function set_url() {
        global $PAGE, $CFG;
        $PAGE->set_url($CFG->wwwroot . '/mod/wikicode/map.php', array('pageid' => $this->page->id));
    }

    protected function create_navbar() {
        global $PAGE;

        parent::create_navbar();
        $PAGE->navbar->add(get_string('map', 'wikicode'));
    }

    /**
     * Prints the contributions tab content
     *
     * @uses $OUTPUT, $USER
     *
     */
    private function print_contributions_content() {
        global $CFG, $OUTPUT, $USER;
        $page = $this->page;

        if ($page->timerendered + wikicode_REFRESH_CACHE_TIME < time()) {
            $fresh = wikicode_refresh_cachedcontent($page);
            $page = $fresh['page'];
        }

        $swid = $this->subwiki->id;

        $table = new html_table();
        $table->head = array(get_string('contributions', 'wikicode') . $OUTPUT->help_icon('contributions', 'wikicode'));
        $table->attributes['class'] = 'wikicode_editor generalbox';
        $table->data = array();
        $table->rowclasses = array();

        $lastversions = array();
        $pages = array();
        $users = array();

        if ($contribs = wikicode_get_contributions($swid, $USER->id)) {
            foreach ($contribs as $contrib) {
                if (!array_key_exists($contrib->pageid, $pages)) {
                    $page = wikicode_get_page($contrib->pageid);
                    $pages[$contrib->pageid] = $page;
                } else {
                    continue;
                }

                if (!array_key_exists($page->id, $lastversions)) {
                    $version = wikicode_get_last_version($page->id);
                    $lastversions[$page->id] = $version;
                } else {
                    $version = $lastversions[$page->id];
                }

                if (!array_key_exists($version->userid, $users)) {
                    $user = wikicode_get_user_info($version->userid);
                    $users[$version->userid] = $user;
                } else {
                    $user = $users[$version->userid];
                }

                $link = wikicode_parser_link(format_string($page->title), array('swid' => $swid));
                $class = ($link['new']) ? 'class="wiki_newentry"' : '';

                $linkpage = '<a href="' . $link['url'] . '"' . $class . '>' . $link['content'] . '</a>';
                $icon = $OUTPUT->user_picture($user, array('popup' => true));

                $table->data[] = array("$icon&nbsp;$linkpage");
            }
        } else {
            $table->data[] = array(get_string('nocontribs', 'wikicode'));
        }
        echo html_writer::table($table);
    }

    /**
     * Prints the navigation tab content
     *
     * @uses $OUTPUT
     *
     */
    private function print_navigation_content() {
        global $OUTPUT;
        $page = $this->page;

        if ($page->timerendered + wikicode_REFRESH_CACHE_TIME < time()) {
            $fresh = wikicode_refresh_cachedcontent($page);
            $page = $fresh['page'];
        }

        $tolinks = wikicode_get_linked_to_pages($page->id);
        $fromlinks = wikicode_get_linked_from_pages($page->id);

        $table = new html_table();
        $table->attributes['class'] = 'wikicode_navigation_from';
        $table->head = array(get_string('navigationfrom', 'wikicode') . $OUTPUT->help_icon('navigationfrom', 'wikicode') . ':');
        $table->data = array();
        $table->rowclasses = array();
        foreach ($fromlinks as $link) {
            $lpage = wikicode_get_page($link->frompageid);
            $link = new moodle_url('/mod/wikicode/view.php', array('pageid' => $lpage->id));
            $table->data[] = array(html_writer::link($link->out(false), format_string($lpage->title)));
            $table->rowclasses[] = 'mdl-align';
        }

        $table_left = html_writer::table($table);

        $table = new html_table();
        $table->attributes['class'] = 'wikicode_navigation_to';
        $table->head = array(get_string('navigationto', 'wikicode') . $OUTPUT->help_icon('navigationto', 'wikicode') . ':');
        $table->data = array();
        $table->rowclasses = array();
        foreach ($tolinks as $link) {
            if ($link->tomissingpage) {
                $viewlink = new moodle_url('/mod/wikicode/create.php', array('swid' => $page->subwikiid, 'title' => $link->tomissingpage, 'action' => 'new'));
                $table->data[] = array(html_writer::link($viewlink->out(false), format_string($link->tomissingpage), array('class' => 'wikicode_newentry')));
            } else {
                $lpage = wikicode_get_page($link->topageid);
                $viewlink = new moodle_url('/mod/wikicode/view.php', array('pageid' => $lpage->id));
                $table->data[] = array(html_writer::link($viewlink->out(false), format_string($lpage->title)));
            }
            $table->rowclasses[] = 'mdl-align';
        }
        $table_right = html_writer::table($table);
        echo $OUTPUT->container($table_left . $table_right, 'wikicode_navigation_container');
    }

    /**
     * Prints the index page tab content
     *
     *
     */
    private function print_index_content() {
        global $OUTPUT;
        $page = $this->page;

        if ($page->timerendered + wikicode_REFRESH_CACHE_TIME < time()) {
            $fresh = wikicode_refresh_cachedcontent($page);
            $page = $fresh['page'];
        }

        $node = new navigation_node($page->title);

        $keys = array();
        $tree = array();
        $tree = wikicode_build_tree($page, $node, $keys);

        $table = new html_table();
        $table->head = array(get_string('pageindex', 'wikicode') . $OUTPUT->help_icon('pageindex', 'wikicode'));
        $table->attributes['class'] = 'wikicode_editor generalbox';
        $table->data[] = array($this->render_navigation_node($tree));

        echo html_writer::table($table);
    }

    /**
     * Prints the page list tab content
     *
     *
     */
    private function print_page_list_content() {
        global $OUTPUT;
        $page = $this->page;

        if ($page->timerendered + wikicode_REFRESH_CACHE_TIME < time()) {
            $fresh = wikicode_refresh_cachedcontent($page);
            $page = $fresh['page'];
        }

        $pages = wikicode_get_page_list($this->subwiki->id);

        $stdaux = new stdClass();
        $strspecial = get_string('special', 'wikicode');

        foreach ($pages as $page) {
            $letter = textlib::strtoupper(textlib::substr($page->title, 0, 1));
            if (preg_match('/[A-Z]/', $letter)) {
                $stdaux->{
                    $letter}
                [] = wikicode_parser_link($page);
            } else {
                $stdaux->{
                    $strspecial}
                [] = wikicode_parser_link($page);
            }
        }

        $table = new html_table();
        $table->head = array(get_string('pagelist', 'wikicode') . $OUTPUT->help_icon('pagelist', 'wikicode'));
        $table->attributes['class'] = 'wikicode_editor generalbox';
        $table->align = array('center');
        foreach ($stdaux as $key => $elem) {
            $table->data[] = array($key);
            foreach ($elem as $e) {
                $table->data[] = array(html_writer::link($e['url'], $e['content']));
            }
        }
        echo html_writer::table($table);
    }

    /**
     * Prints the orphaned tab content
     *
     *
     */
    private function print_orphaned_content() {
        global $OUTPUT;

        $page = $this->page;

        if ($page->timerendered + wikicode_REFRESH_CACHE_TIME < time()) {
            $fresh = wikicode_refresh_cachedcontent($page);
            $page = $fresh['page'];
        }

        $swid = $this->subwiki->id;

        $table = new html_table();
        $table->head = array(get_string('orphaned', 'wikicode') . $OUTPUT->help_icon('orphaned', 'wikicode'));
        $table->attributes['class'] = 'wikicode_editor generalbox';
        $table->data = array();
        $table->rowclasses = array();

        if ($orphanedpages = wikicode_get_orphaned_pages($swid)) {
            foreach ($orphanedpages as $page) {
                $link = wikicode_parser_link($page->title, array('swid' => $swid));
                $class = ($link['new']) ? 'class="wiki_newentry"' : '';
                $table->data[] = array('<a href="' . $link['url'] . '"' . $class . '>' . format_string($link['content']) . '</a>');
            }
        } else {
            $table->data[] = array(get_string('noorphanedpages', 'wikicode'));
        }

        echo html_writer::table($table);
    }

    /**
     * Prints the updated tab content
     *
     * @uses $COURSE, $OUTPUT
     *
     */
    private function print_updated_content() {
        global $COURSE, $OUTPUT;
        $page = $this->page;

        if ($page->timerendered + wikicode_REFRESH_CACHE_TIME < time()) {
            $fresh = wikicode_refresh_cachedcontent($page);
            $page = $fresh['page'];
        }

        $swid = $this->subwiki->id;

        $table = new html_table();
        $table->head = array(get_string('updatedpages', 'wikicode') . $OUTPUT->help_icon('updatedpages', 'wikicode'));
        $table->attributes['class'] = 'wikicode_editor generalbox';
        $table->data = array();
        $table->rowclasses = array();

        if ($pages = wikicode_get_updated_pages_by_subwiki($swid)) {
            $strdataux = '';
            foreach ($pages as $page) {
                $user = wikicode_get_user_info($page->userid);
                $strdata = strftime('%d %b %Y', $page->timemodified);
                if ($strdata != $strdataux) {
                    $table->data[] = array($OUTPUT->heading($strdata, 4));
                    $strdataux = $strdata;
                }
                $link = wikicode_parser_link($page->title, array('swid' => $swid));
                $class = ($link['new']) ? 'class="wiki_newentry"' : '';

                $linkpage = '<a href="' . $link['url'] . '"' . $class . '>' . format_string($link['content']) . '</a>';
                $icon = $OUTPUT->user_picture($user, array($COURSE->id));
                $table->data[] = array("$icon&nbsp;$linkpage");
            }
        } else {
            $table->data[] = array(get_string('noupdatedpages', 'wikicode'));
        }

        echo html_writer::table($table);
    }

    protected function render_navigation_node($items, $attrs = array(), $expansionlimit = null, $depth = 1) {

        // exit if empty, we don't want an empty ul element
        if (count($items) == 0) {
            return '';
        }

        // array of nested li elements
        $lis = array();
        foreach ($items as $item) {
            if (!$item->display) {
                continue;
            }
            $content = $item->get_content();
            $title = $item->get_title();
            if ($item->icon instanceof renderable) {
                $icon = $this->wikioutput->render($item->icon);
                $content = $icon . '&nbsp;' . $content; // use CSS for spacing of icons
                }
            if ($item->helpbutton !== null) {
                $content = trim($item->helpbutton) . html_writer::tag('span', $content, array('class' => 'clearhelpbutton'));
            }

            if ($content === '') {
                continue;
            }

            if ($item->action instanceof action_link) {
                //TODO: to be replaced with something else
                $link = $item->action;
                if ($item->hidden) {
                    $link->add_class('dimmed');
                }
                $content = $this->output->render($link);
            } else if ($item->action instanceof moodle_url) {
                $attributes = array();
                if ($title !== '') {
                    $attributes['title'] = $title;
                }
                if ($item->hidden) {
                    $attributes['class'] = 'dimmed_text';
                }
                $content = html_writer::link($item->action, $content, $attributes);

            } else if (is_string($item->action) || empty($item->action)) {
                $attributes = array();
                if ($title !== '') {
                    $attributes['title'] = $title;
                }
                if ($item->hidden) {
                    $attributes['class'] = 'dimmed_text';
                }
                $content = html_writer::tag('span', $content, $attributes);
            }

            // this applies to the li item which contains all child lists too
            $liclasses = array($item->get_css_type(), 'depth_' . $depth);
            if ($item->has_children() && (!$item->forceopen || $item->collapse)) {
                $liclasses[] = 'collapsed';
            }
            if ($item->isactive === true) {
                $liclasses[] = 'current_branch';
            }
            $liattr = array('class' => join(' ', $liclasses));
            // class attribute on the div item which only contains the item content
            $divclasses = array('tree_item');
            if ((empty($expansionlimit) || $item->type != $expansionlimit) && ($item->children->count() > 0 || ($item->nodetype == navigation_node::NODETYPE_BRANCH && $item->children->count() == 0 && isloggedin()))) {
                $divclasses[] = 'branch';
            } else {
                $divclasses[] = 'leaf';
            }
            if (!empty($item->classes) && count($item->classes) > 0) {
                $divclasses[] = join(' ', $item->classes);
            }
            $divattr = array('class' => join(' ', $divclasses));
            if (!empty($item->id)) {
                $divattr['id'] = $item->id;
            }
            $content = html_writer::tag('p', $content, $divattr) . $this->render_navigation_node($item->children, array(), $expansionlimit, $depth + 1);
            if (!empty($item->preceedwithhr) && $item->preceedwithhr === true) {
                $content = html_writer::empty_tag('hr') . $content;
            }
            $content = html_writer::tag('li', $content, $liattr);
            $lis[] = $content;
        }

        if (count($lis)) {
            return html_writer::tag('ul', implode("\n", $lis), $attrs);
        } else {
            return '';
        }
    }

}

/**
 * Class that models the behavior of wiki's restore version page
 *
 */
class page_wikicode_restoreversion extends page_wikicode {
    private $version;

    function print_header($cm, $course) {
        parent::print_header($cm, $course);
    }

    function print_content() {
        global $CFG, $PAGE;

        require_capability('mod/wikicode:managewiki', $this->modcontext, NULL, true, 'nomanagewikipermission', 'wikicode');

        $this->print_restoreversion();
    }

    function set_url() {
        global $PAGE, $CFG;
        $PAGE->set_url($CFG->wwwroot . '/mod/wikicode/viewversion.php', array('pageid' => $this->page->id, 'versionid' => $this->version->id));
    }

    function set_versionid($versionid) {
        $this->version = wikicode_get_version($versionid);
    }

    protected function create_navbar() {
        global $PAGE, $CFG;

        parent::create_navbar();
        $PAGE->navbar->add(get_string('restoreversion', 'wikicode'));
    }

    protected function setup_tabs() {
        parent::setup_tabs(array('linkedwhenactive' => 'history', 'activetab' => 'history'));
    }

    /**
     * Prints the restore version content
     *
     * @uses $CFG
     *
     * @param page $page The page whose version will be restored
     * @param int  $versionid The version to be restored
     * @param bool $confirm If false, shows a yes/no confirmation page.
     *     If true, restores the old version and redirects the user to the 'view' tab.
     */
    private function print_restoreversion() {
        global $CFG;

        $version = wikicode_get_version($this->version->id);
		
        $restoreurl = $CFG->wwwroot . '/mod/wikicode/restoreversion.php?confirm=1&pageid=' . $this->page->id . '&versionid=' . $version->id . '&sesskey=' . sesskey();
        $return = $CFG->wwwroot . '/mod/wikicode/viewversion.php?pageid=' . $this->page->id . '&versionid=' . $version->id;

        echo get_string('restoreconfirm', 'wikicode', $version->version);
        print_container_start(false, 'wikicode_restoreform');
        echo '<form class="wiki_restore_yes" action="' . $restoreurl . '" method="post" id="restoreversion">';
        echo '<div><input type="submit" name="confirm" value="' . get_string('yes') . '" /></div>';
        echo '</form>';
        echo '<form class="wiki_restore_no" action="' . $return . '" method="post">';
        echo '<div><input type="submit" name="norestore" value="' . get_string('no') . '" /></div>';
        echo '</form>';
        print_container_end();
    }
}
/**
 * Class that models the behavior of wiki's delete comment confirmation page
 *
 */
class page_wikicode_deletecomment extends page_wikicode {
    private $commentid;

    function print_header() {
        parent::print_header();
        $this->print_pagetitle();
    }

    function print_content() {
        $this->printconfirmdelete();
    }

    function set_url() {
        global $PAGE;
        $PAGE->set_url('/mod/wikicode/instancecomments.php', array('pageid' => $this->page->id, 'commentid' => $this->commentid));
    }

    public function set_action($action, $commentid, $content) {
        $this->action = $action;
        $this->commentid = $commentid;
        $this->content = $content;
    }

    protected function create_navbar() {
        global $PAGE;

        parent::create_navbar();
        $PAGE->navbar->add(get_string('deletecommentcheck', 'wikicode'));
    }

    protected function setup_tabs() {
        parent::setup_tabs(array('linkedwhenactive' => 'comments', 'activetab' => 'comments'));
    }

    /**
     * Prints the comment deletion confirmation form
     *
     * @param page $page The page whose version will be restored
     * @param int  $versionid The version to be restored
     * @param bool $confirm If false, shows a yes/no confirmation page.
     *     If true, restores the old version and redirects the user to the 'view' tab.
     */
    private function printconfirmdelete() {
        global $OUTPUT;

        $strdeletecheck = get_string('deletecommentcheck', 'wikicode');
        $strdeletecheckfull = get_string('deletecommentcheckfull', 'wikicode');

        //ask confirmation
        $optionsyes = array('confirm'=>1, 'pageid'=>$this->page->id, 'action'=>'delete', 'commentid'=>$this->commentid, 'sesskey'=>sesskey());
        $deleteurl = new moodle_url('/mod/wikicode/instancecomments.php', $optionsyes);
        $return = new moodle_url('/mod/wikicode/comments.php', array('pageid'=>$this->page->id));

        echo $OUTPUT->heading($strdeletecheckfull);
        print_container_start(false, 'wikicode_deletecommentform');
        echo '<form class="wiki_deletecomment_yes" action="' . $deleteurl . '" method="post" id="deletecomment">';
        echo '<div><input type="submit" name="confirmdeletecomment" value="' . get_string('yes') . '" /></div>';
        echo '</form>';
        echo '<form class="wiki_deletecomment_no" action="' . $return . '" method="post">';
        echo '<div><input type="submit" name="norestore" value="' . get_string('no') . '" /></div>';
        echo '</form>';
        print_container_end();
    }
}

/**
 * Class that models the behavior of wiki's
 * save page
 *
 */
class page_wikicode_save extends page_wikicode_edit {

    private $newcontent;

    function print_header() {
    }

    function print_content() {
		$wiki = wikicode_get_wikicode_from_pageid($this->page->id);
    	$cm = get_coursemodule_from_instance('wikicode', $wiki->id);

        $context = get_context_instance(CONTEXT_MODULE, $cm->id);
        require_capability('mod/wikicode:editpage', $context, NULL, true, 'noeditpermission', 'wikicode');

        $this->print_save();
    }

    function set_newcontent($newcontent) {
        $this->newcontent = $newcontent;
    }

    protected function set_session_url() {
    }

    protected function print_save() {
        global $CFG, $USER, $OUTPUT, $PAGE;

        $url = $CFG->wwwroot . '/mod/wikicode/edit.php?pageid=' . $this->page->id;
        if (!empty($this->section)) {
            $url .= "&section=" . urlencode($this->section);
        }

        $params = array('attachmentoptions' => page_wikicode_edit::$attachmentoptions, 'format' => $this->format, 'version' => $this->versionnumber);

        if ($this->format != 'html') {
            $params['fileitemid'] = $this->page->id;
            $params['contextid']  = $this->modcontext->id;
            $params['component']  = 'mod_wikicode';
            $params['filearea']   = 'attachments';
        }

        $form = new mod_wikicode_edit_form($url, $params);

        $save = false;
        $data = false;
        if ($data = $form->get_data()) {
            if ($this->format == 'html') {
                $data = file_postupdate_standard_editor($data, 'newcontent', page_wikicode_edit::$attachmentoptions, $this->modcontext, 'mod_wikicode', 'attachments', $this->subwiki->id);
            }

            if (isset($this->section)) {
                $save = wikicode_save_section($this->page, $this->section, $data->newcontent, $USER->id);
            } else {
                $save = wikicode_save_page($this->page, $data->newcontent, $USER->id);
            }
        }

        if ($save && $data) {
            if (!empty($CFG->usetags)) {
                tag_set('wikicode_pages', $this->page->id, $data->tags);
            }

            $message = '<p>' . get_string('saving', 'wikicode') . '</p>';

            if (!empty($save['sections'])) {
                foreach ($save['sections'] as $s) {
                    $message .= '<p>' . get_string('repeatedsection', 'wikicode', $s) . '</p>';
                }
            }

            if ($this->versionnumber + 1 != $save['version']) {
                $message .= '<p>' . get_string('wrongversionsave', 'wikicode') . '</p>';
            }

            if (isset($errors) && !empty($errors)) {
                foreach ($errors as $e) {
                    $message .= "<p>" . get_string('filenotuploadederror', 'wikicode', $e->get_filename()) . "</p>";
                }
            }

            //deleting old locks
            wikicode_delete_locks($this->page->id, $USER->id, $this->section);

            redirect($CFG->wwwroot . '/mod/wikicode/edit.php?pageid=' . $this->page->id);
        } else {
            print_error('savingerror', 'wikicode');
        }
    }
}

/**
 * Class that models the behavior of wiki's view an old version of a page
 *
 */
class page_wikicode_viewversion extends page_wikicode {

    private $version;

    function print_header($cm, $course) {
        parent::print_header($cm, $course);
    }

    function print_content() {
        global $PAGE;

        require_capability('mod/wikicode:viewpage', $this->modcontext, NULL, true, 'noviewpagepermission', 'wikicode');

        $this->print_version_view();
    }

    function set_url() {
        global $PAGE, $CFG;
        $PAGE->set_url($CFG->wwwroot . '/mod/wikicode/viewversion.php', array('pageid' => $this->page->id, 'versionid' => $this->version->id));
    }

    function set_versionid($versionid) {
        $this->version = wikicode_get_version($versionid);
    }

    protected function create_navbar() {
        global $PAGE, $CFG;

        parent::create_navbar();
        $PAGE->navbar->add(get_string('history', 'wikicode'), $CFG->wwwroot . '/mod/wikicode/history.php?pageid' . $this->page->id);
        $PAGE->navbar->add(get_string('versionnum', 'wikicode', $this->version->version));
    }

    protected function setup_tabs() {
        parent::setup_tabs(array('linkedwhenactive' => 'history', 'activetab' => 'history', 'inactivetabs' => array('edit')));
    }

    /**
     * Given an old page version, output the version content
     *
     * @global object $CFG
     * @global object $OUTPUT
     * @global object $PAGE
     */
    private function print_version_view() {
        global $CFG, $OUTPUT, $PAGE;
        $pageversion = wikicode_get_version($this->version->id);

        if ($pageversion) {
            $restorelink = $CFG->wwwroot . '/mod/wikicode/restoreversion.php?' . 'pageid=' . $this->page->id . '&versionid=' . $this->version->id;
            echo '<p>' . get_string('viewversion', 'wikicode', $pageversion->version) . '<br />' . html_writer::link($restorelink, '(' . get_string('restorethis', 'wikicode') . ')', array('class' => 'wikicode_restore')) . '&nbsp;' . '</p>';
            $userinfo = wikicode_get_user_info($pageversion->userid);
            $heading = '<p><strong>' . get_string('modified', 'wikicode') . ':</strong>&nbsp;' . userdate($pageversion->timecreated, get_string('strftimedatetime', 'langconfig'));
            $viewlink = $CFG->wwwroot . '/user/view.php?' . 'id=' . $userinfo->id;
            $heading .= '&nbsp;&nbsp;&nbsp;<strong>' . get_string('user') . ':</strong>&nbsp;' . html_writer::link($viewlink, fullname($userinfo));
            print_container($heading, false, 'mdl-align wikicode_modifieduser wikicode_headingtime');
            $options = array('swid' => $this->subwiki->id, 'pretty_print' => true, 'pageid' => $this->page->id);

            $pageversion->content = wikicode_remove_tags($pageversion->content);
    		$content = format_text($pageversion->content, FORMAT_PLAIN, array('overflowdiv'=>true));
			
			print_simple_box_start('center','70%','','20');
            print_box($content);
			print_simple_box_end();

        } else {
            print_error('versionerror', 'wikicode');
        }
    }
}

class page_wikicode_confirmrestore extends page_wikicode_save {

    private $version;

    function set_url() {
        global $PAGE, $CFG;
        $PAGE->set_url($CFG->wwwroot . '/mod/wikicode/viewversion.php', array('pageid' => $this->page->id, 'versionid' => $this->version->id));
    }

    function print_content() {
        global $CFG, $PAGE;

        require_capability('mod/wikicode:managewiki', $this->modcontext, NULL, true, 'nomanagewikipermission', 'wikicode');

        $version = wikicode_get_version($this->version->id);
        if (wikicode_restore_page($this->page, $version->content, $version->userid)) {
            redirect($CFG->wwwroot . '/mod/wikicode/view.php?pageid=' . $this->page->id, get_string('restoring', 'wikicode', $version->version), 3);
        } else {
            print_error('restoreerror', 'wikicode', $version->version);
        }
    }

    function set_versionid($versionid) {
        $this->version = wikicode_get_version($versionid);
    }
}

class page_wikicode_prettyview extends page_wikicode {

    function print_header() {
        global $CFG, $PAGE, $OUTPUT;
        $PAGE->set_pagelayout('embedded');
        echo $OUTPUT->header();

        echo '<h1 id="wiki_printable_title">' . format_string($this->title) . '</h1>';
    }

    function print_content() {
        global $PAGE;

        require_capability('mod/wikicode:viewpage', $this->modcontext, NULL, true, 'noviewpagepermission', 'wikicode');

        $this->print_pretty_view();
    }

    function set_url() {
        global $PAGE, $CFG;

        $PAGE->set_url($CFG->wwwroot . '/mod/wikicode/prettyview.php', array('pageid' => $this->page->id));
    }

    private function print_pretty_view() {
        $version = wikicode_get_current_version($this->page->id);

        $content = wikicode_parse_content($version->contentformat, $version->content, array('printable' => true, 'swid' => $this->subwiki->id, 'pageid' => $this->page->id, 'pretty_print' => true));

        echo '<div id="wiki_printable_content">';
        echo format_text($content['parsed_text'], FORMAT_HTML);
        echo '</div>';
    }
}

class page_wikicode_handlecomments extends page_wikicode {
    private $action;
    private $content;
    private $commentid;
    private $format;

    function print_header() {
        $this->set_url();
    }

    public function print_content() {
        global $CFG, $PAGE, $USER;

        if ($this->action == 'add') {
            if (has_capability('mod/wikicode:editcomment', $this->modcontext)) {
                $this->add_comment($this->content, $this->commentid);
            }
        } else if ($this->action == 'edit') {
            $comment = wikicode_get_comment($this->commentid);
            $edit = has_capability('mod/wikicode:editcomment', $this->modcontext);
            $owner = ($comment->userid == $USER->id);
            if ($owner && $edit) {
                $this->add_comment($this->content, $this->commentid);
            }
        } else if ($this->action == 'delete') {
            $comment = wikicode_get_comment($this->commentid);
            $manage = has_capability('mod/wikicode:managecomment', $this->modcontext);
            $owner = ($comment->userid == $USER->id);
            if ($owner || $manage) {
                $this->delete_comment($this->commentid);
                redirect($CFG->wwwroot . '/mod/wikicode/comments.php?pageid=' . $this->page->id, get_string('deletecomment', 'wikicode'), 2);
            }
        }

    }

    public function set_url() {
        global $PAGE, $CFG;
        $PAGE->set_url($CFG->wwwroot . '/mod/wikicode/comments.php', array('pageid' => $this->page->id));
    }

    public function set_action($action, $commentid, $content) {
        $this->action = $action;
        $this->commentid = $commentid;
        $this->content = $content;

        $version = wikicode_get_current_version($this->page->id);
        $format = $version->contentformat;

        $this->format = $format;
    }

    private function add_comment($content, $idcomment) {
        global $CFG, $PAGE;
        require_once($CFG->dirroot . "/mod/wikicode/locallib.php");

        $pageid = $this->page->id;

        wikicode_add_comment($this->modcontext, $pageid, $content, $this->format);

        if (!$idcomment) {
            redirect($CFG->wwwroot . '/mod/wikicode/comments.php?pageid=' . $pageid, get_string('createcomment', 'wikicode'), 2);
        } else {
            $this->delete_comment($idcomment);
            redirect($CFG->wwwroot . '/mod/wikicode/comments.php?pageid=' . $pageid, get_string('editingcomment', 'wikicode'), 2);
        }
    }

    private function delete_comment($commentid) {
        global $CFG, $PAGE;

        $pageid = $this->page->id;

        wikicode_delete_comment($commentid, $this->modcontext, $pageid);
    }

}

class page_wikicode_lock extends page_wikicode_edit {

    public function print_header() {
        $this->set_url();
    }

    protected function set_url() {
        global $PAGE, $CFG;

        $params = array('pageid' => $this->page->id);

        if ($this->section) {
            $params['section'] = $this->section;
        }

        $PAGE->set_url($CFG->wwwroot . '/mod/wikicode/lock.php', $params);
    }

    protected function set_session_url() {
    }

    public function print_content() {
        global $USER, $PAGE;

        require_capability('mod/wikicode:editpage', $this->modcontext, NULL, true, 'noeditpermission', 'wikicode');

        wikicode_set_lock($this->page->id, $USER->id, $this->section);
    }

    public function print_footer() {
    }
}

class page_wikicode_overridelocks extends page_wikicode_edit {
    function print_header() {
        $this->set_url();
    }

    function print_content() {
        global $CFG, $PAGE;

        require_capability('mod/wikicode:overridelock', $this->modcontext, NULL, true, 'nooverridelockpermission', 'wikicode');

        wikicode_delete_locks($this->page->id, null, $this->section, true, true);

        $args = "pageid=" . $this->page->id;

        if (!empty($this->section)) {
            $args .= "&section=" . urlencode($this->section);
        }

        redirect($CFG->wwwroot . '/mod/wikicode/edit.php?' . $args, get_string('overridinglocks', 'wikicode'), 2);
    }

    function set_url() {
        global $PAGE, $CFG;

        $params = array('pageid' => $this->page->id);

        if (!empty($this->section)) {
            $params['section'] = $this->section;
        }

        $PAGE->set_url($CFG->wwwroot . '/mod/wikicode/overridelocks.php', $params);
    }

    protected function set_session_url() {
    }

    private function print_overridelocks() {
        global $CFG;

        wikicode_delete_locks($this->page->id, null, $this->section, true, true);

        $args = "pageid=" . $this->page->id;

        if (!empty($this->section)) {
            $args .= "&section=" . urlencode($this->section);
        }

        redirect($CFG->wwwroot . '/mod/wikicode/edit.php?' . $args, get_string('overridinglocks', 'wikicode'), 2);
    }

}

/**
 * This class will let user to delete wiki pages and page versions
 *
 */
class page_wikicode_admin extends page_wikicode {

    public $view, $action;
    public $listorphan = false;

    /**
     * Constructor
     *
     * @global object $PAGE
     * @param mixed $wiki instance of wiki
     * @param mixed $subwiki instance of subwiki
     * @param stdClass $cm course module
     */
    function __construct($wiki, $subwiki, $cm) {
        global $PAGE;
        parent::__construct($wiki, $subwiki, $cm);
        $PAGE->requires->js_init_call('M.mod_wikicode.deleteversion', null, true);
    }

    /**
     * Prints header for wiki page
     */
    function print_header() {
        parent::print_header();
        $this->print_pagetitle();
    }

    /**
     * This function will display administration view to users with managewiki capability
     */
    function print_content() {
        //make sure anyone trying to access this page has managewiki capabilities
        require_capability('mod/wikicode:managewiki', $this->modcontext, NULL, true, 'noviewpagepermission', 'wikicode');

        //update wiki cache if timedout
        $page = $this->page;
        if ($page->timerendered + wikicode_REFRESH_CACHE_TIME < time()) {
            $fresh = wikicode_refresh_cachedcontent($page);
            $page = $fresh['page'];
        }

        //dispaly admin menu
        echo $this->wikioutput->menu_admin($this->page->id, $this->view);

        //Display appropriate admin view
        switch ($this->view) {
            case 1: //delete page view
                $this->print_delete_content($this->listorphan);
                break;
            case 2: //delete version view
                $this->print_delete_version();
                break;
            default: //default is delete view
                $this->print_delete_content($this->listorphan);
                break;
        }
    }

    /**
     * Sets admin view option
     *
     * @param int $view page view id
     * @param bool $listorphan is only valid for view 1.
     */
    public function set_view($view, $listorphan = true) {
        $this->view = $view;
        $this->listorphan = $listorphan;
    }

    /**
     * Sets page url
     *
     * @global object $PAGE
     * @global object $CFG
     */
    function set_url() {
        global $PAGE, $CFG;
        $PAGE->set_url($CFG->wwwroot . '/mod/wikicode/admin.php', array('pageid' => $this->page->id));
    }

    /**
     * sets navigation bar for the page
     *
     * @global object $PAGE
     */
    protected function create_navbar() {
        global $PAGE;

        parent::create_navbar();
        $PAGE->navbar->add(get_string('admin', 'wikicode'));
    }

    /**
     * Show wiki page delete options
     *
     * @param bool $showorphan
     */
    protected function print_delete_content($showorphan = true) {
        $contents = array();
        $table = new html_table();
        $table->head = array('','Page name');
        $table->attributes['class'] = 'generaltable mdl-align';
        $swid = $this->subwiki->id;
        if ($showorphan) {
            if ($orphanedpages = wikicode_get_orphaned_pages($swid)) {
                $this->add_page_delete_options($orphanedpages, $swid, $table);
            } else {
                $table->data[] = array('', get_string('noorphanedpages', 'wikicode'));
            }
        } else {
            if ($pages = wikicode_get_page_list($swid)) {
                $this->add_page_delete_options($pages, $swid, $table);
            } else {
                $table->data[] = array('', get_string('nopages', 'wikicode'));
            }
        }

        ///Print the form
        echo html_writer::start_tag('form', array(
                                                'action' => new moodle_url('/mod/wikicode/admin.php'),
                                                'method' => 'post'));
        echo html_writer::tag('div', html_writer::empty_tag('input', array(
                                                                         'type'  => 'hidden',
                                                                         'name'  => 'pageid',
                                                                         'value' => $this->page->id)));

        echo html_writer::empty_tag('input', array('type' => 'hidden', 'name' => 'option', 'value' => $this->view));
        echo html_writer::table($table);
        echo html_writer::start_tag('div', array('class' => 'mdl-align'));
        if (!$showorphan) {
            echo html_writer::empty_tag('input', array(
                                                     'type'    => 'submit',
                                                     'class'   => 'wikicode_form-button',
                                                     'value'   => get_string('listorphan', 'wikicode'),
                                                     'sesskey' => sesskey()));
        } else {
            echo html_writer::empty_tag('input', array('type'=>'hidden', 'name'=>'listall', 'value'=>'1'));
            echo html_writer::empty_tag('input', array(
                                                     'type'    => 'submit',
                                                     'class'   => 'wikicode_form-button',
                                                     'value'   => get_string('listall', 'wikicode'),
                                                     'sesskey' => sesskey()));
        }
        echo html_writer::end_tag('div');
        echo html_writer::end_tag('form');
    }

    /**
     * helper function for print_delete_content. This will add data to the table.
     *
     * @global object $OUTPUT
     * @param array $pages objects of wiki pages in subwiki
     * @param int $swid id of subwiki
     * @param object $table reference to the table in which data needs to be added
     */
    protected function add_page_delete_options($pages, $swid, &$table) {
        global $OUTPUT;
        foreach ($pages as $page) {
            $link = wikicode_parser_link($page->title, array('swid' => $swid));
            $class = ($link['new']) ? 'class="wiki_newentry"' : '';
            $pagelink = '<a href="' . $link['url'] . '"' . $class . '>' . format_string($link['content']) . '</a>';
            $urledit = new moodle_url('/mod/wikicode/edit.php', array('pageid' => $page->id, 'sesskey' => sesskey()));
            $urldelete = new moodle_url('/mod/wikicode/admin.php', array(
                                                                   'pageid'  => $this->page->id,
                                                                   'delete'  => $page->id,
                                                                   'option'  => $this->view,
                                                                   'listall' => !$this->listorphan?'1': '',
                                                                   'sesskey' => sesskey()));

            $editlinks = $OUTPUT->action_icon($urledit, new pix_icon('t/edit', get_string('edit')));
            $editlinks .= $OUTPUT->action_icon($urldelete, new pix_icon('t/delete', get_string('delete')));
            $table->data[] = array($editlinks, $pagelink);
        }
    }

    /**
     * Prints lists of versions which can be deleted
     *
     * @global object $OUTPUT
     */
    private function print_delete_version() {
        global $OUTPUT;
        $pageid = $this->page->id;

        // versioncount is the latest version
        $versioncount = wikicode_count_wikicode_page_versions($pageid) - 1;
        $versions = wikicode_get_wikicode_page_versions($pageid, 0, $versioncount);

        // We don't want version 0 to be displayed
        // version 0 is blank page
        if (end($versions)->version == 0) {
            array_pop($versions);
        }

        $contents = array();
        $version0page = wikicode_get_wikicode_page_version($this->page->id, 0);
        $creator = wikicode_get_user_info($version0page->userid);
        $a = new stdClass();
        $a->date = userdate($this->page->timecreated, get_string('strftimedaydatetime', 'langconfig'));
        $a->username = fullname($creator);
        echo $OUTPUT->heading(get_string('createddate', 'wikicode', $a), 4, 'wikicode_headingtime');
        if ($versioncount > 0) {
            /// If there is only one version, we don't need radios nor forms
            if (count($versions) == 1) {
                $row = array_shift($versions);
                $username = wikicode_get_user_info($row->userid);
                $picture = $OUTPUT->user_picture($username);
                $date = userdate($row->timecreated, get_string('strftimedate', 'langconfig'));
                $time = userdate($row->timecreated, get_string('strftimetime', 'langconfig'));
                $versionid = wikicode_get_version($row->id);
                $versionlink = new moodle_url('/mod/wikicode/viewversion.php', array('pageid' => $pageid, 'versionid' => $versionid->id));
                $userlink = new moodle_url('/user/view.php', array('id' => $username->id));
                $picturelink = $picture . html_writer::link($userlink->out(false), fullname($username));
                $historydate = $OUTPUT->container($date, 'wikicode_histdate');
                $contents[] = array('', html_writer::link($versionlink->out(false), $row->version), $picturelink, $time, $historydate);

                //Show current version
                $table = new html_table();
                $table->head = array('', get_string('version'), get_string('user'), get_string('modified'), '');
                $table->data = $contents;
                $table->attributes['class'] = 'mdl-align';

                echo html_writer::table($table);
            } else {
                $lastdate = '';
                $rowclass = array();

                foreach ($versions as $version) {
                    $user = wikicode_get_user_info($version->userid);
                    $picture = $OUTPUT->user_picture($user, array('popup' => true));
                    $date = userdate($version->timecreated, get_string('strftimedate'));
                    if ($date == $lastdate) {
                        $date = '';
                        $rowclass[] = '';
                    } else {
                        $lastdate = $date;
                        $rowclass[] = 'wikicode_histnewdate';
                    }

                    $time = userdate($version->timecreated, get_string('strftimetime', 'langconfig'));
                    $versionid = wikicode_get_version($version->id);
                    if ($versionid) {
                        $url = new moodle_url('/mod/wikicode/viewversion.php', array('pageid' => $pageid, 'versionid' => $versionid->id));
                        $viewlink = html_writer::link($url->out(false), $version->version);
                    } else {
                        $viewlink = $version->version;
                    }

                    $userlink = new moodle_url('/user/view.php', array('id' => $version->userid));
                    $picturelink = $picture . html_writer::link($userlink->out(false), fullname($user));
                    $historydate = $OUTPUT->container($date, 'wikicode_histdate');
                    $radiofromelement = $this->choose_from_radio(array($version->version  => null), 'fromversion', 'M.mod_wikicode.deleteversion()', $versioncount, true);
                    $radiotoelement = $this->choose_from_radio(array($version->version  => null), 'toversion', 'M.mod_wikicode.deleteversion()', $versioncount, true);
                    $contents[] = array( $radiofromelement . $radiotoelement, $viewlink, $picturelink, $time, $historydate);
                }

                $table = new html_table();
                $table->head = array(get_string('deleteversions', 'wikicode'), get_string('version'), get_string('user'), get_string('modified'), '');
                $table->data = $contents;
                $table->attributes['class'] = 'generaltable mdl-align';
                $table->rowclasses = $rowclass;

                ///Print the form
                echo html_writer::start_tag('form', array('action'=>new moodle_url('/mod/wikicode/admin.php'), 'method' => 'post'));
                echo html_writer::tag('div', html_writer::empty_tag('input', array('type' => 'hidden', 'name' => 'pageid', 'value' => $pageid)));
                echo html_writer::empty_tag('input', array('type' => 'hidden', 'name' => 'option', 'value' => $this->view));
                echo html_writer::empty_tag('input', array('type' => 'hidden', 'name' => 'sesskey', 'value' =>  sesskey()));
                echo html_writer::table($table);
                echo html_writer::start_tag('div', array('class' => 'mdl-align'));
                echo html_writer::empty_tag('input', array('type' => 'submit', 'class' => 'wikicode_form-button', 'value' => get_string('deleteversions', 'wikicode')));
                echo html_writer::end_tag('div');
                echo html_writer::end_tag('form');
            }
        } else {
            print_string('nohistory', 'wikicode');
        }
    }

    /**
     * Given an array of values, creates a group of radio buttons to be part of a form
     * helper function for print_delete_version
     *
     * @param array  $options  An array of value-label pairs for the radio group (values as keys).
     * @param string $name     Name of the radiogroup (unique in the form).
     * @param string $onclick  Function to be executed when the radios are clicked.
     * @param string $checked  The value that is already checked.
     * @param bool   $return   If true, return the HTML as a string, otherwise print it.
     *
     * @return mixed If $return is false, returns nothing, otherwise returns a string of HTML.
     */
    private function choose_from_radio($options, $name, $onclick = '', $checked = '', $return = false) {

        static $idcounter = 0;

        if (!$name) {
            $name = 'unnamed';
        }

        $output = '<span class="radiogroup ' . $name . "\">\n";

        if (!empty($options)) {
            $currentradio = 0;
            foreach ($options as $value => $label) {
                $htmlid = 'auto-rb' . sprintf('%04d', ++$idcounter);
                $output .= ' <span class="radioelement ' . $name . ' rb' . $currentradio . "\">";
                $output .= '<input name="' . $name . '" id="' . $htmlid . '" type="radio" value="' . $value . '"';
                if ($value == $checked) {
                    $output .= ' checked="checked"';
                }
                if ($onclick) {
                    $output .= ' onclick="' . $onclick . '"';
                }
                if ($label === '') {
                    $output .= ' /> <label for="' . $htmlid . '">' . $value . '</label></span>' . "\n";
                } else {
                    $output .= ' /> <label for="' . $htmlid . '">' . $label . '</label></span>' . "\n";
                }
                $currentradio = ($currentradio + 1) % 2;
            }
        }

        $output .= '</span>' . "\n";

        if ($return) {
            return $output;
        } else {
            echo $output;
        }
    }
}

 /* @copyright 2009 David Mudrak <david.mudrak@gmail.com>
 * @license http://www.gnu.org/copyleft/gpl.html GNU GPL v3 or later
 * @since Moodle 2.0
 * @package core
 * @category output
 */
class html_table {

    /**
     * @var string Value to use for the id attribute of the table
     */
    public $id = null;

    /**
     * @var array Attributes of HTML attributes for the <table> element
     */
    public $attributes = array();

    /**
     * @var array An array of headings. The n-th array item is used as a heading of the n-th column.
     * For more control over the rendering of the headers, an array of html_table_cell objects
     * can be passed instead of an array of strings.
     *
     * Example of usage:
     * $t->head = array('Student', 'Grade');
     */
    public $head;

    /**
     * @var array An array that can be used to make a heading span multiple columns.
     * In this example, {@link html_table:$data} is supposed to have three columns. For the first two columns,
     * the same heading is used. Therefore, {@link html_table::$head} should consist of two items.
     *
     * Example of usage:
     * $t->headspan = array(2,1);
     */
    public $headspan;

    /**
     * @var array An array of column alignments.
     * The value is used as CSS 'text-align' property. Therefore, possible
     * values are 'left', 'right', 'center' and 'justify'. Specify 'right' or 'left' from the perspective
     * of a left-to-right (LTR) language. For RTL, the values are flipped automatically.
     *
     * Examples of usage:
     * $t->align = array(null, 'right');
     * or
     * $t->align[1] = 'right';
     */
    public $align;

    /**
     * @var array The value is used as CSS 'size' property.
     *
     * Examples of usage:
     * $t->size = array('50%', '50%');
     * or
     * $t->size[1] = '120px';
     */
    public $size;

    /**
     * @var array An array of wrapping information.
     * The only possible value is 'nowrap' that sets the
     * CSS property 'white-space' to the value 'nowrap' in the given column.
     *
     * Example of usage:
     * $t->wrap = array(null, 'nowrap');
     */
    public $wrap;

    /**
     * @var array Array of arrays or html_table_row objects containing the data. Alternatively, if you have
     * $head specified, the string 'hr' (for horizontal ruler) can be used
     * instead of an array of cells data resulting in a divider rendered.
     *
     * Example of usage with array of arrays:
     * $row1 = array('Harry Potter', '76 %');
     * $row2 = array('Hermione Granger', '100 %');
     * $t->data = array($row1, $row2);
     *
     * Example with array of html_table_row objects: (used for more fine-grained control)
     * $cell1 = new html_table_cell();
     * $cell1->text = 'Harry Potter';
     * $cell1->colspan = 2;
     * $row1 = new html_table_row();
     * $row1->cells[] = $cell1;
     * $cell2 = new html_table_cell();
     * $cell2->text = 'Hermione Granger';
     * $cell3 = new html_table_cell();
     * $cell3->text = '100 %';
     * $row2 = new html_table_row();
     * $row2->cells = array($cell2, $cell3);
     * $t->data = array($row1, $row2);
     */
    public $data;

    /**
     * @deprecated since Moodle 2.0. Styling should be in the CSS.
     * @var string Width of the table, percentage of the page preferred.
     */
    public $width = null;

    /**
     * @deprecated since Moodle 2.0. Styling should be in the CSS.
     * @var string Alignment for the whole table. Can be 'right', 'left' or 'center' (default).
     */
    public $tablealign = null;

    /**
     * @deprecated since Moodle 2.0. Styling should be in the CSS.
     * @var int Padding on each cell, in pixels
     */
    public $cellpadding = null;

    /**
     * @var int Spacing between cells, in pixels
     * @deprecated since Moodle 2.0. Styling should be in the CSS.
     */
    public $cellspacing = null;

    /**
     * @var array Array of classes to add to particular rows, space-separated string.
     * Classes 'r0' or 'r1' are added automatically for every odd or even row,
     * respectively. Class 'lastrow' is added automatically for the last row
     * in the table.
     *
     * Example of usage:
     * $t->rowclasses[9] = 'tenth'
     */
    public $rowclasses;

    /**
     * @var array An array of classes to add to every cell in a particular column,
     * space-separated string. Class 'cell' is added automatically by the renderer.
     * Classes 'c0' or 'c1' are added automatically for every odd or even column,
     * respectively. Class 'lastcol' is added automatically for all last cells
     * in a row.
     *
     * Example of usage:
     * $t->colclasses = array(null, 'grade');
     */
    public $colclasses;

    /**
     * @var string Description of the contents for screen readers.
     */
    public $summary;

    /**
     * Constructor
     */
    public function __construct() {
        $this->attributes['class'] = '';
    }
}

/**
 * Simple html output class
 *
 * @copyright 2009 Tim Hunt, 2010 Petr Skoda
 * @license http://www.gnu.org/copyleft/gpl.html GNU GPL v3 or later
 * @since Moodle 2.0
 * @package core
 * @category output
 */
class html_writer {

    /**
     * Outputs a tag with attributes and contents
     *
     * @param string $tagname The name of tag ('a', 'img', 'span' etc.)
     * @param string $contents What goes between the opening and closing tags
     * @param array $attributes The tag attributes (array('src' => $url, 'class' => 'class1') etc.)
     * @return string HTML fragment
     */
    public static function tag($tagname, $contents, array $attributes = null) {
        return self::start_tag($tagname, $attributes) . $contents . self::end_tag($tagname);
    }

    /**
     * Outputs an opening tag with attributes
     *
     * @param string $tagname The name of tag ('a', 'img', 'span' etc.)
     * @param array $attributes The tag attributes (array('src' => $url, 'class' => 'class1') etc.)
     * @return string HTML fragment
     */
    public static function start_tag($tagname, array $attributes = null) {
        return '<' . $tagname . self::attributes($attributes) . '>';
    }

    /**
     * Outputs a closing tag
     *
     * @param string $tagname The name of tag ('a', 'img', 'span' etc.)
     * @return string HTML fragment
     */
    public static function end_tag($tagname) {
        return '</' . $tagname . '>';
    }

    /**
     * Outputs an empty tag with attributes
     *
     * @param string $tagname The name of tag ('input', 'img', 'br' etc.)
     * @param array $attributes The tag attributes (array('src' => $url, 'class' => 'class1') etc.)
     * @return string HTML fragment
     */
    public static function empty_tag($tagname, array $attributes = null) {
        return '<' . $tagname . self::attributes($attributes) . ' />';
    }

    /**
     * Outputs a tag, but only if the contents are not empty
     *
     * @param string $tagname The name of tag ('a', 'img', 'span' etc.)
     * @param string $contents What goes between the opening and closing tags
     * @param array $attributes The tag attributes (array('src' => $url, 'class' => 'class1') etc.)
     * @return string HTML fragment
     */
    public static function nonempty_tag($tagname, $contents, array $attributes = null) {
        if ($contents === '' || is_null($contents)) {
            return '';
        }
        return self::tag($tagname, $contents, $attributes);
    }

    /**
     * Outputs a HTML attribute and value
     *
     * @param string $name The name of the attribute ('src', 'href', 'class' etc.)
     * @param string $value The value of the attribute. The value will be escaped with {@link s()}
     * @return string HTML fragment
     */
    public static function attribute($name, $value) {
        if (is_array($value)) {
            debugging("Passed an array for the HTML attribute $name", DEBUG_DEVELOPER);
        }
        if ($value instanceof moodle_url) {
            return ' ' . $name . '="' . $value->out() . '"';
        }

        // special case, we do not want these in output
        if ($value === null) {
            return '';
        }

        // no sloppy trimming here!
        return ' ' . $name . '="' . s($value) . '"';
    }

    /**
     * Outputs a list of HTML attributes and values
     *
     * @param array $attributes The tag attributes (array('src' => $url, 'class' => 'class1') etc.)
     *       The values will be escaped with {@link s()}
     * @return string HTML fragment
     */
    public static function attributes(array $attributes = null) {
        $attributes = (array)$attributes;
        $output = '';
        foreach ($attributes as $name => $value) {
            $output .= self::attribute($name, $value);
        }
        return $output;
    }

    /**
     * Generates random html element id.
     *
     * @staticvar int $counter
     * @staticvar type $uniq
     * @param string $base A string fragment that will be included in the random ID.
     * @return string A unique ID
     */
    public static function random_id($base='random') {
        static $counter = 0;
        static $uniq;

        if (!isset($uniq)) {
            $uniq = uniqid();
        }

        $counter++;
        return $base.$uniq.$counter;
    }

    /**
     * Generates a simple html link
     *
     * @param string|moodle_url $url The URL
     * @param string $text The text
     * @param array $attributes HTML attributes
     * @return string HTML fragment
     */
    public static function link($url, $text, array $attributes = null) {
        $attributes = (array)$attributes;
        $attributes['href']  = $url;
        return self::tag('a', $text, $attributes);
    }

    /**
     * Generates a simple checkbox with optional label
     *
     * @param string $name The name of the checkbox
     * @param string $value The value of the checkbox
     * @param bool $checked Whether the checkbox is checked
     * @param string $label The label for the checkbox
     * @param array $attributes Any attributes to apply to the checkbox
     * @return string html fragment
     */
    public static function checkbox($name, $value, $checked = true, $label = '', array $attributes = null) {
        $attributes = (array)$attributes;
        $output = '';

        if ($label !== '' and !is_null($label)) {
            if (empty($attributes['id'])) {
                $attributes['id'] = self::random_id('checkbox_');
            }
        }
        $attributes['type']    = 'checkbox';
        $attributes['value']   = $value;
        $attributes['name']    = $name;
        $attributes['checked'] = $checked ? 'checked' : null;

        $output .= self::empty_tag('input', $attributes);

        if ($label !== '' and !is_null($label)) {
            $output .= self::tag('label', $label, array('for'=>$attributes['id']));
        }

        return $output;
    }

    /**
     * Generates a simple select yes/no form field
     *
     * @param string $name name of select element
     * @param bool $selected
     * @param array $attributes - html select element attributes
     * @return string HTML fragment
     */
    public static function select_yes_no($name, $selected=true, array $attributes = null) {
        $options = array('1'=>get_string('yes'), '0'=>get_string('no'));
        return self::select($options, $name, $selected, null, $attributes);
    }

    /**
     * Generates a simple select form field
     *
     * @param array $options associative array value=>label ex.:
     *                array(1=>'One, 2=>Two)
     *              it is also possible to specify optgroup as complex label array ex.:
     *                array(array('Odd'=>array(1=>'One', 3=>'Three)), array('Even'=>array(2=>'Two')))
     *                array(1=>'One', '--1uniquekey'=>array('More'=>array(2=>'Two', 3=>'Three')))
     * @param string $name name of select element
     * @param string|array $selected value or array of values depending on multiple attribute
     * @param array|bool $nothing add nothing selected option, or false of not added
     * @param array $attributes html select element attributes
     * @return string HTML fragment
     */
    public static function select(array $options, $name, $selected = '', $nothing = array('' => 'choosedots'), array $attributes = null) {
        $attributes = (array)$attributes;
        if (is_array($nothing)) {
            foreach ($nothing as $k=>$v) {
                if ($v === 'choose' or $v === 'choosedots') {
                    $nothing[$k] = get_string('choosedots');
                }
            }
            $options = $nothing + $options; // keep keys, do not override

        } else if (is_string($nothing) and $nothing !== '') {
            // BC
            $options = array(''=>$nothing) + $options;
        }

        // we may accept more values if multiple attribute specified
        $selected = (array)$selected;
        foreach ($selected as $k=>$v) {
            $selected[$k] = (string)$v;
        }

        if (!isset($attributes['id'])) {
            $id = 'menu'.$name;
            // name may contaion [], which would make an invalid id. e.g. numeric question type editing form, assignment quickgrading
            $id = str_replace('[', '', $id);
            $id = str_replace(']', '', $id);
            $attributes['id'] = $id;
        }

        if (!isset($attributes['class'])) {
            $class = 'menu'.$name;
            // name may contaion [], which would make an invalid class. e.g. numeric question type editing form, assignment quickgrading
            $class = str_replace('[', '', $class);
            $class = str_replace(']', '', $class);
            $attributes['class'] = $class;
        }
        $attributes['class'] = 'select ' . $attributes['class']; // Add 'select' selector always

        $attributes['name'] = $name;

        if (!empty($attributes['disabled'])) {
            $attributes['disabled'] = 'disabled';
        } else {
            unset($attributes['disabled']);
        }

        $output = '';
        foreach ($options as $value=>$label) {
            if (is_array($label)) {
                // ignore key, it just has to be unique
                $output .= self::select_optgroup(key($label), current($label), $selected);
            } else {
                $output .= self::select_option($label, $value, $selected);
            }
        }
        return self::tag('select', $output, $attributes);
    }

    /**
     * Returns HTML to display a select box option.
     *
     * @param string $label The label to display as the option.
     * @param string|int $value The value the option represents
     * @param array $selected An array of selected options
     * @return string HTML fragment
     */
    private static function select_option($label, $value, array $selected) {
        $attributes = array();
        $value = (string)$value;
        if (in_array($value, $selected, true)) {
            $attributes['selected'] = 'selected';
        }
        $attributes['value'] = $value;
        return self::tag('option', $label, $attributes);
    }

    /**
     * Returns HTML to display a select box option group.
     *
     * @param string $groupname The label to use for the group
     * @param array $options The options in the group
     * @param array $selected An array of selected values.
     * @return string HTML fragment.
     */
    private static function select_optgroup($groupname, $options, array $selected) {
        if (empty($options)) {
            return '';
        }
        $attributes = array('label'=>$groupname);
        $output = '';
        foreach ($options as $value=>$label) {
            $output .= self::select_option($label, $value, $selected);
        }
        return self::tag('optgroup', $output, $attributes);
    }

    /**
     * This is a shortcut for making an hour selector menu.
     *
     * @param string $type The type of selector (years, months, days, hours, minutes)
     * @param string $name fieldname
     * @param int $currenttime A default timestamp in GMT
     * @param int $step minute spacing
     * @param array $attributes - html select element attributes
     * @return HTML fragment
     */
    public static function select_time($type, $name, $currenttime = 0, $step = 5, array $attributes = null) {
        if (!$currenttime) {
            $currenttime = time();
        }
        $currentdate = usergetdate($currenttime);
        $userdatetype = $type;
        $timeunits = array();

        switch ($type) {
            case 'years':
                for ($i=1970; $i<=2020; $i++) {
                    $timeunits[$i] = $i;
                }
                $userdatetype = 'year';
                break;
            case 'months':
                for ($i=1; $i<=12; $i++) {
                    $timeunits[$i] = userdate(gmmktime(12,0,0,$i,15,2000), "%B");
                }
                $userdatetype = 'month';
                $currentdate['month'] = (int)$currentdate['mon'];
                break;
            case 'days':
                for ($i=1; $i<=31; $i++) {
                    $timeunits[$i] = $i;
                }
                $userdatetype = 'mday';
                break;
            case 'hours':
                for ($i=0; $i<=23; $i++) {
                    $timeunits[$i] = sprintf("%02d",$i);
                }
                break;
            case 'minutes':
                if ($step != 1) {
                    $currentdate['minutes'] = ceil($currentdate['minutes']/$step)*$step;
                }

                for ($i=0; $i<=59; $i+=$step) {
                    $timeunits[$i] = sprintf("%02d",$i);
                }
                break;
            default:
                throw new coding_exception("Time type $type is not supported by html_writer::select_time().");
        }

        if (empty($attributes['id'])) {
            $attributes['id'] = self::random_id('ts_');
        }
        $timerselector = self::select($timeunits, $name, $currentdate[$userdatetype], null, array('id'=>$attributes['id']));
        $label = self::tag('label', get_string(substr($type, 0, -1), 'form'), array('for'=>$attributes['id'], 'class'=>'accesshide'));

        return $label.$timerselector;
    }

    /**
     * Shortcut for quick making of lists
     *
     * Note: 'list' is a reserved keyword ;-)
     *
     * @param array $items
     * @param array $attributes
     * @param string $tag ul or ol
     * @return string
     */
    public static function alist(array $items, array $attributes = null, $tag = 'ul') {
        $output = '';

        foreach ($items as $item) {
            $output .= html_writer::start_tag('li') . "\n";
            $output .= $item . "\n";
            $output .= html_writer::end_tag('li') . "\n";
        }

        return html_writer::tag($tag, $output, $attributes);
    }

    /**
     * Returns hidden input fields created from url parameters.
     *
     * @param moodle_url $url
     * @param array $exclude list of excluded parameters
     * @return string HTML fragment
     */
    public static function input_hidden_params(moodle_url $url, array $exclude = null) {
        $exclude = (array)$exclude;
        $params = $url->params();
        foreach ($exclude as $key) {
            unset($params[$key]);
        }

        $output = '';
        foreach ($params as $key => $value) {
            $attributes = array('type'=>'hidden', 'name'=>$key, 'value'=>$value);
            $output .= self::empty_tag('input', $attributes)."\n";
        }
        return $output;
    }

    /**
     * Generate a script tag containing the the specified code.
     *
     * @param string $jscode the JavaScript code
     * @param moodle_url|string $url optional url of the external script, $code ignored if specified
     * @return string HTML, the code wrapped in <script> tags.
     */
    public static function script($jscode, $url=null) {
        if ($jscode) {
            $attributes = array('type'=>'text/javascript');
            return self::tag('script', "\n//<![CDATA[\n$jscode\n//]]>\n", $attributes) . "\n";

        } else if ($url) {
            $attributes = array('type'=>'text/javascript', 'src'=>$url);
            return self::tag('script', '', $attributes) . "\n";

        } else {
            return '';
        }
    }

    /**
     * Renders HTML table
     *
     * This method may modify the passed instance by adding some default properties if they are not set yet.
     * If this is not what you want, you should make a full clone of your data before passing them to this
     * method. In most cases this is not an issue at all so we do not clone by default for performance
     * and memory consumption reasons.
     *
     * @param html_table $table data to be rendered
     * @return string HTML code
     */
    public static function table(html_table $table) {
        // prepare table data and populate missing properties with reasonable defaults
        
        if (!empty($table->align)) {
            foreach ($table->align as $key => $aa) {
                if ($aa) {
                    $table->align[$key] = 'text-align:'. fix_align_rtl($aa) .';';  // Fix for RTL languages
                } else {
                    $table->align[$key] = null;
                }
            }
        }
        if (!empty($table->size)) {
            foreach ($table->size as $key => $ss) {
                if ($ss) {
                    $table->size[$key] = 'width:'. $ss .';';
                } else {
                    $table->size[$key] = null;
                }
            }
        }
        if (!empty($table->wrap)) {
            foreach ($table->wrap as $key => $ww) {
                if ($ww) {
                    $table->wrap[$key] = 'white-space:nowrap;';
                } else {
                    $table->wrap[$key] = '';
                }
            }
        }
        if (!empty($table->head)) {
            foreach ($table->head as $key => $val) {
                if (!isset($table->align[$key])) {
                    $table->align[$key] = null;
                }
                if (!isset($table->size[$key])) {
                    $table->size[$key] = null;
                }
                if (!isset($table->wrap[$key])) {
                    $table->wrap[$key] = null;
                }

            }
        }
        if (empty($table->attributes['class'])) {
            $table->attributes['class'] = 'generaltable';
        }
        if (!empty($table->tablealign)) {
            $table->attributes['class'] .= ' boxalign' . $table->tablealign;
        }

        // explicitly assigned properties override those defined via $table->attributes
        $table->attributes['class'] = trim($table->attributes['class']);
        $attributes = array_merge($table->attributes, array(
                'id'            => $table->id,
                'width'         => $table->width,
                'summary'       => $table->summary,
                'cellpadding'   => $table->cellpadding,
                'cellspacing'   => $table->cellspacing,
            ));
        $output = html_writer::start_tag('table', $attributes) . "\n";

        $countcols = 0;

        if (!empty($table->head)) {
            $countcols = count($table->head);

            $output .= html_writer::start_tag('thead', array()) . "\n";
            $output .= html_writer::start_tag('tr', array()) . "\n";
            $keys = array_keys($table->head);
            $lastkey = end($keys);

            foreach ($table->head as $key => $heading) {
                // Convert plain string headings into html_table_cell objects
                if (!($heading instanceof html_table_cell)) {
                    $headingtext = $heading;
                    $heading = new html_table_cell();
                    $heading->text = $headingtext;
                    $heading->header = true;
                }

                if ($heading->header !== false) {
                    $heading->header = true;
                }

                if ($heading->header && empty($heading->scope)) {
                    $heading->scope = 'col';
                }

                $heading->attributes['class'] .= ' header c' . $key;
                if (isset($table->headspan[$key]) && $table->headspan[$key] > 1) {
                    $heading->colspan = $table->headspan[$key];
                    $countcols += $table->headspan[$key] - 1;
                }

                if ($key == $lastkey) {
                    $heading->attributes['class'] .= ' lastcol';
                }
                if (isset($table->colclasses[$key])) {
                    $heading->attributes['class'] .= ' ' . $table->colclasses[$key];
                }
                $heading->attributes['class'] = trim($heading->attributes['class']);
                $attributes = array_merge($heading->attributes, array(
                        'style'     => $table->align[$key] . $table->size[$key] . $heading->style,
                        'scope'     => $heading->scope,
                        'colspan'   => $heading->colspan,
                    ));

                $tagtype = 'td';
                if ($heading->header === true) {
                    $tagtype = 'th';
                }
                $output .= html_writer::tag($tagtype, $heading->text, $attributes) . "\n";
            }
            $output .= html_writer::end_tag('tr') . "\n";
            $output .= html_writer::end_tag('thead') . "\n";

            if (empty($table->data)) {
                // For valid XHTML strict every table must contain either a valid tr
                // or a valid tbody... both of which must contain a valid td
                $output .= html_writer::start_tag('tbody', array('class' => 'empty'));
                $output .= html_writer::tag('tr', html_writer::tag('td', '', array('colspan'=>count($table->head))));
                $output .= html_writer::end_tag('tbody');
            }
        }

        if (!empty($table->data)) {
            $oddeven    = 1;
            $keys       = array_keys($table->data);
            $lastrowkey = end($keys);
            $output .= html_writer::start_tag('tbody', array());

            foreach ($table->data as $key => $row) {
                if (($row === 'hr') && ($countcols)) {
                    $output .= html_writer::tag('td', html_writer::tag('div', '', array('class' => 'tabledivider')), array('colspan' => $countcols));
                } else {
                    // Convert array rows to html_table_rows and cell strings to html_table_cell objects
                    if (!($row instanceof html_table_row)) {
                        $newrow = new html_table_row();

                        foreach ($row as $cell) {
                            if (!($cell instanceof html_table_cell)) {
                                $cell = new html_table_cell($cell);
                            }
                            $newrow->cells[] = $cell;
                        }
                        $row = $newrow;
                    }

                    $oddeven = $oddeven ? 0 : 1;
                    if (isset($table->rowclasses[$key])) {
                        $row->attributes['class'] .= ' ' . $table->rowclasses[$key];
                    }

                    $row->attributes['class'] .= ' r' . $oddeven;
                    if ($key == $lastrowkey) {
                        $row->attributes['class'] .= ' lastrow';
                    }

                    $output .= html_writer::start_tag('tr', array('class' => trim($row->attributes['class']), 'style' => $row->style, 'id' => $row->id)) . "\n";
                    $keys2 = array_keys($row->cells);
                    $lastkey = end($keys2);

                    $gotlastkey = false; //flag for sanity checking
                    foreach ($row->cells as $key => $cell) {
                        if ($gotlastkey) {
                            //This should never happen. Why do we have a cell after the last cell?
                            mtrace("A cell with key ($key) was found after the last key ($lastkey)");
                        }

                        if (!($cell instanceof html_table_cell)) {
                            $mycell = new html_table_cell();
                            $mycell->text = $cell;
                            $cell = $mycell;
                        }

                        if (($cell->header === true) && empty($cell->scope)) {
                            $cell->scope = 'row';
                        }

                        if (isset($table->colclasses[$key])) {
                            $cell->attributes['class'] .= ' ' . $table->colclasses[$key];
                        }

                        $cell->attributes['class'] .= ' cell c' . $key;
                        if ($key == $lastkey) {
                            $cell->attributes['class'] .= ' lastcol';
                            $gotlastkey = true;
                        }
                        $tdstyle = '';
                        $tdstyle .= isset($table->align[$key]) ? $table->align[$key] : '';
                        $tdstyle .= isset($table->size[$key]) ? $table->size[$key] : '';
                        $tdstyle .= isset($table->wrap[$key]) ? $table->wrap[$key] : '';
                        $cell->attributes['class'] = trim($cell->attributes['class']);
                        $tdattributes = array_merge($cell->attributes, array(
                                'style' => $tdstyle . $cell->style,
                                'colspan' => $cell->colspan,
                                'rowspan' => $cell->rowspan,
                                'id' => $cell->id,
                                'abbr' => $cell->abbr,
                                'scope' => $cell->scope,
                            ));
                        $tagtype = 'td';
                        if ($cell->header === true) {
                            $tagtype = 'th';
                        }
                        $output .= html_writer::tag($tagtype, $cell->text, $tdattributes) . "\n";
                    }
                }
                $output .= html_writer::end_tag('tr') . "\n";
            }
            $output .= html_writer::end_tag('tbody') . "\n";
        }
        $output .= html_writer::end_tag('table') . "\n";

        return $output;
    }

    /**
     * Renders form element label
     *
     * By default, the label is suffixed with a label separator defined in the
     * current language pack (colon by default in the English lang pack).
     * Adding the colon can be explicitly disabled if needed. Label separators
     * are put outside the label tag itself so they are not read by
     * screenreaders (accessibility).
     *
     * Parameter $for explicitly associates the label with a form control. When
     * set, the value of this attribute must be the same as the value of
     * the id attribute of the form control in the same document. When null,
     * the label being defined is associated with the control inside the label
     * element.
     *
     * @param string $text content of the label tag
     * @param string|null $for id of the element this label is associated with, null for no association
     * @param bool $colonize add label separator (colon) to the label text, if it is not there yet
     * @param array $attributes to be inserted in the tab, for example array('accesskey' => 'a')
     * @return string HTML of the label element
     */
    public static function label($text, $for, $colonize = true, array $attributes=array()) {
        if (!is_null($for)) {
            $attributes = array_merge($attributes, array('for' => $for));
        }
        $text = trim($text);
        $label = self::tag('label', $text, $attributes);

        // TODO MDL-12192 $colonize disabled for now yet
        // if (!empty($text) and $colonize) {
        //     // the $text may end with the colon already, though it is bad string definition style
        //     $colon = get_string('labelsep', 'langconfig');
        //     if (!empty($colon)) {
        //         $trimmed = trim($colon);
        //         if ((substr($text, -strlen($trimmed)) == $trimmed) or (substr($text, -1) == ':')) {
        //             //debugging('The label text should not end with colon or other label separator,
        //             //           please fix the string definition.', DEBUG_DEVELOPER);
        //         } else {
        //             $label .= $colon;
        //         }
        //     }
        // }

        return $label;
    }
}

/**
 * Component representing a table cell.
 *
 * @copyright 2009 Nicolas Connault
 * @license http://www.gnu.org/copyleft/gpl.html GNU GPL v3 or later
 * @since Moodle 2.0
 * @package core
 * @category output
 */
class html_table_cell {

    /**
     * @var string Value to use for the id attribute of the cell.
     */
    public $id = null;

    /**
     * @var string The contents of the cell.
     */
    public $text;

    /**
     * @var string Abbreviated version of the contents of the cell.
     */
    public $abbr = null;

    /**
     * @var int Number of columns this cell should span.
     */
    public $colspan = null;

    /**
     * @var int Number of rows this cell should span.
     */
    public $rowspan = null;

    /**
     * @var string Defines a way to associate header cells and data cells in a table.
     */
    public $scope = null;

    /**
     * @var bool Whether or not this cell is a header cell.
     */
    public $header = null;

    /**
     * @var string Value to use for the style attribute of the table cell
     */
    public $style = null;

    /**
     * @var array Attributes of additional HTML attributes for the <td> element
     */
    public $attributes = array();

    /**
     * Constructs a table cell
     *
     * @param string $text
     */
    public function __construct($text = null) {
        $this->text = $text;
        $this->attributes['class'] = '';
    }
}


/**
 * Component representing a table row.
 *
 * @copyright 2009 Nicolas Connault
 * @license http://www.gnu.org/copyleft/gpl.html GNU GPL v3 or later
 * @since Moodle 2.0
 * @package core
 * @category output
 */
class html_table_row {

    /**
     * @var string Value to use for the id attribute of the row.
     */
    public $id = null;

    /**
     * @var array Array of html_table_cell objects
     */
    public $cells = array();

    /**
     * @var string Value to use for the style attribute of the table row
     */
    public $style = null;

    /**
     * @var array Attributes of additional HTML attributes for the <tr> element
     */
    public $attributes = array();

    /**
     * Constructor
     * @param array $cells
     */
    public function __construct(array $cells=null) {
        $this->attributes['class'] = '';
        $cells = (array)$cells;
        foreach ($cells as $cell) {
            if ($cell instanceof html_table_cell) {
                $this->cells[] = $cell;
            } else {
                $this->cells[] = new html_table_cell($cell);
            }
        }
    }
}
\end{lstlisting}

\subsection{version.php}
\begin{lstlisting}[language=PHP]
<?php

defined('MOODLE_INTERNAL') || die();

$module->version   = 2012022100;       // The current module version (Date: YYYYMMDDXX)
$module->requires  = 2006101592;       // Requires this Moodle version
$module->component = 'mod_wikicode';       // Full name of the plugin (used for diagnostics)
$module->cron      = 0;
\end{lstlisting}

\subsection{view.php}
\begin{lstlisting}[language=PHP]
<?php

require_once('../../config.php');
require_once($CFG->dirroot . '/mod/wikicode/lib.php');
require_once($CFG->dirroot . '/mod/wikicode/locallib.php');
require_once($CFG->dirroot . '/mod/wikicode/pagelib.php');

$id = optional_param('id', 0, PARAM_INT); // Course Module ID

$pageid = optional_param('pageid', 0, PARAM_INT); // Page ID

$wid = optional_param('wid', 0, PARAM_INT); // Wiki ID
$title = optional_param('title', '', PARAM_TEXT); // Page Title
$currentgroup = optional_param('group', 0, PARAM_INT); // Group ID
$userid = optional_param('uid', 0, PARAM_INT); // User ID
$groupanduser = optional_param('groupanduser', 0, PARAM_TEXT);

$edit = optional_param('edit', -1, PARAM_BOOL);

$action = optional_param('action', '', PARAM_ALPHA);
$swid = optional_param('swid', 0, PARAM_INT); // Subwiki ID

/*
 * Case 0:
 *
 * User that comes from a course. First wiki page must be shown
 *
 * URL params: id -> course module id
 *
 */
if ($id) {
    // Cheacking course module instance
    if (!$cm = get_coursemodule_from_id('wikicode', $id)) {
        print_error('invalidcoursemodule');
    }

    // Checking course instance
    $course = get_record('course', 'id', $cm->course);

    // Checking wiki instance
    if (!$wiki = wikicode_get_wiki($cm->instance)) {
        print_error('incorrectwikiid', 'wikicode');
    }

    //$PAGE->set_cm($cm);

    // Getting the subwiki corresponding to that wiki, group and user.
    //
    // Also setting the page if it exists or getting the first page title form
    // that wiki

    // Getting current group id
    $currentgroup = groups_get_activity_group($cm);
    $currentgroup = !empty($currentgroup) ? $currentgroup : 0;
    // Getting current user id
    if ($wiki->wikimode == 'individual') {
        $userid = $USER->id;
    } else {
        $userid = 0;
    }

    // Getting subwiki. If it does not exists, redirecting to create page
    if (!$subwiki = wikicode_get_subwiki_by_group($wiki->id, $currentgroup, $userid)) {
        $params = array('wid' => $wiki->id, 'gid' => $currentgroup, 'uid' => $userid, 'title' => $wiki->firstpagetitle);	
        $url = new moodle_url('./create.php', $params);
        redirect($url->out());
    }

    // Getting first page. If it does not exists, redirecting to create page
    if (!$page = wikicode_get_first_page($subwiki->id, $wiki)) {
        $params = array('swid'=>$subwiki->id, 'title'=>$wiki->firstpagetitle);
        $url = new moodle_url('./create.php', $params);
        redirect($url->out());
    }

    /*
     * Case 1:
     *
     * A user wants to see a page.
     *
     * URL Params: pageid -> page id
     *
     */
} elseif ($pageid) {

    // Checking page instance
    if (!$page = wikicode_get_page($pageid)) {
        print_error('incorrectpageid', 'wikicode');
    }

    // Checking subwiki
    if (!$subwiki = wikicode_get_subwiki($page->subwikiid)) {
        print_error('incorrectsubwikiid', 'wikicode');
    }

    // Checking wiki instance of that subwiki
    if (!$wiki = wikicode_get_wiki($subwiki->wikiid)) {
        print_error('incorrectwikiid', 'wikicode');
    }

    // Checking course module instance
    if (!$cm = get_coursemodule_from_instance("wikicode", $subwiki->wikiid)) {
        print_error('invalidcoursemodule');
    }

    // Checking course instance
    $course = get_record('course', 'id', $cm->course);

    /*
     * Case 2:
     *
     * Trying to read a page from another group or user
     *
     * Page can exists or not.
     *  * If it exists, page must be shown
     *  * If it does not exists, system must ask for its creation
     *
     * URL params: wid -> subwiki id (required)
     *             title -> a page title (required)
     *             group -> group id (optional)
     *             uid -> user id (optional)
     *             groupanduser -> (optional)
     */
} elseif ($wid && $title) {

    // Setting wiki instance
    if (!$wiki = wikicode_get_wiki($wid)) {
        print_error('incorrectwikiid', 'wikicode');
    }

    // Checking course module
    if (!$cm = get_coursemodule_from_instance("wikicode", $wiki->id)) {
        print_error('invalidcoursemodule');
    }

    // Checking course instance
    if (!$course = get_record("course", "id", $cm->course)) {
        print_error('coursemisconf');
    }

    $groupmode = groups_get_activity_groupmode($cm);
    if (empty($currentgroup)) {
        $currentgroup = groups_get_activity_group($cm);
        $currentgroup = !empty($currentgroup) ? $currentgroup : 0;
    }

    if ($wiki->wikimode == 'individual' && ($groupmode == SEPARATEGROUPS || $groupmode == VISIBLEGROUPS)) {
        list($gid, $uid) = explode('-', $groupanduser);
    } else if ($wiki->wikimode == 'individual') {
        $gid = 0;
        $uid = $userid;
    } else if ($groupmode == NOGROUPS) {
        $gid = 0;
        $uid = 0;
    } else {
        $gid = $currentgroup;
        $uid = 0;
    }

    // Getting subwiki instance. If it does not exists, redirect to create page
    if (!$subwiki = wikicode_get_subwiki_by_group($wiki->id, $gid, $uid)) {
        $context = get_context_instance(CONTEXT_MODULE, $cm->id);

        $modeanduser = $wiki->wikimode == 'individual' && $uid != $USER->id;
        $modeandgroupmember = $wiki->wikimode == 'collaborative' && !groups_is_member($gid);

        $manage = has_capability('mod/wikicode:managewiki', $context);
        $edit = has_capability('mod/wikicode:editpage', $context);
        $manageandedit = $manage && $edit;

        if ($groupmode == VISIBLEGROUPS and ($modeanduser || $modeandgroupmember) and !$manageandedit) {
            print_error('nocontent','wikicode');
        }

        $params = array('wid' => $wiki->id, 'gid' => $gid, 'uid' => $uid, 'title' => $title);
        $url = new moodle_url('/mod/wikicode/create.php', $params);
        redirect($url);
    }

    // Checking is there is a page with this title. If it does not exists, redirect to first page
    if (!$page = wikicode_get_page_by_title($subwiki->id, $title)) {
        $params = array('wid' => $wiki->id, 'gid' => $gid, 'uid' => $uid, 'title' => $wiki->firstpagetitle);
        $url = new moodle_url('/mod/wikicode/view.php', $params);
        redirect($url);
    }

    //    /*
    //     * Case 3:
    //     *
    //     * A user switches group when is 'reading' a non-existent page.
    //     *
    //     * URL Params: wid -> wiki id
    //     *             title -> page title
    //     *             currentgroup -> group id
    //     *
    //     */
    //} elseif ($wid && $title && $currentgroup) {
    //
    //    // Checking wiki instance
    //    if (!$wiki = wikicode_get_wiki($wid)) {
    //        print_error('incorrectwikiid', 'wiki');
    //    }
    //
    //    // Checking subwiki instance
    //    // @TODO: Fix call to wikicode_get_subwiki_by_group
    //    if (!$currentgroup = groups_get_activity_group($cm)){
    //        $currentgroup = 0;
    //    }
    //    if (!$subwiki = wikicode_get_subwiki_by_group($wid, $currentgroup)) {
    //        print_error('incorrectsubwikiid', 'wiki');
    //    }
    //
    //    // Checking page instance
    //    if ($page = wikicode_get_page_by_title($subwiki->id, $title)) {
    //        unset($title);
    //    }
    //
    //    // Checking course instance
    //    $course = $DB->get_record('course', array('id'=>$cm->course), '*', MUST_EXIST);
    //
    //    // Checking course module instance
    //    if (!$cm = get_coursemodule_from_instance("wiki", $wiki->id, $course->id)) {
    //        print_error('invalidcoursemodule');
    //    }
    //
    //    $subwiki = null;
    //    $page = null;
    //
    //    /*
    //     * Case 4:
    //     *
    //     * Error. No more options
    //     */
} else {
    print_error('incorrectparameters');
}
require_login($course, true, $cm);


$context = get_context_instance(CONTEXT_MODULE, $cm->id);
require_capability('mod/wikicode:viewpage', $context);

add_to_log($course->id, 'wikicode', 'view', 'view.php?id=' . $cm->id, $wiki->id);

// Update 'viewed' state if required by completion system
/*require_once($CFG->libdir . '/completionlib.php');
$completion = new completion_info($course);
$completion->set_module_viewed($cm);

if (($edit != - 1) and $PAGE->user_allowed_editing()) {
    $USER->editing = $edit;
}*/

$wikipage = new page_wikicode_view($wiki, $subwiki, $cm);

/*The following piece of code is used in order
 * to perform set_url correctly. It is necessary in order
 * to make page_wikicode_view class know that this page
 * has been called via its id.
 */
if ($id) {
    $wikipage->set_coursemodule($id);
}

$wikipage->set_gid($currentgroup);
$wikipage->set_page($page);

$wikipage->print_header($cm, $course);

$wikipage->print_content();

print_footer($course);
\end{lstlisting}

\subsection{viewversion.php}
\begin{lstlisting}[language=PHP]
<?php

require_once('../../config.php');

require_once($CFG->dirroot . '/mod/wikicode/lib.php');
require_once($CFG->dirroot . '/mod/wikicode/locallib.php');
require_once($CFG->dirroot . '/mod/wikicode/pagelib.php');

$pageid = required_param('pageid', PARAM_TEXT);
$versionid = required_param('versionid', PARAM_INT);

if (!$page = wikicode_get_page($pageid)) {
    print_error('incorrectpageid', 'wikicode');
}

if (!$subwiki = wikicode_get_subwiki($page->subwikiid)) {
    print_error('incorrectsubwikiid', 'wikicode');
}

if (!$wiki = wikicode_get_wiki($subwiki->wikiid)) {
    print_error('incorrectwikiid', 'wikicode');
}

if (!$cm = get_coursemodule_from_instance('wikicode', $wiki->id)) {
    print_error('invalidcoursemodule');
}

$course = get_record('course', 'id', $cm->course);

require_login($course->id, true, $cm);
add_to_log($course->id, "wikicode", "history", "history.php?id=$cm->id", "$wiki->id");

/// Print the page header
$wikipage = new page_wikicode_viewversion($wiki, $subwiki, $cm);

$wikipage->set_page($page);
$wikipage->set_versionid($versionid);

$wikipage->print_header($cm, $course);
$wikipage->print_content();

print_footer($course);

\end{lstlisting}

\section{Back-End}

\subsection{lib.php}
\begin{lstlisting}[language=PHP]
<?php

/**
 * Given an object containing all the necessary data,
 * (defined by the form in mod.html) this function
 * will create a new instance and return the id number
 * of the new instance.
 *
 * @param object $instance An object from the form in mod.html
 * @return int The id of the newly inserted wiki record
 **/
function wikicode_add_instance($wiki) {

    $wiki->timemodified = time();
    # May have to add extra stuff in here #
    if (empty($wiki->forceformat)) {
        $wiki->forceformat = 0;
    }
	return insert_record("wikicode", $wiki);
}

/**
 * Given an object containing all the necessary data,
 * (defined by the form in mod.html) this function
 * will update an existing instance with new data.
 *
 * @param object $instance An object from the form in mod.html
 * @return boolean Success/Fail
 **/
function wikicode_update_instance($wiki) {

    $wiki->timemodified = time();
    $wiki->id = $wiki->instance;
    if (empty($wiki->forceformat)) {
        $wiki->forceformat = 0;
    }

    # May have to add extra stuff in here #

    return update_record('wikicode', $wiki);
}

/**
 * Given an ID of an instance of this module,
 * this function will permanently delete the instance
 * and any data that depends on it.
 *
 * @param int $id Id of the module instance
 * @return boolean Success/Failure
 **/
function wikicode_delete_instance($id) {
    
    if (!$wiki = get_record('wikicode', 'id', $id)) {
        return false;
    }

    $result = true;

    # Get subwiki information #
    $subwikis = get_records('wikicode_subwikis', array('wikiid' => $wiki->id));

    foreach ($subwikis as $subwiki) {
        # Get existing links, and delete them #
        if (!delete_records('wikicode_links', 'subwikiid',$subwiki->id)) {
            $result = false;
        }

        # Get existing pages #
        if ($pages = get_records('wikicode_pages', 'subwikiid', $subwiki->id)) {
            foreach ($pages as $page) {
                # Get locks, and delete them #
                if (!delete_records('wikicode_locks', 'pageid', $page->id)) {
                    $result = false;
                }

                # Get versions, and delete them #
                if (!delete_records('wikicode_versions', 'pageid', $page->id)) {
                    $result = false;
                }
            }

            # Delete pages #
            if (!delete_records('wikicode_pages', 'subwikiid', $subwiki->id)) {
                $result = false;
            }
        }

        # Get existing synonyms, and delete them #
        if (!delete_records('wikicode_synonyms', 'subwikiid', $subwiki->id)) {
            $result = false;
        }

        # Delete any subwikis #
        if (!delete_records('wikicode_subwikis', 'id', $subwiki->id)) {
            $result = false;
        }
    }

    # Delete any dependent records here #
    if (!delete_records('wikicode', 'id', $wiki->id)) {
        $result = false;
    }

    return $result;
}

function wikicode_reset_userdata($data) {
    global $CFG,$DB;
    require_once($CFG->dirroot . '/mod/wikicode/pagelib.php');
    require_once($CFG->dirroot . '/tag/lib.php');

    $componentstr = get_string('modulenameplural', 'wikicode');
    $status = array();

    //get the wiki(s) in this course.
    if (!$wikis = get_records('wikicode', 'course', $data->courseid)) {
        return false;
    }
    $errors = false;
    foreach ($wikis as $wiki) {

        // remove all comments
        if (!empty($data->reset_wikicode_comments)) {
            if (!$cm = get_coursemodule_from_instance('wikicode', $wiki->id)) {
                continue;
            }
            $context = get_context_instance(CONTEXT_MODULE, $cm->id);
            delete_records_select('comments', "contextid = ".$context->id." AND commentarea='wikicode_page'");
            $status[] = array('component'=>$componentstr, 'item'=>get_string('deleteallcomments'), 'error'=>false);
        }

        if (!empty($data->reset_wikicode_tags)) {
            # Get subwiki information #
            $subwikis = get_records('wikicode_subwikis', 'wikiid', $wiki->id);

            foreach ($subwikis as $subwiki) {
                if ($pages = get_records('wikicode_pages', 'subwikiid', $subwiki->id)) {
                    foreach ($pages as $page) {
                        $tags = tag_get_tags_array('wikicode_pages', $page->id);
                        foreach ($tags as $tagid => $tagname) {
                            // Delete the related tag_instances related to the wiki page.
                            $errors = tag_delete_instance('wikicode_pages', $page->id, $tagid);
                            $status[] = array('component' => $componentstr, 'item' => get_string('tagsdeleted', 'wikicode'), 'error' => $errors);
                        }
                    }
                }
            }
        }
    }
    return $status;
}


function wikicode_reset_course_form_definition(&$mform) {
    $mform->addElement('header', 'wikiheader', get_string('modulenameplural', 'wikicode'));
    $mform->addElement('advcheckbox', 'reset_wikicode_tags', get_string('removeallwikitags', 'wikicode'));
    $mform->addElement('advcheckbox', 'reset_wikicode_comments', get_string('deleteallcomments'));
}

/**
 * Return a small object with summary information about what a
 * user has done with a given particular instance of this module
 * Used for user activity reports.
 * $return->time = the time they did it
 * $return->info = a short text description
 *
 * @return null
 * @todo Finish documenting this function
 **/
function wikicode_user_outline($course, $user, $mod, $wiki) {
    $return = NULL;
    return $return;
}

/**
 * Print a detailed representation of what a user has done with
 * a given particular instance of this module, for user activity reports.
 *
 * @return boolean
 * @todo Finish documenting this function
 **/
function wikicode_user_complete($course, $user, $mod, $wiki) {
    return true;
}

/**
 * Indicates API features that the wiki supports.
 *
 * @uses FEATURE_GROUPS
 * @uses FEATURE_GROUPINGS
 * @uses FEATURE_GROUPMEMBERSONLY
 * @uses FEATURE_MOD_INTRO
 * @uses FEATURE_COMPLETION_TRACKS_VIEWS
 * @uses FEATURE_COMPLETION_HAS_RULES
 * @uses FEATURE_GRADE_HAS_GRADE
 * @uses FEATURE_GRADE_OUTCOMES
 * @param string $feature
 * @return mixed True if yes (some features may use other values)
 */
function wikicode_supports($feature) {
    switch ($feature) {
    case FEATURE_GROUPS:
        return true;
    case FEATURE_GROUPINGS:
        return true;
    case FEATURE_GROUPMEMBERSONLY:
        return true;
    case FEATURE_MOD_INTRO:
        return true;
    case FEATURE_COMPLETION_TRACKS_VIEWS:
        return true;
    case FEATURE_GRADE_HAS_GRADE:
        return false;
    case FEATURE_GRADE_OUTCOMES:
        return false;
    case FEATURE_RATE:
        return false;
    case FEATURE_BACKUP_MOODLE2:
        return true;
    case FEATURE_SHOW_DESCRIPTION:
        return true;

    default:
        return null;
    }
}

/**
 * Given a course and a time, this module should find recent activity
 * that has occurred in wiki activities and print it out.
 * Return true if there was output, or false is there was none.
 *
 * @global $CFG
 * @global $DB
 * @uses CONTEXT_MODULE
 * @uses VISIBLEGROUPS
 * @param object $course
 * @param bool $viewfullnames capability
 * @param int $timestart
 * @return boolean
 **/
function wikicode_print_recent_activity($course, $viewfullnames, $timestart) {
    global $CFG, $OUTPUT;

    $sql = "SELECT p.*, w.id as wikiid, sw.groupid
            FROM {$CFG->prefix}wikicode_pages p
                JOIN {$CFG->prefix}wikicode_subwikis sw ON sw.id = p.subwikiid
                JOIN {$CFG->prefix}wikicode w ON w.id = sw.wikiid
            WHERE p.timemodified > ".$timestart." AND w.course = ".$course->id."
            ORDER BY p.timemodified ASC";
    if (!$pages = get_records_sql($sql)) {
        return false;
    }
    $modinfo =& get_fast_modinfo($course);

    $wikis = array();

    $modinfo = get_fast_modinfo($course);

    foreach ($pages as $page) {
        if (!isset($modinfo->instances['wikicode'][$page->wikiid])) {
            // not visible
            continue;
        }
        $cm = $modinfo->instances['wikicode'][$page->wikiid];
        if (!$cm->uservisible) {
            continue;
        }
        $context = get_context_instance(CONTEXT_MODULE, $cm->id);

        if (!has_capability('mod/wikicode:viewpage', $context)) {
            continue;
        }

        $groupmode = groups_get_activity_groupmode($cm, $course);

        if ($groupmode) {
            if ($groupmode == SEPARATEGROUPS and !has_capability('mod/wikicode:managewiki', $context)) {
                // separate mode
                if (isguestuser()) {
                    // shortcut
                    continue;
                }

                if (is_null($modinfo->groups)) {
                    $modinfo->groups = groups_get_user_groups($course->id); // load all my groups and cache it in modinfo
                    }

                if (!in_array($page->groupid, $modinfo->groups[0])) {
                    continue;
                }
            }
        }
        $wikis[] = $page;
    }
    unset($pages);

    if (!$wikis) {
        return false;
    }
    echo $OUTPUT->heading(get_string("updatedwikipages", 'wikicode') . ':', 3);
    foreach ($wikis as $wiki) {
        $cm = $modinfo->instances['wikicode'][$wiki->wikiid];
        $link = $CFG->wwwroot . '/mod/wikicode/view.php?pageid=' . $wiki->id;
        print_recent_activity_note($wiki->timemodified, $wiki, $cm->name, $link, false, $viewfullnames);
    }

    return true; //  True if anything was printed, otherwise false
}
/**
 * Function to be run periodically according to the moodle cron
 * This function searches for things that need to be done, such
 * as sending out mail, toggling flags etc ...
 *
 * @uses $CFG
 * @return boolean
 * @todo Finish documenting this function
 **/
function wikicode_cron() {
    global $CFG;

    return true;
}

/**
 * Must return an array of grades for a given instance of this module,
 * indexed by user.  It also returns a maximum allowed grade.
 *
 * Example:
 *    $return->grades = array of grades;
 *    $return->maxgrade = maximum allowed grade;
 *
 *    return $return;
 *
 * @param int $wikiid ID of an instance of this module
 * @return mixed Null or object with an array of grades and with the maximum grade
 **/
function wikicode_grades($wikiid) {
    return null;
}

/**
 * Must return an array of user records (all data) who are participants
 * for a given instance of wiki. Must include every user involved
 * in the instance, independient of his role (student, teacher, admin...)
 * See other modules as example.
 *
 * @todo: deprecated - to be deleted in 2.2
 *
 * @param int $wikiid ID of an instance of this module
 * @return mixed boolean/array of students
 **/
function wikicode_get_participants($wikiid) {
    return false;
}

/**
 * This function returns if a scale is being used by one wiki
 * it it has support for grading and scales. Commented code should be
 * modified if necessary. See forum, glossary or journal modules
 * as reference.
 *
 * @param int $wikiid ID of an instance of this module
 * @return mixed
 * @todo Finish documenting this function
 **/
function wikicode_scale_used($wikiid, $scaleid) {
    $return = false;

    //$rec = get_record("wiki","id","$wikiid","scale","-$scaleid");
    //
    //if (!empty($rec)  && !empty($scaleid)) {
    //    $return = true;
    //}

    return $return;
}

/**
 * Checks if scale is being used by any instance of wiki.
 * This function was added in 1.9
 *
 * This is used to find out if scale used anywhere
 * @param $scaleid int
 * @return boolean True if the scale is used by any wiki
 */
function wikicode_scale_used_anywhere($scaleid) {

    //if ($scaleid and $DB->record_exists('wikicode', array('grade' => -$scaleid))) {
    //    return true;
    //} else {
    //    return false;
    //}

    return false;
}

/**
 * Pluginfile hook
 *
 */
function wikicode_pluginfile($course, $cm, $context, $filearea, $args, $forcedownload) {
    global $CFG;

    if ($context->contextlevel != CONTEXT_MODULE) {
        return false;
    }

    require_login($course, true, $cm);

    require_once($CFG->dirroot . "/mod/wikicode/locallib.php");

    if ($filearea == 'attachments') {
        $swid = (int) array_shift($args);

        if (!$subwiki = wikicode_get_subwiki($swid)) {
            return false;
        }

        require_capability('mod/wikicode:viewpage', $context);

        $relativepath = implode('/', $args);

        $fullpath = "/$context->id/mod_wikicode/attachments/$swid/$relativepath";

        $fs = get_file_storage();
        if (!$file = $fs->get_file_by_hash(sha1($fullpath)) or $file->is_directory()) {
            return false;
        }

        $lifetime = isset($CFG->filelifetime) ? $CFG->filelifetime : 86400;

        send_stored_file($file, $lifetime, 0);
    }
}

function wikicode_search_form($cm, $search = '') {
    global $CFG, $OUTPUT;

    $output = '<div class="wikisearch">';
    $output .= '<form method="post" action="' . $CFG->wwwroot . '/mod/wikicode/search.php" style="display:inline">';
    $output .= '<fieldset class="invisiblefieldset">';
    $output .= '<input name="searchstring" type="text" size="18" value="' . s($search, true) . '" alt="search" />';
    $output .= '<input name="courseid" type="hidden" value="' . $cm->course . '" />';
    $output .= '<input name="cmid" type="hidden" value="' . $cm->id . '" />';
    $output .= '<input name="searchwikicontent" type="hidden" value="1" />';
    $output .= ' <input value="' . get_string('searchwikis', 'wikicode') . '" type="submit" />';
    $output .= '</fieldset>';
    $output .= '</form>';
    $output .= '</div>';

    return $output;
}
function wikicode_extend_navigation(navigation_node $navref, $course, $module, $cm) {
    global $CFG, $PAGE, $USER;

    require_once($CFG->dirroot . '/mod/wikicode/locallib.php');

    $context = get_context_instance(CONTEXT_MODULE, $cm->id);
    $url = $PAGE->url;
    $userid = 0;
    if ($module->wikimode == 'individual') {
        $userid = $USER->id;
    }

    if (!$wiki = wikicode_get_wiki($cm->instance)) {
        return false;
    }

    if (!$gid = groups_get_activity_group($cm)) {
        $gid = 0;
    }
    if (!$subwiki = wikicode_get_subwiki_by_group($cm->instance, $gid, $userid)) {
        return null;
    } else {
        $swid = $subwiki->id;
    }

    $pageid = $url->param('pageid');
    $cmid = $url->param('id');
    if (empty($pageid) && !empty($cmid)) {
        // wiki main page
        $page = wikicode_get_page_by_title($swid, $wiki->firstpagetitle);
        $pageid = $page->id;
    }

    if (has_capability('mod/wikicode:createpage', $context)) {
       //$link = new moodle_url('/mod/wikicode/create.php', array('action' => 'new', 'swid' => $swid));
       //$node = $navref->add(get_string('newpage', 'wikicode'), $link, navigation_node::TYPE_SETTING);
    }

    if (is_numeric($pageid)) {

        if (has_capability('mod/wikicode:viewpage', $context)) {
            $link = new moodle_url('/mod/wikicode/view.php', array('pageid' => $pageid));
            $node = $navref->add(get_string('view', 'wikicode'), $link, navigation_node::TYPE_SETTING);
        }

        if (has_capability('mod/wikicode:editpage', $context)) {
            $link = new moodle_url('/mod/wikicode/edit.php', array('pageid' => $pageid));
            $node = $navref->add(get_string('edit', 'wikicode'), $link, navigation_node::TYPE_SETTING);
        }

        if (has_capability('mod/wikicode:viewcomment', $context)) {
            //$link = new moodle_url('/mod/wikicode/comments.php', array('pageid' => $pageid));
            //$node = $navref->add(get_string('comments', 'wikicode'), $link, navigation_node::TYPE_SETTING);
        }

        if (has_capability('mod/wikicode:viewpage', $context)) {
            $link = new moodle_url('/mod/wikicode/history.php', array('pageid' => $pageid));
            $node = $navref->add(get_string('history', 'wikicode'), $link, navigation_node::TYPE_SETTING);
        }

        if (has_capability('mod/wikicode:viewpage', $context)) {
            //$link = new moodle_url('/mod/wikicode/map.php', array('pageid' => $pageid));
            //$node = $navref->add(get_string('map', 'wikicode'), $link, navigation_node::TYPE_SETTING);
        }

        if (has_capability('mod/wikicode:viewpage', $context)) {
            //$link = new moodle_url('/mod/wikicode/files.php', array('pageid' => $pageid));
            //$node = $navref->add(get_string('files', 'wikicode'), $link, navigation_node::TYPE_SETTING);
        }

		if (has_capability('mod/wikicode:viewpage', $context)) {
            $link = new moodle_url('/mod/wikicode/log.php', array('pageid' => $pageid));
            $node = $navref->add(get_string('log', 'wikicode'), $link, navigation_node::TYPE_SETTING);
        }

        if (has_capability('mod/wikicode:managewiki', $context)) {
            $link = new moodle_url('/mod/wikicode/admin.php', array('pageid' => $pageid));
            $node = $navref->add(get_string('admin', 'wikicode'), $link, navigation_node::TYPE_SETTING);
        }
    }
}
/**
 * Returns all other caps used in wiki module
 *
 * @return array
 */
function wikicode_get_extra_capabilities() {
    return array('moodle/comment:view', 'moodle/comment:post', 'moodle/comment:delete');
}

/**
 * Running addtional permission check on plugin, for example, plugins
 * may have switch to turn on/off comments option, this callback will
 * affect UI display, not like pluginname_comment_validate only throw
 * exceptions.
 * Capability check has been done in comment->check_permissions(), we
 * don't need to do it again here.
 *
 * @param stdClass $comment_param {
 *              context  => context the context object
 *              courseid => int course id
 *              cm       => stdClass course module object
 *              commentarea => string comment area
 *              itemid      => int itemid
 * }
 * @return array
 */
function wikicode_comment_permissions($comment_param) {
    return array('post'=>true, 'view'=>true);
}

/**
 * Validate comment parameter before perform other comments actions
 *
 * @param stdClass $comment_param {
 *              context  => context the context object
 *              courseid => int course id
 *              cm       => stdClass course module object
 *              commentarea => string comment area
 *              itemid      => int itemid
 * }
 * @return boolean
 */
function wikicode_comment_validate($comment_param) {
    global $CFG;
    require_once($CFG->dirroot . '/mod/wikicode/locallib.php');
    // validate comment area
    if ($comment_param->commentarea != 'wikicode_page') {
        throw new comment_exception('invalidcommentarea');
    }
    // validate itemid
    if (!$record = get_record('wikicode_pages', 'id', $comment_param->itemid)) {
        throw new comment_exception('invalidcommentitemid');
    }
    if (!$subwiki = wikicode_get_subwiki($record->subwikiid)) {
        throw new comment_exception('invalidsubwikiid');
    }
    if (!$wiki = wikicode_get_wikicode_from_pageid($comment_param->itemid)) {
        throw new comment_exception('invalidid', 'data');
    }
    if (!$course = get_record('course', 'id', $wiki->course)) {
        throw new comment_exception('coursemisconf');
    }
    if (!$cm = get_coursemodule_from_instance('wikicode', $wiki->id, $course->id)) {
        throw new comment_exception('invalidcoursemodule');
    }
    $context = get_context_instance(CONTEXT_MODULE, $cm->id);
    // group access
    if ($subwiki->groupid) {
        $groupmode = groups_get_activity_groupmode($cm, $course);
        if ($groupmode == SEPARATEGROUPS and !has_capability('moodle/site:accessallgroups', $context)) {
            if (!groups_is_member($subwiki->groupid)) {
                throw new comment_exception('notmemberofgroup');
            }
        }
    }
    // validate context id
    if ($context->id != $comment_param->context->id) {
        throw new comment_exception('invalidcontext');
    }
    // validation for comment deletion
    if (!empty($comment_param->commentid)) {
        if ($comment = get_record('comments', 'id', $comment_param->commentid)) {
            if ($comment->commentarea != 'wikicode_page') {
                throw new comment_exception('invalidcommentarea');
            }
            if ($comment->contextid != $context->id) {
                throw new comment_exception('invalidcontext');
            }
            if ($comment->itemid != $comment_param->itemid) {
                throw new comment_exception('invalidcommentitemid');
            }
        } else {
            throw new comment_exception('invalidcommentid');
        }
    }
    return true;
}

/**
 * Return a list of page types
 * @param string $pagetype current page type
 * @param stdClass $parentcontext Block's parent context
 * @param stdClass $currentcontext Current context of block
 */
function wikicode_page_type_list($pagetype, $parentcontext, $currentcontext) {
    $module_pagetype = array(
        'mod-wiki-*'=>get_string('page-mod-wiki-x', 'wikicode'),
        'mod-wiki-view'=>get_string('page-mod-wiki-view', 'wikicode'),
        'mod-wiki-comments'=>get_string('page-mod-wiki-comments', 'wikicode'),
        'mod-wiki-history'=>get_string('page-mod-wiki-history', 'wikicode'),
        'mod-wiki-map'=>get_string('page-mod-wiki-map', 'wikicode')
    );
    return $module_pagetype;
}
\end{lstlisting}

\subsection{localib.php}
\begin{lstlisting}[language=PHP]
<?php
require_once($CFG->dirroot . '/mod/wikicode/lib.php');
require_once($CFG->libdir . '/filelib.php');
require_once($CFG->libdir.'/adminlib.php');

define('WIKI_REFRESH_CACHE_TIME', 30); // @TODO: To be deleted.
define('FORMAT_CREOLE', '37');
define('FORMAT_NWIKI', '38');
define('FORMAT_WCODE', '39');
define('NO_VALID_RATE', '-999');
define('IMPROVEMENT', '+');
define('EQUAL', '=');
define('WORST', '-');

define('LOCK_TIMEOUT', 30);

/**
 * Get a wiki instance
 * @param int $wikiid the instance id of wiki
 */
function wikicode_get_wiki($wikiid) {
    
    return get_record('wikicode', 'id', $wikiid);
}

/**
 * Get sub wiki instances with same wiki id
 * @param int $wikiid
 */
function wikicode_get_subwikis($wikiid) {
        return get_records('wikicode_subwikis', 'wikiid', $wikiid);
}

/**
 * Get a sub wiki instance by wiki id and group id
 * @param int $wikiid
 * @param int $groupid
 * @return object
 */
function wikicode_get_subwiki_by_group($wikiid, $groupid, $userid = 0) {
        return get_record('wikicode_subwikis', 'wikiid', $wikiid, 'groupid', $groupid, 'userid', $userid);
}

/**
 * Get a sub wiki instace by instance id
 * @param int $subwikiid
 * @return object
 */
function wikicode_get_subwiki($subwikiid) {
        return get_record('wikicode_subwikis', 'id', $subwikiid);

}

/**
 * Add a new sub wiki instance
 * @param int $wikiid
 * @param int $groupid
 * @return int $insertid
 */
function wikicode_add_subwiki($wikiid, $groupid, $userid = 0) {
    
    $record = new StdClass();
    $record->wikiid = $wikiid;
    $record->groupid = $groupid;
    $record->userid = $userid;

    $insertid = insert_record('wikicode_subwikis', $record);
    return $insertid;
}

/**
 * Get a wiki instance by pageid
 * @param int $pageid
 * @return object
 */
function wikicode_get_wikicode_from_pageid($pageid) {
	
	global $CFG;
    
    $sql = "SELECT w.*
            FROM {$CFG->prefix}wikicode w, {$CFG->prefix}wikicode_subwikis s, {$CFG->prefix}wikicode_pages p
            WHERE p.id = ".$pageid." AND
            p.subwikiid = s.id AND
            s.wikiid = w.id";

    return get_record_sql($sql);
}

/**
 * Get gcc path most appearances
 * @return object
 */
function wikicode_get_gccpath() {
	
	global $CFG;

    $sql = "SELECT w.gccpath
            FROM {$CFG->prefix}wikicode w
            GROUP BY w.gccpath ORDER BY w.gccpath DESC
            LIMIT 1";

    return get_record_sql($sql);
}

/**
 * Get mingw path most appearances
 * @return object
 */
function wikicode_get_mingwpath() {
	
	global $CFG;

    $sql = "SELECT w.mingwpath
            FROM {$CFG->prefix}wikicode w
            GROUP BY w.mingwpath ORDER BY w.mingwpath DESC
            LIMIT 1";

    return get_record_sql($sql);
}

/**
 * Get a wiki page by pageid
 * @param int $pageid
 * @return object
 */
function wikicode_get_page($pageid) {
        return get_record('wikicode_pages', 'id', $pageid);
}

/**
 * Get latest version of wiki page
 * @param int $pageid
 * @return object
 */
function wikicode_get_current_version($pageid) {
		
	global $CFG;
    
    // @TODO: Fix this query
    $sql = "SELECT *
            FROM {$CFG->prefix}wikicode_versions
            WHERE pageid = ".$pageid."
            ORDER BY version DESC";
			
    return array_pop(get_records_sql($sql, 0, 1));

}

/**
 * Alias of wikicode_get_current_version
 * @TODO, does the exactly same thing as wikicode_get_current_version, should be removed
 * @param int $pageid
 * @return object
 */
function wikicode_get_last_version($pageid) {
    return wikicode_get_current_version($pageid);
}

/**
 * Get page section
 * @param int $pageid
 * @param string $section
 */
function wikicode_get_section_page($page, $section) {

    $version = wikicode_get_current_version($page->id);
    return wikicode_parser_proxy::get_section($version->content, $version->contentformat, $section);
}

/**
 * Get a wiki page by page title
 * @param int $swid, sub wiki id
 * @param string $title
 * @return object
 */
function wikicode_get_page_by_title($swid, $title) {
        return get_record('wikicode_pages', 'subwikiid', $swid, 'title', $title);
}

/**
 * Get a version record by record id
 * @param int $versionid, the version id
 * @return object
 */
function wikicode_get_version($versionid) {
        return get_record('wikicode_versions', 'id', $versionid);
}

/**
 * Get first page of wiki instace
 * @param int $subwikiid
 * @param int $module, wiki instance object
 */
function wikicode_get_first_page($subwikid, $module = null) {
    global $USER, $CFG;

    $sql = "SELECT p.*
            FROM {$CFG->prefix}wikicode w, {$CFG->prefix}wikicode_subwikis s, {$CFG->prefix}wikicode_pages p
            WHERE s.id = ".$subwikid." AND
            s.wikiid = w.id AND
            w.firstpagetitle = p.title AND
            p.subwikiid = s.id";
    return get_record_sql($sql);
}

function wikicode_save_section($wikipage, $sectiontitle, $sectioncontent, $userid) {

    $wiki = wikicode_get_wikicode_from_pageid($wikipage->id);
    $cm = get_coursemodule_from_instance('wikicode', $wiki->id);
    $context = get_context_instance(CONTEXT_MODULE, $cm->id);

    if (has_capability('mod/wikicode:editpage', $context)) {
        $version = wikicode_get_current_version($wikipage->id);
        $content = wikicode_parser_proxy::get_section($version->content, $version->contentformat, $sectiontitle, true);

        $newcontent = $content[0] . $sectioncontent . $content[2];

        return wikicode_save_page($wikipage, $newcontent, $userid);
    } else {
        return false;
    }
}

/**
 * Save page content
 * @param object $wikipage
 * @param string $newcontent
 * @param int $userid
 */
function wikicode_save_page($wikipage, $newcontent, $userid) {
        
    $wiki = wikicode_get_wikicode_from_pageid($wikipage->id);
    $cm = get_coursemodule_from_instance('wikicode', $wiki->id);
    $context = get_context_instance(CONTEXT_MODULE, $cm->id);

    if (has_capability('mod/wikicode:editpage', $context)) {
        $version = wikicode_get_current_version($wikipage->id);
		

        $version->content = $newcontent;
        $version->userid = $userid;
        $version->version++;
        $version->timecreated = time();
        $versionid = insert_record('wikicode_versions', $version);

        $wikipage->timemodified = $version->timecreated;
        $wikipage->userid = $userid;
        $return = wikicode_refresh_cachedcontent($wikipage, $newcontent);

        return $return;
    } else {
        return false;
    }
}

/**
 * Compile page content
 * @param object $wikipage
 * @param string $newcontent
 * @param int $userid
 */
function wikicode_compile_page($wikipage, $newcontent, $userid, $download) {
        
    $wiki = wikicode_get_wikicode_from_pageid($wikipage->id);
    $cm = get_coursemodule_from_instance('wikicode', $wiki->id);
    $context = get_context_instance(CONTEXT_MODULE, $cm->id);

    if (has_capability('mod/wikicode:editpage', $context)) {
    	
		//Creamos el fichero
		$fileC = fopen("code.c","w"); 
		$codigo = wikicode_remove_tags($newcontent);
		$codigo = str_replace("\\", "", $codigo);
		fputs($fileC,$codigo); 
		fclose($fileC);
		
		//Detectamos el OS
		$userAgent = $_SERVER[HTTP_USER_AGENT];
		$userAgent = strtolower ($userAgent);
		
		//Llamamos al compilador correspondiente
		if(strpos($userAgent, "windows") !== false) {
        	$output=system($wiki->mingwpath . " code.c -o moodle.exe -w 2> output.txt");
			$ejecutable='moodle.exe';
		}
		else {
			$output=system($wiki->gccpath . " code.c -o moodle.out -w 2> output.txt");
			$ejecutable='moodle.out';
		}
		
		$file_output = fopen ("output.txt", "r");
   		$output = fread($file_output, filesize("output.txt"));
            	$output = utf8_encode($output);
		
		if ( $output == '0' || $output == '') {
			$output = 'Compilacion correcta';
		} else {
			$wikipage->errorcompile++;
		}
		
        //Guardamos los datos
        $wikipage->cachedcompile = $newcontent;
		$wikipage->cachedgcc = $output;
		update_record('wikicode_pages', $wikipage);
		
		//Descargamos el ejecutable
		if ($output == 'Compilacion correcta' and $download == TRUE) {
			echo "<script language='JavaScript'>window.open('".$ejecutable."')</script>";
		}

        return array('page' => $wikipage);
    } else {
        return false;
    }
}

function wikicode_remove_tags($newcontent) {
	
	$codigo = str_replace("</!>", "", $newcontent);
	
	$desde = strpos($codigo,"<!");
	
	while (is_numeric($desde)) {
		$hasta = strpos($codigo, ">", $desde);
		$codigo = substr_replace($codigo, '', $desde, $hasta - $desde + 1);
		
		$desde = strpos($codigo,"<!");
	}
	
	return $codigo;
}

function wikicode_remove_tags_owner($codigo) {
	
	global $USER;
	
	$desde = strpos($codigo,"<!".$USER->username.">");
	
	while (is_numeric($desde)) {
		$codigoaux = substr($codigo, $desde);
		$desdeaux = strpos($codigoaux, "</!>");
		
		$codigo = substr_replace($codigo, '', $desde + $desdeaux, 4);
		$codigo = substr_replace($codigo, '', $desde, strlen("<!".$USER->username.">"));
		
		$desde = strpos($codigo,"<!".$USER->username.">");
	}
	
	return $codigo;
}

function wikicode_refresh_cachedcontent($page, $newcontent = null) {
    
    $version = wikicode_get_current_version($page->id);
    if (empty($version)) {
        return null;
    }
    if (!isset($newcontent)) {
        $newcontent = $version->content;
    }

    $options = array('swid' => $page->subwikiid, 'pageid' => $page->id);

	if ($veresion->contentformat = 'nwiki') {
		$page->cachedcontent = $version->content;
	}
	else {
		$page->cachedcontent = $parseroutput['toc'] . $parseroutput['parsed_text'];
	}
    
    $page->timerendered = time();
    update_record('wikicode_pages', $page);

    wikicode_refresh_page_links($page, $parseroutput['link_count']);

    return array('page' => $page, 'sections' => $parseroutput['repeated_sections'], 'version' => $version->version);
}
/**
 * Restore a page
 */
function wikicode_restore_page($wikipage, $newcontent, $userid) {
    $return = wikicode_save_page($wikipage, $newcontent, $userid);
    return $return['page'];
}

function wikicode_refresh_page_links($page, $links) {
    
    delete_records('wikicode_links', 'frompageid', $page->id);
    foreach ($links as $linkname => $linkinfo) {

        $newlink = new stdClass();
        $newlink->subwikiid = $page->subwikiid;
        $newlink->frompageid = $page->id;

        if ($linkinfo['new']) {
            $newlink->tomissingpage = $linkname;
        } else {
            $newlink->topageid = $linkinfo['pageid'];
        }

        try {
            insert_record('wikicode_links', $newlink);
        } catch (dml_exception $e) {
            debugging($e->getMessage());
        }

    }
}

/**
 * Create a new wiki page, if the page exists, return existing pageid
 * @param int $swid
 * @param string $title
 * @param string $format
 * @param int $userid
 */
function wikicode_create_page($swid, $title, $format, $userid) {
    global $PAGE;
    $subwiki = wikicode_get_subwiki($swid);
    $cm = get_coursemodule_from_instance('wikicode', $subwiki->wikiid);
    $context = get_context_instance(CONTEXT_MODULE, $cm->id);
    require_capability('mod/wikicode:editpage', $context);
    // if page exists
    if ($page = wikicode_get_page_by_title($swid, $title)) {
        return $page->id;
    }

    // Creating a new empty version
    $version = new stdClass();
    $version->content = '';
    $version->contentformat = $format;
    $version->version = 0;
    $version->timecreated = time();
    $version->userid = $userid;

    $versionid = null;
    $versionid = insert_record('wikicode_versions', $version);

    // Createing a new empty page
    $page = new stdClass();
    $page->subwikiid = $swid;
    $page->title = $title;
    $page->cachedcontent = '';
    $page->timecreated = $version->timecreated;
    $page->timemodified = $version->timecreated;
    $page->timerendered = $version->timecreated;
    $page->userid = $userid;
    $page->pageviews = 0;
    $page->readonly = 0;
	$page->cachedcompile = '';
	$page->cachedgcc = '';
	$page->errorcompile = 0;
	$page->timestartedit = 0;
	$page->timeendedit = 0;

    $pageid = insert_record('wikicode_pages', $page);

    // Setting the pageid
    $version->id = $versionid;
    $version->pageid = $pageid;
    update_record('wikicode_versions', $version);

    wikicode_make_cache_expire($page->title);
    return $pageid;
}

function wikicode_make_cache_expire($pagename) {
	
	global $CFG;
    
    $sql = "UPDATE {$CFG->prefix}wikicode_pages
            SET timerendered = 0
            WHERE id IN ( SELECT l.frompageid
                FROM {$CFG->prefix}wikicode_links l
                WHERE l.tomissingpage = ".$pagename."
            )";
    execute_sql ($sql);
}

/**
 * Get a specific version of page
 * @param int $pageid
 * @param int $version
 */
function wikicode_get_wikicode_page_version($pageid, $version) {
        return get_record('wikicode_versions', 'pageid', $pageid, 'version', $version);
}

/**
 * Get version list
 * @param int $pageid
 * @param int $limitfrom
 * @param int $limitnum
 */
function wikicode_get_wikicode_page_versions($pageid, $limitfrom, $limitnum) {
        return get_records('wikicode_versions', 'pageid', $pageid, 'version DESC', '*', $limitfrom, $limitnum);
}

/**
 * Count the number of page version
 * @param int $pageid
 */
function wikicode_count_wikicode_page_versions($pageid) {
        return count_records('wikicode_versions', 'pageid', $pageid);
}

/**
 * Get linked from page
 * @param int $pageid
 */
function wikicode_get_linked_to_pages($pageid) {
        return get_records('wikicode_links', 'frompageid', $pageid);
}

/**
 * Get linked from page
 * @param int $pageid
 */
function wikicode_get_linked_from_pages($pageid) {
        return get_records('wikicode_links', 'topageid', $pageid);
}

/**
 * Get pages which user have been edited
 * @param int $swid
 * @param int $userid
 */
function wikicode_get_contributions($swid, $userid) {
	
	global $CFG;
    
    $sql = "SELECT v.*
            FROM {$CFG->prefix}wikicode_versions v, {$CFG->prefix}wikicode_pages p
            WHERE p.subwikiid = ".$swid." AND
            v.pageid = p.id AND
            v.userid = ".$userid."";

    return get_records_sql($sql);
}

/**
 * Get missing or empty pages in wiki
 * @param int $swid sub wiki id
 */
function wikicode_get_missing_or_empty_pages($swid) {
	
	global $CFG;
    
    $sql = "SELECT DISTINCT p.title, p.id, p.subwikiid
            FROM {$CFG->prefix}wikicode w, {$CFG->prefix}wikicode_subwikis s, {$CFG->prefix}wikicode_pages p
            WHERE s.wikiid = w.id and
            s.id = ".$swid." and
            w.firstpagetitle != p.title and
            p.subwikiid = ".$swid." and
            1 =  (SELECT count(*)
                FROM {$CFG->prefix}wikicode_versions v
                WHERE v.pageid = p.id)
            UNION
            SELECT DISTINCT l.tomissingpage as title, 0 as id, l.subwikiid
            FROM {$CFG->prefix}wikicode_links l
            WHERE l.subwikiid = ".$swid." and
            l.topageid = 0";

    return get_records_sql($sql);
}

/**
 * Get pages list in wiki
 * @param int $swid sub wiki id
 */
function wikicode_get_page_list($swid) {
        $records = get_records('wikicode_pages', 'subwikiid', $swid, 'title ASC');
    return $records;
}

/**
 * Return a list of orphaned wikis for one specific subwiki
 * @global object
 * @param int $swid sub wiki id
 */
function wikicode_get_orphaned_pages($swid) {
	
	global $CFG;
    
    $sql = "SELECT p.id, p.title
            FROM {$CFG->prefix}wikicode_pages p, {$CFG->prefix}wikicode w , {$CFG->prefix}wikicode_subwikis s
            WHERE p.subwikiid = ".$swid."
            AND s.id = ".$swid."
            AND w.id = s.wikiid
            AND p.title != w.firstpagetitle
            AND p.id NOT IN (SELECT topageid FROM {$CFG->prefix}wikicode_links WHERE subwikiid = ".$swid.");";

    return get_records_sql($sql);
}

/**
 * Search wiki title
 * @param int $swid sub wiki id
 * @param string $search
 */
function wikicode_search_title($swid, $search) {
    
    return get_records_select('wikicode_pages', "subwikiid = ? AND title LIKE ?", $swid, '%'.$search.'%');
}

/**
 * Search wiki content
 * @param int $swid sub wiki id
 * @param string $search
 */
function wikicode_search_content($swid, $search) {
    
    return get_records_select('wikicode_pages', "subwikiid = ? AND cachedcontent LIKE ?", $swid, '%'.$search.'%');
}

/**
 * Search wiki title and content
 * @param int $swid sub wiki id
 * @param string $search
 */
function wikicode_search_all($swid, $search) {
    
    return get_records_select('wikicode_pages', "subwikiid = ? AND (cachedcontent LIKE ? OR title LIKE ?)", $swid, '%'.$search.'%', '%'.$search.'%');
}

/**
 * Get user data
 */
function wikicode_get_user_info($userid) {
        return get_record('user', 'id', $userid);
}

/**
 * Increase page view nubmer
 * @param int $page, database record
 */
function wikicode_increment_pageviews($page) {
    
    $page->pageviews++;
    update_record('wikicode_pages', $page);
}

//----------------------------------------------------------
//----------------------------------------------------------

/**
 * Text format supported by wiki module
 */
function wikicode_get_formats() {
    return array('wcode');
}

/**
 * Parses a string with the wiki markup language in $markup.
 *
 * @return Array or false when something wrong has happened.
 *
 * Returned array contains the following fields:
 *     'parsed_text', String. Contains the parsed wiki content.
 *     'unparsed_text', String. Constains the original wiki content.
 *     'link_count', Array of 'destination', ..., 'new', "is new?"). Contains the internal wiki links found in the wiki content.
 *      'deleted_sections', the list of deleted sections.
 *              '' =>
 *
 **/
function wikicode_parse_content($markup, $pagecontent, $options = array()) {
    global $PAGE;

    $subwiki = wikicode_get_subwiki($options['swid']);
    $cm = get_coursemodule_from_instance("wikicode", $subwiki->wikiid);
    $context = get_context_instance(CONTEXT_MODULE, $cm->id);

    $parser_options = array(
        'link_callback' => '/mod/wikicode/locallib.php:wiki_parser_link',
        'link_callback_args' => array('swid' => $options['swid']),
        'table_callback' => '/mod/wikicode/locallib.php:wiki_parser_table',
        'real_path_callback' => '/mod/wikicode/locallib.php:wiki_parser_real_path',
        'real_path_callback_args' => array(
            'context' => $context,
            'component' => 'mod_wikicode',
            'filearea' => 'attachments',
            'subwikiid'=> $subwiki->id,
            'pageid' => $options['pageid']
        ),
        'pageid' => $options['pageid'],
        'pretty_print' => (isset($options['pretty_print']) && $options['pretty_print']),
        'printable' => (isset($options['printable']) && $options['printable'])
    );

    return wikicode_parser_proxy::parse($pagecontent, $markup, $parser_options);
}

/**
 * This function is the parser callback to parse wiki links.
 *
 * It returns the necesary information to print a link.
 *
 * NOTE: Empty pages and non-existent pages must be print in red color.
 *
 * @param link name of a page
 * @param $options
 *
 * @return
 *
 * @TODO Doc return and options
 */
function wikicode_parser_link($link, $options = null) {
    global $CFG;

    if (is_object($link)) {
        $parsedlink = array('content' => $link->title, 'url' => $CFG->wwwroot . '/mod/wikicode/view.php?pageid=' . $link->id, 'new' => false, 'link_info' => array('link' => $link->title, 'pageid' => $link->id, 'new' => false));

        $version = wikicode_get_current_version($link->id);
        if ($version->version == 0) {
            $parsedlink['new'] = true;
        }
        return $parsedlink;
    } else {
        $swid = $options['swid'];

        if ($page = wikicode_get_page_by_title($swid, $link)) {
            $parsedlink = array('content' => $link, 'url' => $CFG->wwwroot . '/mod/wikicode/view.php?pageid=' . $page->id, 'new' => false, 'link_info' => array('link' => $link, 'pageid' => $page->id, 'new' => false));

            $version = wikicode_get_current_version($page->id);
            if ($version->version == 0) {
                $parsedlink['new'] = true;
            }

            return $parsedlink;

        } else {
            return array('content' => $link, 'url' => $CFG->wwwroot . '/mod/wikicode/create.php?swid=' . $swid . '&amp;title=' . urlencode($link) . '&amp;action=new', 'new' => true, 'link_info' => array('link' => $link, 'new' => true, 'pageid' => 0));
        }
    }
}

/**
 * Returns the table fully parsed (HTML)
 *
 * @return HTML for the table $table
 *
 **/
function wikicode_parser_table($table) {
    global $OUTPUT;

    $htmltable = new html_table();

    $headers = $table[0];
    $htmltable->head = array();
    foreach ($headers as $h) {
        $htmltable->head[] = $h[1];
    }

    array_shift($table);
    $htmltable->data = array();
    foreach ($table as $row) {
        $row_data = array();
        foreach ($row as $r) {
            $row_data[] = $r[1];
        }
        $htmltable->data[] = $row_data;
    }

    return html_writer::table($htmltable);
}

/**
 * Returns an absolute path link, unless there is no such link.
 *
 * @param string $url Link's URL or filename
 * @param stdClass $context filearea params
 * @param string $component The component the file is associated with
 * @param string $filearea The filearea the file is stored in
 * @param int $swid Sub wiki id
 *
 * @return string URL for files full path
 */

function wikicode_parser_real_path($url, $context, $component, $filearea, $swid) {
    global $CFG;

    if (preg_match("/^(?:http|ftp)s?\:\/\//", $url)) {
        return $url;
    } else {

        $file = 'pluginfile.php';
        if (!$CFG->slasharguments) {
            $file = $file . '?file=';
        }
        $baseurl = "$CFG->wwwroot/$file/{$context->id}/$component/$filearea/$swid/";
        // it is a file in current file area
        return $baseurl . $url;
    }
}

/**
 * Returns the token used by a wiki language to represent a given tag or "object" (bold -> **)
 *
 * @return A string when it has only one token at the beginning (f. ex. lists). An array composed by 2 strings when it has 2 tokens, one at the beginning and one at the end (f. ex. italics). Returns false otherwise.
 **/
function wikicode_parser_get_token($markup, $name) {

    return wikicode_parser_proxy::get_token($name, $markup);
}

/**
 * Checks if current user can view a subwiki
 *
 * @param $subwiki
 */
function wikicode_user_can_view($subwiki) {
    global $USER;

    $wiki = wikicode_get_wiki($subwiki->wikiid);
    $cm = get_coursemodule_from_instance('wikicode', $wiki->id);
    $context = get_context_instance(CONTEXT_MODULE, $cm->id);

    // Working depending on activity groupmode
    switch (groups_get_activity_groupmode($cm)) {
    case NOGROUPS:

        if ($wiki->wikimode == 'collaborative') {
            // Collaborative Mode:
            // There is one wiki for all the class.
            //
            // Only view capbility needed
            return has_capability('mod/wikicode:viewpage', $context);
        } else if ($wiki->wikimode == 'individual') {
            // Individual Mode:
            // Each person owns a wiki.
            if ($subwiki->userid == $USER->id) {
                // Only the owner of the wiki can view it
                return has_capability('mod/wikicode:viewpage', $context);
            } else { // User has special capabilities
                // User must have:
                //      mod/wiki:viewpage capability
                // and
                //      mod/wiki:managewiki capability
                $view = has_capability('mod/wikicode:viewpage', $context);
                $manage = has_capability('mod/wikicode:managewiki', $context);

                return $view && $manage;
            }
        } else {
            //Error
            return false;
        }
    case SEPARATEGROUPS:
        // Collaborative and Individual Mode
        //
        // Collaborative Mode:
        //      There is one wiki per group.
        // Individual Mode:
        //      Each person owns a wiki.
        if ($wiki->wikimode == 'collaborative' || $wiki->wikimode == 'individual') {
            // Only members of subwiki group could view that wiki
            if ($subwiki->groupid == groups_get_activity_group($cm)) {
                // Only view capability needed
                return has_capability('mod/wikicode:viewpage', $context);

            } else { // User is not part of that group
                // User must have:
                //      mod/wiki:managewiki capability
                // or
                //      moodle/site:accessallgroups capability
                // and
                //      mod/wiki:viewpage capability
                $view = has_capability('mod/wikicode:viewpage', $context);
                $manage = has_capability('mod/wikicode:managewiki', $context);
                $access = has_capability('moodle/site:accessallgroups', $context);
                return ($manage || $access) && $view;
            }
        } else {
            //Error
            return false;
        }
    case VISIBLEGROUPS:
        // Collaborative and Individual Mode
        //
        // Collaborative Mode:
        //      There is one wiki per group.
        // Individual Mode:
        //      Each person owns a wiki.
        if ($wiki->wikimode == 'collaborative' || $wiki->wikimode == 'individual') {
            // Everybody can read all wikis
            //
            // Only view capability needed
            return has_capability('mod/wikicode:viewpage', $context);
        } else {
            //Error
            return false;
        }
    default: // Error
        return false;
    }
}

/**
 * Checks if current user can edit a subwiki
 *
 * @param $subwiki
 */
function wikicode_user_can_edit($subwiki) {
    global $USER;

    $wiki = wikicode_get_wiki($subwiki->wikiid);
    $cm = get_coursemodule_from_instance('wikicode', $wiki->id);
    $context = get_context_instance(CONTEXT_MODULE, $cm->id);

    // Working depending on activity groupmode
    switch (groups_get_activity_groupmode($cm)) {
    case NOGROUPS:

        if ($wiki->wikimode == 'collaborative') {
            // Collaborative Mode:
            // There is a wiki for all the class.
            //
            // Only edit capbility needed
            return has_capability('mod/wikicode:editpage', $context);
        } else if ($wiki->wikimode == 'individual') {
            // Individual Mode
            // There is a wiki per user

            // Only the owner of that wiki can edit it
            if ($subwiki->userid == $USER->id) {
                return has_capability('mod/wikicode:editpage', $context);
            } else { // Current user is not the owner of that wiki.

                // User must have:
                //      mod/wiki:editpage capability
                // and
                //      mod/wiki:managewiki capability
                $edit = has_capability('mod/wikicode:editpage', $context);
                $manage = has_capability('mod/wikicode:managewiki', $context);

                return $edit && $manage;
            }
        } else {
            //Error
            return false;
        }
    case SEPARATEGROUPS:
        if ($wiki->wikimode == 'collaborative') {
            // Collaborative Mode:
            // There is one wiki per group.
            //
            // Only members of subwiki group could edit that wiki
            if ($subwiki->groupid == groups_get_activity_group($cm)) {
                // Only edit capability needed
                return has_capability('mod/wikicode:editpage', $context);
            } else { // User is not part of that group
                // User must have:
                //      mod/wiki:managewiki capability
                // and
                //      moodle/site:accessallgroups capability
                // and
                //      mod/wiki:editpage capability
                $manage = has_capability('mod/wikicode:managewiki', $context);
                $access = has_capability('moodle/site:accessallgroups', $context);
                $edit = has_capability('mod/wikicode:editpage', $context);
                return $manage && $access && $edit;
            }
        } else if ($wiki->wikimode == 'individual') {
            // Individual Mode:
            // Each person owns a wiki.
            //
            // Only the owner of that wiki can edit it
            if ($subwiki->userid == $USER->id) {
                return has_capability('mod/wikicode:editpage', $context);
            } else { // Current user is not the owner of that wiki.
                // User must have:
                //      mod/wiki:managewiki capability
                // and
                //      moodle/site:accessallgroups capability
                // and
                //      mod/wiki:editpage capability
                $manage = has_capability('mod/wikicode:managewiki', $context);
                $access = has_capability('moodle/site:accessallgroups', $context);
                $edit = has_capability('mod/wikicode:editpage', $context);
                return $manage && $access && $edit;
            }
        } else {
            //Error
            return false;
        }
    case VISIBLEGROUPS:
        if ($wiki->wikimode == 'collaborative') {
            // Collaborative Mode:
            // There is one wiki per group.
            //
            // Only members of subwiki group could edit that wiki
            if (groups_is_member($subwiki->groupid)) {
                // Only edit capability needed
                return has_capability('mod/wikicode:editpage', $context);
            } else { // User is not part of that group
                // User must have:
                //      mod/wiki:managewiki capability
                // and
                //      mod/wiki:editpage capability
                $manage = has_capability('mod/wikicode:managewiki', $context);
                $edit = has_capability('mod/wikicode:editpage', $context);
                return $manage && $edit;
            }
        } else if ($wiki->wikimode == 'individual') {
            // Individual Mode:
            // Each person owns a wiki.
            //
            // Only the owner of that wiki can edit it
            if ($subwiki->userid == $USER->id) {
                return has_capability('mod/wikicode:editpage', $context);
            } else { // Current user is not the owner of that wiki.
                // User must have:
                //      mod/wiki:managewiki capability
                // and
                //      mod/wiki:editpage capability
                $manage = has_capability('mod/wikicode:managewiki', $context);
                $edit = has_capability('mod/wikicode:editpage', $context);
                return $manage && $edit;
            }
        } else {
            //Error
            return false;
        }
    default: // Error
        return false;
    }
}

//----------------
// Locks
//----------------

/**
 * Checks if a page-section is locked.
 *
 * @return true if the combination of section and page is locked, FALSE otherwise.
 */
function wikicode_is_page_section_locked($pageid, $userid, $section = null) {
    
    $sql = "pageid = ".$pageid." AND lockedat > ".time()." AND userid != ".$userid;

    if (!empty($section)) {
        $sql .= " AND (sectionname = ".$section." OR sectionname IS null)";
    }

    return record_exists_select('wikicode_locks', $sql);
}

/**
 * Inserts or updates a wikicode_locks record.
 */
function wikicode_set_lock($pageid, $userid, $section = null, $insert = false) {
    
    if (wikicode_is_page_section_locked($pageid, $userid, $section)) {
        return false;
    }

    $lock = get_record('wikicode_locks', 'pageid', $pageid, 'userid', $userid, 'sectionname', $section);

    if (!empty($lock)) {
        update_record('wikicode_locks', array('id' => $lock->id, 'lockedat' => time() + LOCK_TIMEOUT));
    } else if ($insert) {
        insert_record('wikicode_locks', array('pageid' => $pageid, 'sectionname' => $section, 'userid' => $userid, 'lockedat' => time() + 30));
    }

    return true;
}

/**
 * Deletes wikicode_locks that are not in use. (F.Ex. after submitting the changes). If no userid is present, it deletes ALL the wikicode_locks of a specific page.
 */
function wikicode_delete_locks($pageid, $userid = null, $section = null, $delete_from_db = true, $delete_section_and_page = false) {
    
    
}

/**
 * Deletes wikicode_locks that expired 1 hour ago.
 */
function wikicode_delete_old_locks() {
    
    delete_records_select('wikicode_locks', "lockedat < ?", time() - 3600);
}

/**
 * Deletes wikicode_links. It can be sepecific link or links attached in subwiki
 *
 * @global mixed $DB database object
 * @param int $linkid id of the link to be deleted
 * @param int $topageid links to the specific page
 * @param int $frompageid links from specific page
 * @param int $subwikiid links to subwiki
 */
function wikicode_delete_links($linkid = null, $topageid = null, $frompageid = null, $subwikiid = null) {
    $params = array();

    // if link id is givien then don't check for anything else
    if (!empty($linkid)) {
        $params['id'] = $linkid;
    } else {
        if (!empty($topageid)) {
            $params['topageid'] = $topageid;
        }
        if (!empty($frompageid)) {
            $params['frompageid'] = $frompageid;
        }
        if (!empty($subwikiid)) {
            $params['subwikiid'] = $subwikiid;
        }
    }

    //Delete links if any params are passed, else nothing to delete.
    if (!empty($params)) {
        delete_records('wikicode_links', $params);
    }
}

/**
 * Delete wiki synonyms related to subwikiid or page
 *
 * @param int $subwikiid id of sunbwiki
 * @param int $pageid id of page
 */
function wikicode_delete_synonym($subwikiid, $pageid = null) {
    
    if (!is_null($pageid)) {
        $params['pageid'] = $pageid;
		delete_records('wikicode_synonyms', 'subwikiid', $subwikiid, 'pageid', $pageid);
    } else {
    	delete_records('wikicode_synonyms', 'subwikiid', $subwikiid);
	}
}

/**
 * Delete pages and all related data
 *
 * @param mixed $context context in which page needs to be deleted.
 * @param mixed $pageids id's of pages to be deleted
 * @param int $subwikiid id of the subwiki for which all pages should be deleted
 */
function wikicode_delete_pages($context, $pageids = null, $subwikiid = null) {
    
    if (!empty($pageids) && is_int($pageids)) {
       $pageids = array($pageids);
    } else if (!empty($subwikiid)) {
        $pageids = wikicode_get_page_list($subwikiid);
    }

    //If there is no pageid then return as we can't delete anything.
    if (empty($pageids)) {
        return;
    }

    /// Delete page and all it's relevent data
    foreach ($pageids as $pageid) {
        if (is_object($pageid)) {
            $pageid = $pageid->id;
        }

        //Delete page comments
        $comments = wikicode_get_comments($context->id, $pageid);
        foreach ($comments as $commentid => $commentvalue) {
            wikicode_delete_comment($commentid, $context, $pageid);
        }

        //Delete page tags
        $tags = tag_get_tags_array('wikicode_pages', $pageid);
        foreach ($tags as $tagid => $tagvalue) {
            tag_delete_instance('wikicode_pages', $pageid, $tagid);
        }

        //Delete Synonym
        wikicode_delete_synonym($subwikiid, $pageid);

        //Delete all page versions
        wikicode_delete_page_versions(array($pageid=>array(0)));

        //Delete all page locks
        wikicode_delete_locks($pageid);

        //Delete all page links
        wikicode_delete_links(null, $pageid);

        //Delete page
		delete_records('wikicode_pages', 'id', $pageid);
    }
}

/**
 * Delete specificed versions of a page or versions created by users
 * if version is 0 then it will remove all versions of the page
 *
 * @param array $deleteversions delete versions for a page
 */
function wikicode_delete_page_versions($deleteversions) {
    
    /// delete page-versions
    foreach ($deleteversions as $id => $versions) {
        foreach ($versions as $version) {
            //If version = 0, then remove all versions of this page, else remove
            //specified version
            if ($version != 0) {
                delete_records('wikicode_versions', 'pageid', $id, 'version', $version);
            } else {
            	delete_records('wikicode_versions', 'pageid', $id);
			}
        }
    }
}

function wikicode_get_comment($commentid){
        return get_record('comments', 'id', $commentid);
}

/**
 * Returns all comments by context and pageid
 *
 * @param $context. Current context
 * @param $pageid. Current pageid
 **/
function wikicode_get_comments($contextid, $pageid) {
    
    return get_records('comments', 'contextid', $contextid, 'itemid', $pageid, 'commentarea', 'wikicode_page');
}

/**
 * Add comments ro database
 *
 * @param object $context. Current context
 * @param int $pageid. Current pageid
 * @param string $content. Content of the comment
 * @param string editor. Version of editor we are using.
 **/
function wikicode_add_comment($context, $pageid, $content, $editor) {
    global $CFG;
    require_once($CFG->dirroot . '/comment/lib.php');

    list($context, $course, $cm) = get_context_info_array($context->id);
    $cmt = new stdclass();
    $cmt->context = $context;
    $cmt->itemid = $pageid;
    $cmt->area = 'wikicode_page';
    $cmt->course = $course;
    $cmt->component = 'mod_wikicode';

    $manager = new comment($cmt);

    if ($editor == 'creole') {
        $manager->add($content, FORMAT_CREOLE);
    } else if ($editor == 'html') {
        $manager->add($content, FORMAT_HTML);
    } else if ($editor == 'nwiki') {
        $manager->add($content, FORMAT_NWIKI);
    } else if ($editor == 'wcode' ) {
    	$manager->add($content, FORMAT_WCODE);
    }

}

/**
 * Delete comments from database
 *
 * @param $idcomment. Id of comment which will be deleted
 * @param $context. Current context
 * @param $pageid. Current pageid
 **/
function wikicode_delete_comment($idcomment, $context, $pageid) {
    global $CFG;
    require_once($CFG->dirroot . '/comment/lib.php');

    list($context, $course, $cm) = get_context_info_array($context->id);
    $cmt = new stdClass();
    $cmt->context = $context;
    $cmt->itemid = $pageid;
    $cmt->area = 'wikicode_page';
    $cmt->course = $course;
    $cmt->component = 'mod_wikicode';

    $manager = new comment($cmt);
    $manager->delete($idcomment);

}

/**
 * Delete al comments from wiki
 *
 **/
function wikicode_delete_comments_wiki() {
    global $PAGE;

    $cm = $PAGE->cm;
    $context = get_context_instance(CONTEXT_MODULE, $cm->id);

    $table = 'comments';
    $select = 'contextid = ?';

    delete_records_select($table, $select, $context->id);

}

function wikicode_add_progress($pageid, $oldversionid, $versionid, $progress) {
        for ($v = $oldversionid + 1; $v <= $versionid; $v++) {
        $user = wikicode_get_wikicode_page_id($pageid, $v);

        insert_record('wikicode_progress', array('userid' => $user->userid, 'pageid' => $pageid, 'versionid' => $v, 'progress' => $progress));
    }
}

function wikicode_get_wikicode_page_id($pageid, $id) {
        return get_record('wikicode_versions', 'pageid', $pageid, 'id', $id);
}

function wikicode_print_page_content($page, $context, $subwikiid) {
    global $OUTPUT, $CFG;

    if ($page->timerendered + WIKI_REFRESH_CACHE_TIME < time()) {
        $content = wikicode_refresh_cachedcontent($page);
        $page = $content['page'];
    }

    if (isset($content)) {
        $box = '';
        foreach ($content['sections'] as $s) {
            $box .= '<p>' . get_string('repeatedsection', 'wikicode', $s) . '</p>';
        }

        if (!empty($box)) {
            echo $OUTPUT->box($box);
        }
    }

    $html = format_text(wikicode_remove_tags($page->cachedcontent), FORMAT_PLAIN, array('overflowdiv'=>true));

	print_simple_box_start('center','70%','','20');
    print_box($html);
	print_simple_box_end();
	
    wikicode_increment_pageviews($page);
	
}

/**
 * This function trims any given text and returns it with some dots at the end
 *
 * @param string $text
 * @param string $limit
 *
 * @return string
 */
function wikicode_trim_string($text, $limit = 25) {

    if (textlib::strlen($text) > $limit) {
        $text = textlib::substr($text, 0, $limit) . '...';
    }

    return $text;
}

/**
 * Prints default edit form fields and buttons
 *
 * @param string $format Edit form format (html, creole...)
 * @param integer $version Version number. A negative number means no versioning.
 */

function wikicode_print_edit_form_default_fields($format, $pageid, $version = -1, $upload = false, $deleteuploads = array()) {
    global $CFG, $PAGE, $OUTPUT;

    echo '<input type="hidden" name="sesskey" value="' . sesskey() . '" />';

    if ($version >= 0) {
        echo '<input type="hidden" name="version" value="' . $version . '" />';
    }

    echo '<input type="hidden" name="format" value="' . $format . '"/>';

    //attachments
    require_once($CFG->dirroot . '/lib/form/filemanager.php');

    $filemanager = new MoodleQuickForm_filemanager('attachments', get_string('wikiattachments', 'wikicode'), array('id' => 'attachments'), array('subdirs' => false, 'maxfiles' => 99, 'maxbytes' => $CFG->maxbytes));

    $value = file_get_submitted_draft_itemid('attachments');
    if (!empty($value) && !$upload) {
        $filemanager->setValue($value);
    }

    echo "<fieldset class=\"wiki-upload-section clearfix\"><legend class=\"ftoggler\">" . get_string("uploadtitle", 'wikicode') . "</legend>";

    echo $OUTPUT->container_start('mdl-align wiki-form-center aaaaa');
    print $filemanager->toHtml();
    echo $OUTPUT->container_end();

    $cm = $PAGE->cm;
    $context = get_context_instance(CONTEXT_MODULE, $cm->id);

    echo $OUTPUT->container_start('mdl-align wiki-form-center wiki-upload-table');
    wikicode_print_upload_table($context, 'wikicode_upload', $pageid, $deleteuploads);
    echo $OUTPUT->container_end();

    echo "</fieldset>";

    echo '<input class="wiki_button" type="submit" name="editoption" value="' . get_string('save', 'wikicode') . '"/>';
    echo '<input class="wiki_button" type="submit" name="editoption" value="' . get_string('upload', 'wikicode') . '"/>';
    echo '<input class="wiki_button" type="submit" name="editoption" value="' . get_string('preview') . '"/>';
    echo '<input class="wiki_button" type="submit" name="editoption" value="' . get_string('cancel') . '" />';
}

/**
 * Prints a table with the files attached to a wiki page
 * @param object $context
 * @param string $filearea
 * @param int $fileitemid
 * @param array deleteuploads
 */
function wikicode_print_upload_table($context, $filearea, $fileitemid, $deleteuploads = array()) {
    global $CFG, $OUTPUT;

    $htmltable = new html_table();

    $htmltable->head = array(get_string('deleteupload', 'wikicode'), get_string('uploadname', 'wikicode'), get_string('uploadactions', 'wiki'));

    $fs = get_file_storage();
    $files = $fs->get_area_files($context->id, 'mod_wikicode', $filearea, $fileitemid); //TODO: this is weird (skodak)

    foreach ($files as $file) {
        if (!$file->is_directory()) {
            $checkbox = '<input type="checkbox" name="deleteupload[]", value="' . $file->get_pathnamehash() . '"';

            if (in_array($file->get_pathnamehash(), $deleteuploads)) {
                $checkbox .= ' checked="checked"';
            }

            $checkbox .= " />";

            $htmltable->data[] = array($checkbox, '<a href="' . file_encode_url($CFG->wwwroot . '/pluginfile.php', '/' . $context->id . '/wikicode_upload/' . $fileitemid . '/' . $file->get_filename()) . '">' . $file->get_filename() . '</a>', "");
        }
    }

    print '<h3 class="upload-table-title">' . get_string('uploadfiletitle', 'wikicode') . "</h3>";
    print html_writer::table($htmltable);
}

/**
 * Generate wiki's page tree
 *
 * @param $page. A wiki page object
 * @param $node. Starting navigation_node
 * @param $keys. An array to store keys
 * @return an array with all tree nodes
 */
function wikicode_build_tree($page, $node, &$keys) {
    $content = array();
    static $icon;
    $icon = new pix_icon('f/odt', '');
    $pages = wikicode_get_linked_pages($page->id);
    foreach ($pages as $p) {
        $key = $page->id . ':' . $p->id;
        if (in_array($key, $keys)) {
            break;
        }
        array_push($keys, $key);
        $l = wikicode_parser_link($p);
        $link = new moodle_url('/mod/wikicode/view.php', array('pageid' => $p->id));
        $nodeaux = $node->add($p->title, $link, null, null, null, $icon);
        if ($l['new']) {
            $nodeaux->add_class('wikicode_newentry');
        }
        wikicode_build_tree($p, $nodeaux, $keys);
    }
    $content[] = $node;
    return $content;
}

/**
 * Get linked pages from page
 * @param int $pageid
 */
function wikicode_get_linked_pages($pageid) {
	
	global $CFG;
    
    $sql = "SELECT p.id, p.title
            FROM {$CFG->prefix}wikicode_pages p
            JOIN {$CFG->prefix}wikicode_links l ON l.topageid = p.id
            WHERE l.frompageid = ".$pageid."
            ORDER BY p.title ASC";
    return get_records_sql($sql);
}

/**
 * Get updated pages from wiki
 * @param int $pageid
 */
function wikicode_get_updated_pages_by_subwiki($swid) {
    global $USER, $CFG;

    $sql = "SELECT *
            FROM {$CFG->prefix}wikicode_pages
            WHERE subwikiid = ".$swid." AND timemodified > ".$USER->lastlogin."
            ORDER BY timemodified DESC";
    return get_records_sql($sql);
}
\end{lstlisting}