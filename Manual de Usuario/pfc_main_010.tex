%!TEX root=./pfc.tex
\chapter[Introducción]{\label{}
Introducción}

La programación informática está pasando a ser un área estratégica clave dentro de cualquier sector técnico de nuestra sociedad. Es por ello que no únicamente se da docencia de ella dentro del grado de Informática, sino que abarca la amplitud de los grados técnicos.

Sin embargo, es bastante amplia la dificultad que supone para alguien que no haya estado en contacto con el sector informático aprender a programar en la gran mayoría de ocasiones. Esto no sucede con todos los casos, por lo que en una amplia cantidad de ocasiones se puede aprovechar la capacidad de algún alumno que ya maneje estos conocimientos con anterioridad o simplemente tenga la habilidad para ello en beneficio del resto de compañeros.

Del mismo modo, conocemos que Moodle es la herramienta más utilizada por las universidades para la gestión de cursos de manera virtual. Esta herramienta nos ofrece la posibilidad de extenderla de la manera que consideremos oportuna, ya sea modificando módulos existentes o bien creando nuevos.

Es por ello, que se pretende desarrollar un nuevo módulo dentro del LMS Moodle, similar a una Wiki de edición de contenidos, pero que nos sirva para la edición colaborativa de código en lenguaje C. Del mismo modo, desde este mismo editor, debemos poder compilar como si de un entorno de desarrollo se tratara.

A su vez, aprovechando la potencia que nos ofrece Moodle, se asignarán herramientas de ayuda tanto al profesor y como al estudiante. Un profesor podrá agrupar la edición de código entre los alumnos que considere necesarios y ver una serie de estadísticas para hacerse una idea sobre el grado de dificultad que ha supuesto la práctica para los alumnos o grupo de estos. A su vez, el alumno podrá disponer de herramientas para interaccionar con el resto de componentes de su grupo o bien un historial de código para ver las variaciones que ha ido sufriendo este.